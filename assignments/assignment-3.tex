\renewcommand{\newpageorvspace}{\vspace{2em}}

\date{第三次作业}

\begin{document}

\maketitle

{\bf 习题2.2}

一般来说,将向量$\alpha_1,\cdots,\alpha_n$排成向量,组成矩阵$A$。求向量组$\{\alpha_1,\cdots,\alpha_n\}$的线性关系,即求齐次线性方程组
$$0 = Ax = (\alpha_1,\cdots,\alpha_n) \begin{pmatrix} x_{1} \\ \vdots \\ x_{n} \end{pmatrix}$$
的性质。事实上,如果$\{\alpha_{i_1},\cdots,\alpha_{i_m}\}$线性无关,那么
$$(\alpha_{i_1},\cdots,\alpha_{i_m}) \begin{pmatrix} x_{i_1} \\ \vdots \\ x_{i_m} \end{pmatrix} = 0$$
只有零解。于是,只要将$A$通过高斯消元法化为阶梯阵$B = (\beta_1,\cdots,\beta_n)$,将每行第一个非零元对应的列拿到一起,设其下标为$i_1,\cdots,i_m$,那么$\{\alpha_{i_1},\cdots,\alpha_{i_m}\}$就是一个极大线性无关组。这是因为,此时
$$(\beta_{i_1},\cdots,\beta_{i_m}) = \begin{pmatrix} 1 & & \makebox(0,0){\text{\huge $\ast$}} \\ & \ddots & \\ \makebox(0,0){\text{\huge\bf 0}} & & 1 \\ & & \\ & \makebox(0,0){\text{\huge\bf 0}} & \\ & & \end{pmatrix}$$
要注意的是改变$\alpha_1,\cdots,\alpha_n$排列次序组成矩阵$A$,可能得到不同的极大线性无关组。

第(1)问,全部四个向量$\alpha_1,\alpha_2,\alpha_3,\alpha_4$组成极大线性无关组。第(2)问的一个线性无关组为$\alpha_1,\alpha_2,\alpha_4$.

\newpageorvspace

{\bf 习题2.2 第2题}

将$\alpha_1,\cdots,\alpha_5$排成向量,组成矩阵$A$(可以顺序不同,但$\alpha_1,\alpha_2$排在前2位),并化为阶梯型
$$
A = \begin{pmatrix} 0 & 1 & 3 & 4 & 6 \\ 1 & 2 & 4 & 3 & 5 \\ 2 & 3 & 5 & 2 & 4 \\ 3 & 4 & 6 & 1 & 3 \end{pmatrix} \to
\begin{pmatrix} 1 & 0 & -2 & -5 & -7 \\ 0 & 1 & 3 & 4 & 6 \\ 0 & 0 & 0 & 0 & 0 \\ 0 & 0 & 0 & 0 & 0 \end{pmatrix}
$$
于是$\alpha_1,\alpha_2$线性无关,且式原向量组的一个极大线性无关组,不能再扩充了。

\newpageorvspace

{\bf 习题2.2 第3题}

\newpageorvspace

{\bf 习题2.2 第6题}

\newpageorvspace

{\bf 习题2.2 第7题}

\end{document}
