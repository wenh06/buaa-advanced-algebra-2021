\renewcommand{\newpageorvspace}{\vspace{2em}}

\date{第六次作业}

\begin{document}

\maketitle

{\bf 习题2.8 第1题}

利用过$A,B$点的一般的圆方程
$$\lambda (x^2+y^2-x+2y-10) + (x^2+y^2+3x-4y-1) = 0$$
令其过点$C = (2,0)$,解出$\lambda = 9/8$。再代入上述方程得
$$x^2+y^2+\dfrac{15}{17}x-\dfrac{14}{17}y-\dfrac{98}{17}=0$$

\newpageorvspace

{\bf 习题2.8 第2题}

由$x^2 = 5x - 6$解得$x = 2,3$。于是,$a_n = \alpha 2^n + \beta3^n$。利用$a_1=a_2=1$的条件解得$\alpha = 1, \beta = -1/3$。所以有
$$a_n = 2^n - 3^{n-1}$$

\newpageorvspace

{\bf 习题2.8 第3题}

一般给定$n$个点$(x_1,y_1),\cdots,(x_n,y_n)$,$x_1,\cdots,x_n$互不相同,则有拉格朗日插值多项式
\begin{align*}
& L(x) = \sum_{i=1}^{n}y_{i}\ell_{i}(x) \\
\text{其中,} & \ell_{i}(x) = \prod_{\begin{smallmatrix}1\leq m\leq k\\m\neq i\end{smallmatrix}}{\frac {x-x_{m}}{x_{i}-x_{m}}}={\frac {(x-x_{0})}{(x_{i}-x_{0})}}\cdots {\frac {(x-x_{i-1})}{(x_{i}-x_{i-1})}}{\frac {(x-x_{i+1})}{(x_{i}-x_{i+1})}}\cdots {\frac {(x-x_{n})}{(x_{i}-x_{n})}},
\end{align*}
满足$L(x_i) = y_i, i=1,\cdots,n$。本题对应的拉格朗日插值多项式为$f(x) = L(x) = \dfrac{1}{2}x^2 - \dfrac{1}{2}x + 1$. 当然也存在更高次多项式满足题目条件。

任取满足题设条件的多项式$f(x)$,考察$g(x) = f(x) - L(x)$。$1,2,3$是$g(x)$的根,于是存在$h(x)$使得
$$g(x) = f(x) - L(x) = (x-1)(x-2)(x-3) h(x)$$
于是在$\mathbb{R}[X]$中有
$$f(x) \equiv L(x) \mod (x-1)(x-2)(x-3)$$
由于$f(X)$在$\mathbb{R}[x]$中利用辗转相除法除以$(x-1)(x-2)(x-3)$所得的商多项式与余多项式都是整系数的,所以上式不可能成立。

\newpageorvspace

{\bf 习题3.1 第2题}(1)

$$\tau(n(n-1)\cdots 21) = (n-1) + \cdots + 1 = \dfrac{n(n-1)}{2}$$
于是$n(n-1)\cdots 21$是偶排列,当且仅当$n(n-1)\equiv 0 \mod 4$,即$n\equiv 0,1 \mod 4$

\newpageorvspace

{\bf 习题3.1 第3题}

令$A$为每个位置元素值都为$1$的$n$阶方阵,那么
$$0 = det(A) = \text{偶排列个数} - \text{奇排列个数}$$
所以$\text{偶排列个数} = \text{奇排列个数} = \dfrac{\text{全体排列个数}}{2} = \dfrac{n!}{2}$

\newpageorvspace

{\bf 习题3.1 第5题}

$x^4$的项:由于后3列只有一个$x$,于是必须选择$(2,2)$, $(3,3)$, $(4,4)$位的$x$,剩余只能选择$(1,1)$位的$x$。于是$x^4$的项为$x^4$。

$x^3$的项:由于后3列只有一个$x$,如果都选择相应位置的$x$,剩余只能选择$(1,1)$位的$x$,那么会得到$x^4$。于是后3列只能选取2个$x$,以下是可行的组合:
\begin{itemize}
    \item $(2,2)$, $(3,3)$, $(4,1)$选取$x$,相应的项为$(-1)^{\tau(4231)}3x^3$.
    \item $(3,3)$, $(4,4)$, $(2,1)$选取$x$,相应的项为$(-1)^{\tau(2134)}x^3$.
\end{itemize}
于是$x^3$的项为$-3x^3-x^3 = -4x^3$。

\newpageorvspace

\end{document}
