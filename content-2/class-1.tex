
% \renewcommand{\newpageorvspace}{\newpage}
\renewcommand{\newpageorvspace}{\vspace{2em}}

\date{2022-3-11  第一次习题课}

\begin{document}

\maketitle

{\bf 习题6.1 第2题}. (1)设$\mathscr{A, B}$是平面上绕原点分别旋转角$\alpha, \beta$的变换。试分别写出$\mathscr{A, B}$的矩阵$A,B$,计算$\mathscr{BA}$的矩阵$BA$,它表示什么变换?

(2)设在直角坐标平面上将$x$轴绕原点沿逆时针方向旋转角$\alpha, \beta$分别得到直线$\ell_{\alpha}, \ell_{\beta}$. $\mathscr{A, B}$是平面上的点分别关于直线$\ell_{\alpha}, \ell_{\beta}$作轴对称的变换。试分别写出$\mathscr{A, B}$的矩阵$A,B$,计算$\mathscr{BA}$的矩阵$BA$和计算$\mathscr{AB}$的矩阵$AB$,它们分别表示什么变换?

{\bf 解}:任取平面上非原点的一点,设其坐标为$(r\cos\theta, r\sin\theta)^T$, 那么其绕原点(逆时针)旋转$\alpha$角之后的坐标为
\begin{align*}
(r\cos(\theta+\alpha), r\sin(\theta+\alpha))^T & = (r\cos\theta\cos\alpha - r\sin\theta\sin\alpha, r\sin\theta\cos\alpha + r\cos\theta\sin\alpha)^T \\
& = \begin{pmatrix} \cos\alpha & -\sin\alpha \\ \sin\alpha & \cos\alpha \end{pmatrix} (r\cos\theta, r\sin\theta)^T.
\end{align*}
于是$A = \begin{pmatrix} \cos\alpha & -\sin\alpha \\ \sin\alpha & \cos\alpha \end{pmatrix}$, 同理$B = \begin{pmatrix} \cos\beta & -\sin\beta \\ \sin\beta & \cos\beta \end{pmatrix}$, $BA = \begin{pmatrix} \cos(\alpha+\beta) & -\sin(\alpha+\beta) \\ \sin(\alpha+\beta) & \cos(\alpha+\beta) \end{pmatrix}$. $\mathscr{BA}$表示的是平面上绕原点(逆时针)旋转角$\alpha+\beta$的变换。

(2)任取$r > 0$为实数,则$v_{\alpha} = (r\cos\alpha, r\sin\alpha)^T$为直线$\ell_{\alpha}$上的一个向量。在平面上任取一点$P = (x,y)^T$, 那么向量$v_P = \overrightarrow{OP}$在$v_{\alpha}$上的投影(即在直线$\ell_{\alpha}$上的投影)为$\frac{1}{r^2} \langle v_{\alpha}, v_P \rangle v_{\alpha}$. 点$P$指向投影点的向量为$-v_P + \frac{1}{r^2} \langle v_{\alpha}, v_P \rangle v_{\alpha}$, 所以$P$点关于直线$\ell_{\alpha}$对称点的坐标为
\begin{align*}
& \frac{1}{r^2} \langle v_{\alpha}, v_P \rangle v_{\alpha} + (-v_P + \frac{1}{r^2} \langle v_{\alpha}, v_P \rangle v_{\alpha}) \\
= & \frac{2}{r^2}\langle v_{\alpha}, v_P \rangle v_{\alpha} - v_P \\
= & \begin{pmatrix} 2x\cos^2\alpha + 2y\sin\alpha\cos\alpha - x \\ 2x\sin\alpha\cos\alpha + 2y\sin^2\alpha - y \end{pmatrix} \\
= & \begin{pmatrix} 2\cos^2\alpha - 1 & 2\sin\alpha\cos\alpha \\ 2\sin\alpha\cos\alpha & 2\sin^2\alpha - 1 \end{pmatrix} \begin{pmatrix} x \\ y \end{pmatrix} \\
= & \begin{pmatrix} \cos 2\alpha & \sin 2\alpha \\ \sin 2\alpha & -\cos 2\alpha \end{pmatrix} \begin{pmatrix} x \\ y \end{pmatrix}
\end{align*}
所以$\mathscr{A}$对应的矩阵$A = \begin{pmatrix} \cos 2\alpha & \sin 2\alpha \\ \sin 2\alpha & -\cos 2\alpha \end{pmatrix}$. 同理$\mathscr{B}$对应的矩阵$B = \begin{pmatrix} \cos 2\beta & \sin 2\beta \\ \sin 2\beta & -\cos 2\beta \end{pmatrix}$.

$BA = \begin{pmatrix} \cos 2\beta & \sin 2\beta \\ \sin 2\beta & -\cos 2\beta \end{pmatrix} \begin{pmatrix} \cos 2\alpha & \sin 2\alpha \\ \sin 2\alpha & -\cos 2\alpha \end{pmatrix} = \begin{pmatrix} \cos 2(\beta-\alpha) & -\sin 2(\beta-\alpha) \\ \sin 2(\beta-\alpha) & \cos 2(\beta-\alpha) \end{pmatrix}$, 代表绕原点旋转$2(\beta-\alpha)$角度的变换。
同理,$AB = \begin{pmatrix} \cos 2(\alpha-\beta) & -\sin 2(\alpha-\beta) \\ \sin 2(\alpha-\beta) & \cos 2(\alpha-\beta) \end{pmatrix}$, 代表绕原点旋转$2(\alpha-\beta)$角度的变换。

\newpageorvspace


{\bf 习题6.1 第3题}. 由2阶可逆实方阵$A$在直角坐标平面$\mathbb{R}^2$上定义可逆线性变换$\mathscr{A}: \begin{pmatrix} x \\ y \end{pmatrix} \mapsto A \begin{pmatrix} x \\ y \end{pmatrix}$.

(1)$\mathscr{A}$将平行四边形$ABCD$变到平行四边形$A'B'C'D'$, 求证:变换后和变换前的面积比$k = \dfrac{S_{A'B'C'D'}}{S_{ABCD}} = |\det A|$;由此可以得出平面上任何图形经过变换$\mathscr{A}$之后的面积为变换前的$|\det A|$倍。

(2)用线性变换$\mathscr{A}: \begin{pmatrix} x \\ y \end{pmatrix} \mapsto \begin{pmatrix} 1 & 0 \\ 0 & \frac{b}{a} \end{pmatrix} \begin{pmatrix} x \\ y \end{pmatrix}$将圆$C: x^2 + y^2 = a^2$变成椭圆$C_1: \frac{x^2}{a^2} + \frac{y^2}{b^2} = 1$. 利用$C_1$与$C$的面积比得出椭圆$C_1$面积公式。

(3)画出下图经过线性变换$\mathscr{A}: \begin{pmatrix} x \\ y \end{pmatrix} \mapsto \begin{pmatrix} 1.2 & -0.8 \\ -0.4 & 1.1 \end{pmatrix} \begin{pmatrix} x \\ y \end{pmatrix}$得到的图形。

\newcommand*{\xMin}{0}
\newcommand*{\xMax}{4}
\newcommand*{\yMin}{0}
\newcommand*{\yMinplusone}{1}
\newcommand*{\yMax}{4}
\begin{figure}[H]
\centering
\begin{tikzpicture}[scale=1.1, transform shape]
\draw[gray,very thin] (\xMin,\yMin) grid (\xMax,\yMax);
\foreach \i in {\xMin,...,\xMax} {
    \node [below] at (\i,\yMin) {$\i$};
}
\foreach \i in {\yMinplusone,...,\yMax} {
    \node [left] at (\xMin,\i) {$\i$};
}
\draw[red, thick] (0,0) -- (4,0);
\draw[blue, thick] (0,0) -- (0,4);
\draw[gray] (2,2) circle (2);
\end{tikzpicture}
\end{figure}


{\bf 证明}:(1)令向量$\overrightarrow{AB} = v, \overrightarrow{AC} = u$(列向量), 那么平行四边形$ABCD$的面积$S_{ABCD} = |\det(u ,v)|$. 同理平行四边形$A'B'C'D'$的面积有
$$S_{A'B'C'D'} = |\det(Au, Av)| = |\det(A(u, v))| = |\det A| \cdot |\det (u, v))| = |\det A| S_{ABCD}$$
平面上任意图形可以用正方形覆盖(的极限)计算面积,因此其经过变换$\mathscr{A}$之后的面积为变换前的$|\det A|$倍。

(2)由第(1)问知$S_{C_1} = |\det A|\cdot S_{C} = |b/a|\cdot \pi a^2 = \pi |ab|$, 其中$A = \begin{pmatrix} 1 & 0 \\ 0 & \frac{b}{a} \end{pmatrix}$

(3)
\begin{figure}[H]
\centering
\begin{tikzpicture}[scale=1.1, transform shape]
    \foreach \i [evaluate=\i as \xstart using (1.2*\i-0.8*\yMin), evaluate=\i as \ystart using (-0.4*\i+1.1*\yMin), evaluate=\i as \xend using (1.2*\i-0.8*\yMax), evaluate=\i as \yend using (-0.4*\i+1.1*\yMax)] in {\xMin,...,\xMax} {
        \draw [very thin,gray] (\xstart,\ystart) -- (\xend,\yend);
    }
    \pgfmathsetmacro{\xend}{1.2*\xMax-0.8*\yMin}
    \pgfmathsetmacro{\yend}{-0.4*\xMax+1.1*\yMin}
    \draw [red, very thick] (0,0) -- (\xend,\yend);
    
    \foreach \i [evaluate=\i as \xstart using (1.2*\xMin-0.8*\i), evaluate=\i as \ystart using (-0.4*\xMin+1.1*\i), evaluate=\i as \xend using (1.2*\xMax-0.8*\i), evaluate=\i as \yend using (-0.4*\xMax+1.1*\i)] in {\yMin,...,\yMax} {
        \draw [very thin,gray] (\xstart,\ystart) -- (\xend,\yend);
    }
    \pgfmathsetmacro{\xend}{1.2*\xMin-0.8*\yMax}
    \pgfmathsetmacro{\yend}{-0.4*\xMin+1.1*\yMax}
    \draw [blue, very thick] (0,0) -- (\xend,\yend);
    
    \pgfmathsetmacro{\xb}{1.2*(\xMin+\xMax)*0.5-0.8*\yMin}
    \pgfmathsetmacro{\yb}{-0.4*(\xMin+\xMax)*0.5+1.1*\yMin}
    \pgfmathsetmacro{\xl}{1.2*\xMax-0.8*(\yMin+\yMax)*0.5}
    \pgfmathsetmacro{\yl}{-0.4*\xMax+1.1*(\yMin+\yMax)*0.5}
    \pgfmathsetmacro{\xa}{1.2*(\xMin+\xMax)*0.5-0.8*\yMax}
    \pgfmathsetmacro{\ya}{-0.4*(\xMin+\xMax)*0.5+1.1*\yMax}
    \pgfmathsetmacro{\xr}{1.2*\xMin-0.8*(\yMin+\yMax)*0.5}
    \pgfmathsetmacro{\yr}{-0.4*\xMin+1.1*(\yMin+\yMax)*0.5}
    \draw [gray] plot [smooth cycle] coordinates {(\xb,\yb) (\xl,\yl) (\xa,\ya) (\xr,\yr)};
\end{tikzpicture}
\end{figure}

\newpageorvspace


{\bf 习题6.2 第11题}. 设$F_n[x]$是数域$F$上次数低于$n$的一元多项式组成的$n$维空间,$n \geqslant 2$。$\mathscr{A}: f(x) \mapsto f(x+1)$与 $\mathscr{D}: f(x) \mapsto f'(x)$是$F_n[x]$的线性变换。求证
$$\mathscr{A} = \mathscr{I} + \dfrac{\mathscr{D}}{1!} + \dfrac{\mathscr{D}^2}{2!} + \cdots + \dfrac{\mathscr{D}^{n-1}}{(n-1)!}$$

{\bf 证明}:令$\mathscr{B} = \mathscr{A} - \left( \mathscr{I} + \dfrac{\mathscr{D}}{1!} + \dfrac{\mathscr{D}^2}{2!} + \cdots + \dfrac{\mathscr{D}^{n-1}}{(n-1)!} \right)$. 任取$f \in F_n[x]$, 令$g = \mathscr{B}(f) \in F_n[x]$. 将$g$写为$g(x) = a_0 + a_1x + \cdots + a_{n-1}x^{n-1}$. 那么$\mathscr{D}^{n-1}(g) = (n-1)! a_{n-1}$. 另一方面,
$$\mathscr{D}^{n-1} \mathscr{B} = \mathscr{D}^{n-1} \mathscr{A} - \left( \mathscr{D}^{n-1} + \dfrac{\mathscr{D}^n}{1!} + \dfrac{\mathscr{D}^{n+1}}{2!} + \cdots + \dfrac{\mathscr{D}^{2n-2}}{(n-1)!} \right) = \mathscr{D}^{n-1} \mathscr{A} - \mathscr{D}^{n-1} = \mathscr{O}.$$
其中$\mathscr{O}$为$F_n[x]$上的零变换,将任意元素映为$0$. 于是$a_{n-1} = 0$, 从而有$g\in F_{n-1}[x] \subset F_n[x]$. 再依次考察$\mathscr{D}^{n-2}(g), \cdots, \mathscr{D}(g)$, 可得出$a_{n-2} = \cdots = a_1 = 0$. 令$f(x) = b_0 + b_1x + \cdots + b_{n-1}x^{n-1}$, 那么
$$a_0 = g(0) = f(1) - \left( f(0) + \dfrac{f'(0)}{1!} + \dfrac{f''(0)}{2!} + \cdots + \dfrac{f^{(n-1)}(0)}{(n-1)!} \right) = \sum_{i=0}^{n-1} b_i - \sum_{i=0}^{n-1} b_i = 0.$$


\newpageorvspace


{\bf 习题6.2 第4题}. 设$\mathbb{R}_n[t]$是实数域$\mathbb{R}$上以$t$为字母、次数$<n$的多项式及零组成的线性空间。$V=\left\{ f(\cos x) \ |\ f \in \mathbb{R}_n[t] \right\}$. 试写出$V$中的基$M_1 = \left\{ 1, \cos x, \cos^2 x, \cdots, \cos^{n-1} x \right\}$到$M_2 = \left\{ 1, \cos x, \cos 2x, \cdots, \cos (n-1)x \right\}$的过渡矩阵。

{\bf 解}. 有$\cos kx + i\sin kx = e^{ikx} = \left( e^{ix} \right)^k = \left( \cos x + i\sin x \right)^k$, 那么
\begin{align*}
\cos kx & = \mathfrak{R}\mathfrak{e}\left( \cos x + i\sin x \right)^k \\
& = \sum_{t=0}^{\lfloor k/2 \rfloor} (-1)^t C_k^{2t} \cos^{k-2t} x \sin^{2t} x \\
& = \sum_{t=0}^{\lfloor k/2 \rfloor} (-1)^t C_k^{2t} \cos^{k-2t} x (1-\cos^2x)^t \\
& = \sum_{t=0}^{\lfloor k/2 \rfloor} (-1)^t C_k^{2t} \cos^{k-2t} x \sum_{s=0}^t (-1)^s C_t^s \cos^{2s} x \\
& = \sum_{t=0}^{\lfloor k/2 \rfloor} C_k^{2t} \sum_{s=0}^t (-1)^{t+s} C_t^s \cos^{k-2t+2s} x \\
& = \sum_{t=0}^{\lfloor k/2 \rfloor} C_k^{2t} \sum_{s=0}^t (-1)^{2t-s} C_t^{t-s} \cos^{k-2s} x \\
& = \sum_{t=0}^{\lfloor k/2 \rfloor} C_k^{2t} \sum_{s=0}^t (-1)^s C_t^s \cos^{k-2s} x \\
& = \sum_{t=0}^{\lfloor k/2 \rfloor} (-1)^t \left(\sum_{m=t}^{\lfloor k/2 \rfloor}  C_k^{2m} C_m^t \right) \cos^{k-2t} x \\
\end{align*}
设过渡矩阵为$A$, 即$M_1A = M_2$, 那么$A$的第$k$列为
\begin{align*}
\left(0, (-1)^{\lfloor k/2 \rfloor} k\cos x, 0, \cdots, 0, (-1)^t \left(\sum_{m=t}^{\lfloor k/2 \rfloor}  C_k^{2m} C_m^t \right), 0, \cdots 0, \sum_{m=0}^{\lfloor k/2 \rfloor}  C_k^{2m}, 0, \cdots \right)^T, & \text{$k$为奇数} \\
\left((-1)^{\lfloor k/2 \rfloor} k\cos x, 0, \cdots, 0, (-1)^t \left(\sum_{m=t}^{\lfloor k/2 \rfloor}  C_k^{2m} C_m^t \right), 0, \cdots 0, \sum_{m=0}^{\lfloor k/2 \rfloor}  C_k^{2m}, 0, \cdots \right)^T, & \text{$k$为偶数} \\
\end{align*}
上式中的$t$对应相应向量的第$k-2t$位。以上即为第一类切比雪夫多项式(Chebyshev polynomials of the first kind)的系数。第一类切比雪夫多项式的一种定义方式即为
$$T_k(\cos x) = \cos (kx)$$


\end{document}
