
% \renewcommand{\newpageorvspace}{\newpage}
\renewcommand{\newpageorvspace}{\vspace{2em}}

\date{2022-3-25  第二次习题课}

\begin{document}

\maketitle

{\bf 第一题}. 设$A \in M_n (\mathbb{F})$, 并且$A^2 = I$. 证明$A$相似于矩阵$B = \begin{pmatrix} I_r & \\ & -I_s \end{pmatrix}$, 其中$r+s=n$.

{\bf 证明}:

\newpageorvspace


{\bf 第二题}. 若$A \in M_n (\mathbb{C})$满足$A^n = I$, 证明$A$的特征值是$n$次单位根。

{\bf 证明}:

\newpageorvspace


{\bf 第三题}. 设$A, B \in M_n (\mathbb{F})$, $AB=BA$, 若$A,B$均相似于对角矩阵,证明存在可逆矩阵$P$使得$P^{-1}AP$与$P^{-1}BP$同时为对角形。

{\bf 证明}:


\newpageorvspace


{\bf 习题6.2 第四题}. 设$V = M_n (\mathbb{F})$, $A,B \in M_n (\mathbb{F})$, 且满足$AB=BA$以及$A,B$均相似于对角矩阵。在$V$中定义线性变换$\sigma: X \mapsto AX - XB, \forall X \in V$. 判断$\sigma$是否可对角化并证明你的结论。

{\bf 解}. 


\end{document}
