
% \renewcommand{\newpageorvspace}{\newpage}
\renewcommand{\newpageorvspace}{\vspace{2em}}

\date{2022-4-8  第三次习题课}

\begin{document}

\maketitle


{\bf 第一题(习题7.1第2题)}. 已知5阶方阵$A$相似与Jordan形矩阵$J$,且满足条件
$$\operatorname{rank}A = 3, \operatorname{rank}A^2 = 2, \operatorname{rank}(A+I) = 4, \operatorname{rank}(A+I)^2 = 3.$$
求$J$.

{\bf 解}. 由于$\operatorname{rank}A = 3 < 5$, 所以$\det A = 0$, 故$0$是$A$的特征值. 类似可知$-1$也是$A$的特征值。由于$\operatorname{rank}A^2 = 2, \operatorname{rank}(A+I)^2 = 3$, 即知
$$\dim \ker A^2 + \dim \ker (A+I)^2 = (5-2) + (5-3) = 5,$$
故(可以考虑根子空间分解)$A$的特征值只有$-1, 0$. 并且由上式可知$\operatorname{rank}A^k = 2, \operatorname{rank}(A+I)^k = 3$对任意$k\geqslant 2$成立。于是,由定理7.1.1知,$J$中
\begin{itemize}
\item $1$阶Jordan块$J_1(0)$的数量为$(5 - \operatorname{rank}A) - (\operatorname{rank}A - \operatorname{rank}A^2) = 1$;
\item $2$阶Jordan块$J_2(0)$的数量为$(\operatorname{rank}A - \operatorname{rank}A^2) = 1$;
\item $1$阶Jordan块$J_1(-1)$的数量为$(5 - \operatorname{rank}(A+I)) - (\operatorname{rank}(A+I) - \operatorname{rank}(A+I)^2) = 0$;
\item $2$阶Jordan块$J_2(-1)$的数量为$(\operatorname{rank}(A+I) - \operatorname{rank}(A+I)^2) = 1$.
\end{itemize}
所以
$$J = \operatorname{diag} (J_1(0), J_2(0), J_2(-1)).$$


\newpageorvspace


{\bf 第二题(习题7.1第3题第(2)问)}. 已知矩阵$A = \begin{pmatrix} 4 & 0 & 0 & 0 \\ 0 & 4 & 0 & 0 \\ 3 & 0 & 4 & 0 \\ 2 & 3 & 0 & 4 \end{pmatrix}$相似于Jordan形$J$. 根据条件$\operatorname{rank}(A - \lambda_i I)^k = \operatorname{rank}(J - \lambda_i I)^k$($\lambda_i$取遍$A$的各特征值,$k=1,2,\ldots$),求$J$.

{\bf 解}:容易计算$A$的特征多项式为$f(\lambda) = \det (\lambda I - A) = (\lambda - 4)^4$, 于是矩阵$A$有$4$重特征值$4$. 那么
\begin{itemize}
\item $\operatorname{rank}(A - 4 I) = \operatorname{rank}(J - 4 I) = \operatorname{rank} \begin{pmatrix} 0 & 0 & 0 & 0 \\ 0 & 0 & 0 & 0 \\ 3 & 0 & 0 & 0 \\ 2 & 3 & 0 & 0 \end{pmatrix} = 2;$
\item $\operatorname{rank}(A - 4 I)^k = \operatorname{rank}(J - 4 I)^k = \operatorname{rank} \mathbf{0} = 0, \forall k \geqslant 2.$
\end{itemize}
那么由定理7.1.1知
\begin{itemize}
\item $1$阶Jordan块$J_1(4)$的数量为$(4 - \operatorname{rank}(A - 4 I)) - (\operatorname{rank}(A - 4 I) - \operatorname{rank}(A - 4 I)^2) = 0$;
\item $2$阶Jordan块$J_2(4)$的数量为$(\operatorname{rank}(A - 4 I) - \operatorname{rank}(A - 4 I)^2) = 2$.
\end{itemize}
所以
$$J = \operatorname{diag} (J_2(4), J_2(4)).$$


\newpageorvspace


{\bf 第三题}. 设$A = J_5(0)^2$相似于一个Jordan形矩阵$J$, 求$J$.

{\bf 解}:易知$\operatorname{rank}A = \operatorname{rank} J_5(0)^2 = 3$, $\operatorname{rank}A^2 = \operatorname{rank} J_5(0)^4 = 1;$ $\operatorname{rank}A^k = \operatorname{rank} J_5(0)^{2k} = 0;$ $k \geqslant 3$. 那么由定理7.1.1知
\begin{itemize}
\item $1$阶Jordan块$J_1(0)$的数量为$(5 - \operatorname{rank}A) - (\operatorname{rank}A - \operatorname{rank}A^2) = 0$;
\item $2$阶Jordan块$J_2(0)$的数量为$(\operatorname{rank}A - \operatorname{rank}A^2) - (\operatorname{rank}A^2 - \operatorname{rank}A^3) = 1$.
\item $3$阶Jordan块$J_3(0)$的数量为$(\operatorname{rank}A^2 - \operatorname{rank}A^3) = 1$.
\end{itemize}
所以
$$J = \operatorname{diag} (J_2(0), J_3(0)).$$


\newpageorvspace


{\bf 第四题}. 设$\mathscr{A} \in \mathcal{L}(V)$在基$M$下的矩阵是$A\in M_n(\mathbb{C})$. 若$d_A(\lambda) = (\lambda-\lambda_0)^m g(\lambda)$, $g(\lambda_0) \neq 0$, 是$A$的一个极小多项式,以及$u(\lambda), v(\lambda) \in \mathbb{C}[\lambda]$使得
$$u(\lambda)(\lambda-\lambda_0)^m + v(\lambda)g(\lambda) = 1.$$
令$W = u(\mathscr{A})(\mathscr{A} - \lambda_0\mathscr{I})^m V$. 证明$(\mathscr{A} - \lambda_0\mathscr{I})|_W$是可逆线性变换。

{\bf 证明}:首先,容易看出$W$是$(\mathscr{A} - \lambda_0\mathscr{I})$的不变子空间。由于$d_A(\lambda)$是$A$的一个极小多项式,所以$d_A(\mathscr{A}) = \mathscr{O}$为$V$上的零映射。

任取$\alpha \in \ker (\mathscr{A} - \lambda_0\mathscr{I})|_W \subset W$. 由于$W = u(\mathscr{A})(\mathscr{A} - \lambda_0\mathscr{I})^m V$, 所以存在$\beta \in V$, 使得
$$\alpha = u(\mathscr{A})(\mathscr{A} - \lambda_0\mathscr{I})^m \beta.$$
又由于$u(\lambda)(\lambda-\lambda_0)^m + v(\lambda)g(\lambda) = 1,$ 所以有$u(\mathscr{A})(\mathscr{A} - \lambda_0\mathscr{I})^m \beta = (\mathscr{I} - v(\mathscr{A})g(\mathscr{A})) \beta$. 所以有
\begin{align*}
0 = (\mathscr{A} - \lambda_0\mathscr{I})^m \alpha & = (\mathscr{A} - \lambda_0\mathscr{I})^m u(\mathscr{A})(\mathscr{A} - \lambda_0\mathscr{I})^m \beta \\
& = (\mathscr{A} - \lambda_0\mathscr{I})^m (\mathscr{I} - v(\mathscr{A})g(\mathscr{A})) \beta \\
& = (\mathscr{A} - \lambda_0\mathscr{I})^m \beta - v(\mathscr{A}) (\mathscr{A} - \lambda_0\mathscr{I})^m g(\mathscr{A}) \beta \\
& = (\mathscr{A} - \lambda_0\mathscr{I})^m \beta - v(\mathscr{A}) d_A(\mathscr{A}) \beta \\
& = (\mathscr{A} - \lambda_0\mathscr{I})^m \beta - v(\mathscr{A}) \mathscr{O} \beta \\
& = (\mathscr{A} - \lambda_0\mathscr{I})^m \beta
\end{align*}
从而知$\alpha = u(\mathscr{A})(\mathscr{A} - \lambda_0\mathscr{I})^m \beta = 0$, 所以$\ker (\mathscr{A} - \lambda_0\mathscr{I})|_W = \{0\}$, 故$(\mathscr{A} - \lambda_0\mathscr{I})|_W$是$W$上的可逆线性变换。


\newpageorvspace


{\bf 第五题(习题7.1第1题)}. 已知$A = \begin{pmatrix} 0 & 0 & -2 \\ 1 & 0 & 3 \\ 0 & 1 & 0 \end{pmatrix}$, 求$A^n$.

{\bf 解}. $A$的特征多项式
$$f_A(\lambda) = \det (\lambda I_3 - A) = \det \begin{pmatrix} \lambda & 0 & 2 \\ -1 & \lambda & -3 \\ 0 & -1 & \lambda \end{pmatrix} = \lambda \begin{vmatrix} \lambda & -3 \\ -1 & \lambda \end{vmatrix} - (-1) \begin{vmatrix} 0 & 2 \\ -1 & \lambda \end{vmatrix} = \lambda(\lambda^2-3) + 2.$$
可算得$A$的特征值为$1, 1, -2$.

由$(A - I) v = 0$解得$v = k \begin{pmatrix} 2 \\ -1 \\ -1 \end{pmatrix}$, $k \neq 0$. 由此可知$A$的Jordan标准形有$1$个$2$阶Jordan块$J_2(1)$, 以及$1$个$1$阶Jordan块$J_1(-2)$.

继续解$(A - I) v' = v$得$v' = k' \begin{pmatrix} 2 \\ -1 \\ -1 \end{pmatrix} - k \begin{pmatrix} 2 \\ 1 \\ 0 \end{pmatrix}$ (和之前同一个$k$. $k'$任取)。

解$(A + 2I) w = w$得$w = t \begin{pmatrix} 1 \\ -2 \\ 1 \end{pmatrix}$, $t \neq 0$. 那么令$k = t = 1, k' = 0$, 取$P = (v, v', w) = \begin{pmatrix} 2 & -2 & 1 \\ -1 & -1 & -2 \\ -1 & 0 & 1 \end{pmatrix}$, 即有
$$AP = (Av, Av', Aw) = (v, v+v', -2w) = (v, v', w) \begin{pmatrix} 1 & 1 & 0 \\ 0 & 1 & 0 \\ 0 & 0 & -2 \end{pmatrix} = P \cdot \operatorname{diag}(J_2(1), J_1(-2)).$$
从而有$A = P \cdot \operatorname{diag}(J_2(1), J_1(-2)) \cdot P^{-1}$, 其中$P^{-1} = \dfrac{1}{9} \begin{pmatrix} 1 & -2 & -5 \\ -3 & -3 & -3 \\ 1 & -2 & 4 \end{pmatrix}$. 所以
\begin{align*}
A^n & = (P \cdot \operatorname{diag}(J_2(1), J_1(-2)) \cdot P^{-1})^n = P \cdot \operatorname{diag}(J_2(1)^n, J_1(-2)^n) \cdot P^{-1} \\
& = P \cdot \begin{pmatrix} 1 & n & 0 \\ 0 & 1 & 0 \\ 0 & 0 & (-2)^n \end{pmatrix} \cdot P^{-1} \\
& = \dfrac{1}{9} \begin{pmatrix} 2 & -2 & 1 \\ -1 & -1 & -2 \\ -1 & 0 & 1 \end{pmatrix} \cdot \begin{pmatrix} 1 & n & 0 \\ 0 & 1 & 0 \\ 0 & 0 & (-2)^n \end{pmatrix} \cdot \begin{pmatrix} 1 & -2 & -5 \\ -3 & -3 & -3 \\ 1 & -2 & 4 \end{pmatrix} \\
& = \dfrac{1}{9} \begin{pmatrix} 8-6n+(-2)^{n} & 2-6n+(-2)^{n+1} & -4-6n+(-2)^{n+2} \\ 2+3n+(-2)^{n+1} & 5+3n+(-2)^{n+2} & 8+3n+(-2)^{n+3} \\ -1+3n+(-2)^{n} & 2+3n+(-2)^{n+1} & 5+3n+(-2)^{n+2} \end{pmatrix}
\end{align*}

\newpageorvspace


{\bf 第六题(习题7.1第4题)}. (1) 已知Jordan形矩阵$J$满足条件$\operatorname{rank} J^k = \operatorname{rank} J^{k+1} = r$, 根据$J^k$所满足的条件,对任意正整数$s$求$\operatorname{rank} J^{k+s}$.

(2) 已知方阵$A$相似于Jordan形矩阵$J$. 且$\operatorname{rank} A^k = \operatorname{rank} A^{k+1} = r$. 对任意正整数$s$求$\operatorname{rank} A^{k+s}$.

{\bf 解}. (1) 考察任意的Jordan块$J_m(\lambda) = \lambda I_m + \Lambda$, 其中$\Lambda$为次对角线($(i,i+1)$位,$i=1,\ldots,m-1$)元素值为$1$, 其余位置元素值为$0$的方阵。$\Lambda$和$\lambda I_m$可交换,且有
$$J_m(\lambda)^k = (\lambda I_m + \Lambda)^k = \sum\limits_{i=0}^k C_{k}^i \lambda^{k-i} I_m \Lambda^{i}.$$
那么
\begin{itemize}
\item 若$\lambda = 0$, 那么$\operatorname{rank} J_m(\lambda)^k = \operatorname{rank} \Lambda^k = \max\{0, m-k\}$;
\item 若$\lambda \neq 0$, 那么$\operatorname{rank} J_m(\lambda)^k = m$;
\end{itemize}
令$k_0$为$J$中对应于特征值$0$的Jordan块的阶数的最大值(若没有这样的Jordan块则令$k_0 = 0$)。由以上讨论,以及由题设条件$\operatorname{rank} J^k = \operatorname{rank} J^{k+1} = r$知$k \geqslant k_0$, 否则$\operatorname{rank} J^{k+1}$必然小于$\operatorname{rank} J^k$. 对于任意的$k'\geqslant k \geqslant k_0$, 有
$$\operatorname{rank} J^{k_0} = \operatorname{rank} J^{k_0+1} = \cdots = \operatorname{rank} J^{k'} = r,$$
故对任意的正整数$s$, 有$\operatorname{rank} J^{k+s} = r$.

(2) 设有可逆矩阵$P$, 使得$A = PJP^{-1}$, 其中$J$为$A$的Jordan标准形。那么$\operatorname{rank} A = \operatorname{rank} J$. 由第(1)问可知对任意正整数$s$求$\operatorname{rank} A^{k+s} = r$.

\newpageorvspace


{\bf 第七题(习题7.1第5题)}. 已知$n$阶方阵$A \in \mathbb{F}^{n\times n}$相似于Jordan形矩阵$J$, 且满足条件$A^n = O \neq A^{n-1}$. 求$J$.

{\bf 解}. 由条件$A^n = O \neq A^{n-1}$知$A$的极小多项式为$f_A(\lambda) = \lambda^n$, 所以$A$的特征值都是$0$, 其Jordan型可写为
$$J = \operatorname{diag} (J_{r_1}(0), \ldots, J_{r_m}(0)), \quad r_1 + \cdots + r_m = n.$$
由第六题的讨论知,
$$\operatorname{rank} J_{r_i}(0)^k = \max\{0, r_i-k\}.$$
所以
$$\operatorname{rank} A^k = \operatorname{rank} J^k = \sum\limits_{i=1}^m \max\{0, r_i-k\}.$$
由条件$A^n = O \neq A^{n-1}$知$\operatorname{rank} A^{n-1} > 0$, 所以我们有
$$
\begin{cases}
\sum\limits_{i=1}^m \max\{0, r_i - (n-1)\} > 0, \\
\sum\limits_{i=1}^m r_i = n.
\end{cases}
$$
于是必然有$m = 1, r_1 = n$. 故$J = J_n(0)$.


% \newpageorvspace


% {\bf 第八题(习题7.1第3题)}. 已知下面的矩阵$A$相似与Jordan形$J$. 根据条件$\operatorname{rank}(A - \lambda_i I)^k = \operatorname{rank}(J - \lambda_i I)^k$($\lambda_i$取遍$A$的各特征值,$k=1,2,\ldots$),求$J$.

% (3) $\begin{pmatrix} 1 & 2 & 4 & 7 \\ 0 & 1 & 3 & 6 \\ 0 & 0 & 1 & 4 \\ 0 & 0 & 0 & 3 \end{pmatrix}$; \quad (4) $\begin{pmatrix} 4 & -3 & 0 & 0 \\ -3 & 2 & 0 & 0 \\ 1 & 2 & -3 & 2 \\ 4 & 3 & 8 & 5 \end{pmatrix}$.

% {\bf 解}. 


\end{document}
