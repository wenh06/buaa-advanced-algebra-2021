\usepackage{relsize}



\renewcommand{\newpageorvspace}{\newpage}
% \renewcommand{\newpageorvspace}{\vspace{2em}}

\date{2022-5-6  第五次习题课}

\begin{document}

\maketitle

% {\larger

{\bf 第一题}. 习题6.8第4题涉及的牛顿公式:记$f_n(X) = X^n + a_1 X^{n-1} + \cdots + a_n = X_n + (-1)^1 \sigma_1 X^{n-1} + \cdots + (-1)^n \sigma_n$, 其复根$k$次幂之和记为$S_k = S_k(f_n)$. 那么当$m \leqslant n$时,有
$$S_m + a_1S_{m-1} + \cdots + a_{m-1}S_1 + ma_m = 0;$$
当$m > n$时,有
$$S_m + a_1S_{m-1} + \cdots + a_nS_{m-n} = 0.$$

\newpageorvspace

{\bf 证明}. 记$f_n(X)$的所有复根为$x_1,\ldots,x_n$. 那么
\begin{align*}
S_k & = \sum\limits_{i=1}^n x^k, \\
\sigma_k & = \operatorname{Sym}_k^1(x_1,\ldots,x_n) = \sum\limits_{\substack{0 \leqslant k_1,\ldots,k_n \leqslant 1 \\ k_1+\cdots+k_n=k}} x_1^{k_1} \cdots x_n^{k_n} \\
& = \sum\limits_{1 \leqslant h_1 < \cdots < h_k \leqslant n} x_{h_1} \cdots x_{h_k}
\end{align*}
我们有
\begin{align*}
\sigma_1 S_{m-1} & = (x_1 + \cdots + x_n) (x_1^{m-1} + \cdots + x_n^{m-1}) = S_m + \sum_{\substack{1 \leqslant i,j \leqslant n \\ i\neq j}} x_i^{m-1} x_j = {\color{red} S_m} + \mathcal{S}(x_1^{m-1}x_2) \\
\sigma_2 S_{m-2} & = \left( \sum_{\substack{1 \leqslant i,j \leqslant n \\ i\neq j}} x_i x_j \right) (x_1^{m-2} + \cdots + x_n^{m-2}) = \mathcal{S}(x_1^{m-1}x_2) + \mathcal{S}(x_1^{m-2}x_2x_3) \\
& \vdots \\
\sigma_k S_{m-k} & = \left( \sum\limits_{1 \leqslant h_1 < \cdots < h_k \leqslant n} x_{h_1} \cdots x_{h_k} \right) (x_1^{m-k} + \cdots + x_n^{m-k}) \\
& = \mathcal{S}(x_1^{m-k+1}x_2\cdots x_k) + \mathcal{S}(x_1^{m-k}x_2\cdots x_{k+1})
\end{align*}
以上的$\mathcal{S}(x_1,\cdots,x_n) = \mathcal{S}(\ast)$为首项为$\ast$的对称多项式,即满足对于任意的$n$阶置换$s\in \mathcal{S}_n$, 有$\mathcal{S}(x_{s(1)},\cdots,x_{s(n)}) = \mathcal{S}(x_1,\cdots,x_n)$. 可以通过对首项的下标用$n$阶置换群$\mathcal{S}_n$作用,剔除重复项之后求和得到。那么,当$m\leqslant n$时,我们有
$$\sigma_{m-1} S_1 = (x_1\cdots x_{m-1} + \cdots + x_{n-m+1}\cdots x_{n}) (x_1 + \cdots + x_{n}) = \mathcal{S}(x_1^2x_2\cdots x_{m-1}) + {\color{red} m\sigma_m}$$
交错求和得
\begin{align*}
& -a_1 S_{m-1} - a_2 S_{m-2} - \cdots - a_{m-1} S_1 \\
= & \sigma_1 S_{m-1} - \sigma_2 S_{m-2} + \cdots + (-1)^{m} \sigma_{m-1} S_{1} \\
= & S_m + m(-1)^{m}\sigma_m = S_m + ma_m
\end{align*}
即有
$$S_m + a_1S_{m-1} + \cdots + a_{m-1}S_1 + ma_m = 0.$$
当$m > n$时,最后一个等式变为
$$\sigma_n S_{m-n} = x_1\cdots x_n (x_1^{m-n} + \cdots + x_n^{m-n}) = \mathcal{S}(x_1^{m-n+1}x_2 \cdots x_n),$$
交错求和得
$$-a_1 S_{m-1} - a_2 S_{m-2} - \cdots - a_n S_{m-n} = \sigma_1 S_{m-1} - \sigma_2 S_{m-2} + \cdots + (-1)^{m} \sigma_n S_{m-n} = S_m$$
即
$$S_m + a_1 S_{m-1} + \cdots + a_n S_{m-n} = 0.$$

回到6.8第4题。我们要证明$A, B$特征值对应相等$\Longleftrightarrow$ $\operatorname{tr} A^k = \operatorname{tr} B^k$, $\forall k \in \mathbb{N}_+$. (在代数闭域下)将$A,B$分别上三角化为$T_1, T_2$,对角线元素为$A$的特征值$\lambda_1,\ldots,\lambda_n$以及$B$的特征值$\mu_1,\ldots,\mu_n$, 那么我们只要证明
$$f_A(X) = f_B(X) \Longleftrightarrow \{\lambda_i\}, \{\mu_j\}\text{对应相等} \Longleftrightarrow \operatorname{tr} T_1^k = \operatorname{tr} T_2^k \Longleftrightarrow S_k(f_A(X)) = S_k(f_B(X)), \forall k \in \mathbb{N}_+$$
其中$f_A, f_B$分别为$A,B$的特征多项式。$\Longrightarrow$是显然的,我们只要证明$\Longleftarrow$. 取$k = 1, 2, \ldots, n$, 那么根据$k \leqslant n$时的牛顿公式
$$
\begin{cases}
a_1 = -S_1 \\
a_1S_1 + 2a_2 = -S_2 \\
\vdots \\
a_1S_{n-1} + \cdots + na_n = -S_n
\end{cases}
$$
以上以$a_1,\ldots,a_n$为未知元的非齐次线性方程组的系数方阵非奇异,故有唯一解,即$a_1,\ldots,a_n$由$S_1,\ldots,S_n$唯一确定,$\Longleftarrow$即证明完毕。


\newpageorvspace


{\bf 第二题}. 习题7.2第4题. (1). 设$\mathscr{A,B}$为奇数维实线性空间$V$上的线性变换且$\mathscr{AB} = \mathscr{BA}$, 求证$\mathscr{A,B}$有公共特征向量。

\newpageorvspace

{\bf 证明}:(1). $\mathscr{A,B}$的特征多项式都是奇数阶的首一的实系数多项式。一个实系数多项式$f(X)$有分解
$$f(X) = (X-x_1) \cdots (X-x_k) (X^2 - (z_1+\overline{z}_1)X + z_1\overline{z}_1) \cdots (X-x_k) (X^2 - (z_m+\overline{z}_m)X + z_m\overline{z}_m),$$
其中$x_1, \cdots, x_k$为实根, $z_1, \overline{z}_1, \cdots, z_m, \overline{z}_m$为复根。奇数阶的实系数多项式至少有一个实根,而且至少有一个实根的代数重数为奇数。

令$\lambda_0$为$\mathscr{A}$的一个代数重数为奇数的实根,令$W_{\lambda_0}$为其根子空间,维数为奇数。任取$\alpha \in W_{\lambda_0}$, 有足够大的正整数$s$满足
$$\mathscr{A}^s(\mathscr{B}(\alpha)) = \mathscr{B} (\mathscr{A}^s(\alpha)) = \mathscr{B}(0) = 0.$$
故$\mathscr{B}(\alpha) \in W_{\lambda_0}$, 即知$W_{\lambda_0}$是$\mathscr{B}$的不变子空间。由于$W_{\lambda_0}$是奇数维的,故存在实特征值$\mu_0$以及对应的特征向量$\beta \in W_{\lambda_0}$, 满足$\mathscr{B}(\beta) = \mu_0 \beta$. 由于$\beta \in W_{\lambda_0}$, 令$s$为满足
$(\mathscr{A} - \lambda_0)^s \beta = 0$的最小的正整数,$s \geqslant 1$. 令$\beta' = (\mathscr{A} - \lambda_0)^{s-1} \beta \neq 0$, 有
\begin{align*}
(\mathscr{A} - \lambda_0) \beta' & = (\mathscr{A} - \lambda_0)^s \beta = 0, \\
(\mathscr{B} - \mu_0) \beta' & = (\mathscr{A} - \lambda_0)^{s-1} ((\mathscr{B} - \mu_0) \beta) = 0.
\end{align*}
于是$\beta'$是$\mathscr{A,B}$的公共特征向量。

% \newpageorvspace


% {\bf 第三题}. 习题7.2第6题. $V$是复数域上$n$维线性空间。$f,g$是$V$上线性变换,且满足$fg - gf = f$. 求证:$f$的特征值都是$0$, 且$f,g$有公共特征向量。

% \newpageorvspace

% {\bf 证明}:


\newpageorvspace


{\bf 第三题}. 习题7.3第5题. 设$\lambda_1, \ldots, \lambda_t$是线性变换$\mathscr{A}$的不同的特征值,$\alpha_1, \ldots, \alpha_t$分别是数域这些特征值的特征向量。求证:$\alpha_1 + \cdots + \alpha_t$生成的循环子空间$U = \mathbb{F}\alpha_1 \oplus \cdots \oplus \mathbb{F}\alpha_t$.

\newpageorvspace

{\bf 证明}:由于$\alpha_1 + \cdots + \alpha_t \in \mathbb{F}\alpha_1 \oplus \cdots \oplus \mathbb{F}\alpha_t$, 所以$U = \mathbb{F}[\mathscr{A}](\alpha_1 + \cdots + \alpha_t) \subseteq \mathbb{F}\alpha_1 \oplus \cdots \oplus \mathbb{F}\alpha_t$. 我们来证明另一个方向的包含关系。我们希望证明,任取$1 \leqslant i \leqslant t$, 都有$\alpha_i \in \mathbb{F}[\mathscr{A}](\alpha_1 + \cdots + \alpha_t)$. 我们需要寻找一个多项式$f(\lambda)$, 使得$f(\mathscr{A}) (\alpha_1 + \cdots + \alpha_t) = \alpha_i$. 我们可以取
$$f(\lambda) = \prod\limits_{\substack{1 \leqslant j \leqslant t \\ j \neq i}} \dfrac{\lambda - \lambda_j}{\lambda_i - \lambda_j},$$
这个多项式满足$f(\mathscr{A})(\alpha_i) = \alpha_i$, $f(\mathscr{A})(\alpha_i) = 0$, $\forall j \neq i$. 


\newpageorvspace


{\bf 第四题}. 习题7.3第6题. 设向量$\alpha, \beta$相对于线性变换$\mathscr{A}$的最小多项式$d_{\alpha}(\lambda)$与$d_{\beta}(\lambda)$互素。求证:$\mathbb{F}[\mathscr{A}]\alpha \oplus \mathbb{F}[\mathscr{A}]\beta = \mathbb{F}[\mathscr{A}](\alpha + \beta)$.

\newpageorvspace

{\bf 证明}. 这题实际上是要证明两个结论:1. $\mathbb{F}[\mathscr{A}]\alpha + \mathbb{F}[\mathscr{A}]\beta = \mathbb{F}[\mathscr{A}]\alpha \oplus \mathbb{F}[\mathscr{A}]\beta$; 2. $\mathbb{F}[\mathscr{A}]\alpha \oplus \mathbb{F}[\mathscr{A}]\beta = \mathbb{F}[\mathscr{A}](\alpha + \beta)$.

对于第1个结论,我们只要证明$\mathbb{F}[\mathscr{A}]\alpha \cap \mathbb{F}[\mathscr{A}]\beta = \{0\}$. 任取$\gamma \in \mathbb{F}[\mathscr{A}]\alpha \cap \mathbb{F}[\mathscr{A}]\beta$, 那么存在多项式$f_1, f_2$使得
$$\gamma = f_1(\mathscr{A})\alpha = f_2(\mathscr{A})\beta.$$
由于$d_{\alpha}(\lambda)$与$d_{\beta}(\lambda)$互素,所以存在多项式$u,v$使得$u(\lambda)d_{\alpha}(\lambda) + v(\lambda)d_{\beta}(\lambda) = 1$, 那么
\begin{align*}
\gamma & = (u(\mathscr{A})d_{\alpha}(\mathscr{A}) + v(\mathscr{A})d_{\beta}(\mathscr{A})) (\gamma) \\
& = f_1(\mathscr{A})u(\mathscr{A})(d_{\alpha}(\mathscr{A})(\alpha)) + f_2(\mathscr{A})v(\mathscr{A})(d_{\beta}(\mathscr{A})(\beta)) \\
& = 0
\end{align*}

对于第2个结论,由于$\alpha + \beta \in \mathbb{F}[\mathscr{A}]\alpha \oplus \mathbb{F}[\mathscr{A}]\beta$, 所以有$\mathbb{F}[\mathscr{A}]\alpha \oplus \mathbb{F}[\mathscr{A}]\beta \supseteq \mathbb{F}[\mathscr{A}](\alpha + \beta)$. 我们来证明另一边的包含关系。我们有
\begin{align*}
\alpha & = (u(\mathscr{A})d_{\alpha}(\mathscr{A}) + v(\mathscr{A})d_{\beta}(\mathscr{A})) (\alpha) \\
& = 0 + v(\mathscr{A})d_{\beta}(\mathscr{A}) (\alpha) \\
& = v(\mathscr{A})d_{\beta}(\mathscr{A}) (\alpha + \beta)
\end{align*}
同样地有$\beta = u(\mathscr{A})d_{\alpha}(\mathscr{A}) (\alpha + \beta)$. 所以$\alpha, \beta \in \mathbb{F}[\mathscr{A}](\alpha + \beta),$ 从而有$\mathbb{F}[\mathscr{A}]\alpha \oplus \mathbb{F}[\mathscr{A}]\beta \subseteq \mathbb{F}[\mathscr{A}](\alpha + \beta)$.


\newpageorvspace


{\bf 第五题}. 设$\mathscr{A}$是$n$维线性空间$V$的线性变换,且存在循环向量向量$\alpha \in V$, 使得$V = \mathbb{F}[\mathscr{A}]\alpha$. 求证:与$\mathscr{A}$可交换的$V$上任一线性变换$\mathscr{B}$必为$\mathscr{A}$的多项式。

\newpageorvspace

{\bf 证明}: 由于$V = \mathbb{F}[\mathscr{A}]\alpha$, 所以存在非零多项式$f_1, \ldots, f_n$使得$f_1(\mathscr{A})(\alpha), \ldots, f_n(\mathscr{A})(\alpha)$为$V$的一组基。故存在数$a_1, \ldots, a_n$, 使得$\mathscr{B}(\alpha) = a_1 f_1(\mathscr{A})(\alpha) + \cdots + a_n f_n(\mathscr{A})(\alpha) = f(\mathscr{A})(\alpha)$, 其中$f = a_1 f_1 + \cdots a_n f_n$.

任取$\beta = b_1 f_1(\mathscr{A})(\alpha) + \cdots + b_n f_n(\mathscr{A})(\alpha) \in V$, 有
\begin{align*}
\mathscr{B} (\beta) & = \mathscr{B} (b_1 f_1(\mathscr{A})(\alpha) + \cdots + b_n f_n(\mathscr{A})(\alpha)) \\
& = b_1 f_1(\mathscr{A}) (\mathscr{B}(\alpha)) + \cdots + b_n f_n(\mathscr{A})(\mathscr{B}(\alpha)) \\
& = b_1 f_1(\mathscr{A}) (f(\mathscr{A})(\alpha)) + \cdots + b_n f_n(\mathscr{A})(f(\mathscr{A})(\alpha)) \\
& = f(\mathscr{A})(b_1 f_1(\mathscr{A})(\alpha)) + \cdots + f(\mathscr{A})(b_n f_n(\mathscr{A})(\alpha)) \\
& = f(\mathscr{A}) (b_1 f_1(\mathscr{A})(\alpha) + \cdots + b_n f_n(\mathscr{A})(\alpha)) \\
& = f(\mathscr{A}) (\beta)
\end{align*}
所以有$\mathscr{B} = f(\mathscr{A})$.


\newpageorvspace


{\bf 第六题}. 设$\mathscr{A}$为线性空间$V$上的线性变换。求证:若$\mathscr{A}^2$有循环向量,即存在$\alpha\in V$使得$V = \mathbb{F}[\mathscr{A}^2](\alpha)$, 则$\mathscr{A}$也有循环向量。请问反过来是否成立?

\newpageorvspace

{\bf 证明}: 我们有
\begin{align*}
V & \supseteq \mathbb{F}[\mathscr{A}](\alpha) \supseteq \operatorname{span} \{ \alpha, \mathscr{A}\alpha, \mathscr{A}^2\alpha, \ldots, \mathscr{A}^{2k-1}\alpha, \mathscr{A}^{2k}\alpha, \ldots \} \\
& \supseteq \operatorname{span} \{ \alpha, \mathscr{A}^2\alpha, \ldots, \mathscr{A}^{2k}\alpha, \ldots \} = \mathbb{F}[\mathscr{A}^2](\alpha) = V
\end{align*}
所以,上式涉及的$\subseteq$实际上都是相等,于是$V = \mathbb{F}[\mathscr{A}](\alpha)$, $\alpha$也是$\mathscr{A}$的循环向量。

反过来是不成立的。例如习题7.3第4题中大家举的例子:
$$\mathscr{A}: \begin{pmatrix} x \\ y \end{pmatrix} \mapsto \begin{pmatrix} 0 & 1 \\ 0 & 0 \end{pmatrix} \begin{pmatrix} x \\ y \end{pmatrix} = \begin{pmatrix} y \\ 0 \end{pmatrix},$$
任一满足$y \neq 0$的向量$\alpha = \begin{pmatrix} x \\ y \end{pmatrix}$都是$\mathscr{A}$的循环向量。但$\mathscr{A}^2 = 0$, 所以$\mathscr{A}^2$不可能有循环向量。


\end{document}
% }
