\usepackage{relsize}

% \renewcommand{\newpageorvspace}{\newpage}
\renewcommand{\newpageorvspace}{\vspace{2em}}

\date{2022-5-20  第六次习题课}

\begin{document}

\maketitle


% \larger[2]


{\bf 第一题}. 9.2例3(最小二乘法)求直线$y = kx + b$尽可能接近已知数据点$(x_i, y_i)$ $(1 \leqslant i \leqslant n)$, 也就是使$d(k, b) = \sum\limits_{i=1}^n (kx_i + b - y_i)^2$达到最小值。

% \newpageorvspace

{\bf 解}. 我们首先从分析的角度来看这个问题。$d(k, b) = \sum\limits_{i=1}^n (kx_i + b - y_i)^2$关于$k, b$都是(下)凸的,所以满足$\operatorname{grad} d(k, b) = (0,0)^T$的点都是使$d(k, b)$达到最小值的点。计算梯度得
$$
\operatorname{grad} d(k, b) = \begin{pmatrix} 2 \sum\limits_{i=1}^n x_i (kx_i + b - y_i) \\ 2 \sum\limits_{i=1}^n (kx_i + b - y_i) \end{pmatrix}
$$
得方程组
$$
\begin{cases}
k \sum\limits_{i=1}^n x_i^2 + b \sum\limits_{i=1}^n x_i - \sum\limits_{i=1}^n x_i y_i = 0 \\
k \sum\limits_{i=1}^n x_i + nb - \sum\limits_{i=1}^n y_i = 0
\end{cases}
$$
或者写成矩阵的形式
$$
M \begin{pmatrix} k \\ b \end{pmatrix} = \begin{pmatrix} \sum\limits_{i=1}^n x_i^2 & \sum\limits_{i=1}^n x_i \\ \sum\limits_{i=1}^n x_i & n \end{pmatrix} \begin{pmatrix} k \\ b \end{pmatrix} = \begin{pmatrix} \sum\limits_{i=1}^n x_i y_i \\ \sum\limits_{i=1}^n y_i \end{pmatrix}
$$
只要满足$x_i$不全相等,系数矩阵就非奇异,$(k, b)$有唯一解
$$\begin{pmatrix} k \\ b \end{pmatrix} = M^{-1}\begin{pmatrix} \sum\limits_{i=1}^n x_i y_i \\ \sum\limits_{i=1}^n y_i \end{pmatrix}$$
令$A = \begin{pmatrix} x_1 & 1 \\ \vdots & \vdots \\ x_n & 1 \end{pmatrix}$, $\mathbf{y} = \begin{pmatrix} y_1 \\ \vdots \\ y_n \end{pmatrix}$, 那么$M = A^TA$, 上述解又可以写为
$$(A^TA)^{-1}A^T\mathbf{y}.$$

下面,我们从代数的角度来解这个问题。在最理想的情况下,所有数据点都满足某个线性方程$y = kx + b$, 写成矩阵的形式即为
$$
\begin{pmatrix} x_1 & 1 \\ \vdots & \vdots \\ x_n & 1 \end{pmatrix} \begin{pmatrix} k \\ b \end{pmatrix} = \begin{pmatrix} y_1 \\ \vdots \\ y_n \end{pmatrix}
$$
一般情况下,满足上式的$(k,b)$不存在。那么就退而求其次,找一个近似解(最小二乘解)。我们考虑更一般的形式$A\mathbf{x} = \mathbf{b}$. 例如在这里,$A = \begin{pmatrix} x_1 & 1 \\ \vdots & \vdots \\ x_n & 1 \end{pmatrix}$, $\mathbf{x} = \begin{pmatrix} k \\ b \end{pmatrix}$, $\mathbf{b} = \begin{pmatrix} y_1 \\ \vdots \\ y_n \end{pmatrix}$.

我们知道$A\mathbf{x} = \mathbf{b}$有解$\Longleftrightarrow$ $\mathbf{b} \in \operatorname{span}(A)$ ($A$的列空间)。当它无解时,$\mathbf{b} \not\in \operatorname{span}(A)$. 最小二乘解$\widehat{\mathbf{x}}$是使得$\lvert A\widehat{\mathbf{x}}-\mathbf{b} \rvert$ (向量长度一般用$\lVert \cdot \rVert$表示,这里遵从课本的记号,用$\lvert \cdot \rvert$表示)最小的向量。令$\widehat{\mathbf{b}} = A\widehat{\mathbf{x}} \in \operatorname{span}(A)$, 那么
$$\widehat{\mathbf{b}} = \argmin_{\mathbf{b}' \in \operatorname{span}(A)} \lvert \mathbf{b}' - \mathbf{b} \rvert$$
$\mathbf{b}' - \mathbf{b}$就是下图虚线(dashed line)代表的向量,它最短当且仅当$\mathbf{b}'$是$\mathbf{b}$到$\operatorname{span}(A)$的投影(之后习题会证明),即$\widehat{\mathbf{b}} - \mathbf{b} = A\widehat{\mathbf{x}}-\mathbf{b} \in \operatorname{span}(A)^{\perp}$, 这等价于正则方程
$$A^T(A\widehat{\mathbf{x}}-\mathbf{b}) = 0$$

\begin{figure}[H]
\centering
\begin{tikzpicture}
\path[-, thin] (0,0) edge (7,0);
\path[-, thin] (0,0) edge (-2,-2);
\path[-, thin] (-2,-2) edge (5,-2);
\path[-, thin] (7,0) edge (5,-2);
\path[-Stealth, very thick] (0,0) edge (4,2);
\path[-Stealth, very thick, dashed] (4,2) edge (4,-1);
\path[-Stealth, very thick] (0,0) edge (4,-1);
\path[-Stealth, very thick] (0,0) edge (2.5,-1.5);
\path[-Stealth, very thick, dashed] (4,2) edge (2.5,-1.5);
\path[-Stealth, dotted] (4,-1) edge (2.5,-1.5);
\node at (4.7, 0.5) {$A\widehat{\mathbf{x}}-\mathbf{b}$};
\node at (2.6, 0.5) {$A\mathbf{x}'-\mathbf{b}$};
\node at (1, 0.7) {$\mathbf{b}$};
\node at (2.3, -0.28) {$\widehat{\mathbf{b}}$};
\node at (0.6, -0.7) {$\mathbf{b}'$};
\node at (0, -1.5) {$\operatorname{span}(A)$};
\node at (4.1, -1.4) {$\mathbf{b}'-\widehat{\mathbf{b}}$};
\draw[-, very thick] (4,-0.6) -- (3.6, -0.5) -- (3.6, -0.9);
\end{tikzpicture}
\end{figure}

当$A$列满秩时,在这题里就是$x_i$不全相等时,$A^TA$是可逆方阵,上述正则方程有唯一解
$$\widehat{\mathbf{x}} = (A^TA)^{-1}A^T\mathbf{b}.$$

当$A$列不满秩时,最小二乘解不唯一,但长度最小的最小二乘解(被称作最优最小二乘解)是唯一的。这部分参阅补充内容,有时间就讲。


\newpageorvspace


{\bf 第二题}. 习题9.2第8题. 用向量的内积证明平面外一点到平面的线段长以垂线段最短,并推广到$n$维欧氏空间:

(1) 取平面上任一点为原点$O$, 将空间$V$每一点$P$用向量$\overrightarrow{OP}$表示,则平面是一个2维子空间$W$. 设$A$是空间中给定的任一点,$B$是平面内任一点,分别对应于向量$\alpha = \overrightarrow{OA}, \beta = \overrightarrow{OB}$, 则$\lvert AB \rvert = \lvert \alpha - \beta \rvert$. 求证:当$\alpha - \beta \in W^{\perp}$时$\lvert \alpha - \beta \rvert$取最小值;

(2) 设$E(\mathbb{R})$是欧氏空间,$W$是它的任意子空间,$\alpha$是$E(\mathbb{R})$任意给定的向量。求证:当$\alpha - \beta \in W^{\perp}$时,$\lvert \alpha - \beta \rvert$ $(\beta \in W)$取最小值;

(3) 设$E(\mathbb{R})$是欧氏空间,$\alpha \in E(\mathbb{R})$, $W$是由$\alpha_1, \ldots, \alpha_k \in E(\mathbb{R})$生成的子空间。当$x_1, \ldots, x_k \in \mathbb{R}$满足什么条件时,$\lvert \alpha - (x_1\alpha_1 + \cdots + x_k\alpha_k) \rvert$取最小值?

% \newpageorvspace

{\bf 证明}:我们还用一下上一题的图,从而有一个比较直观的认识:我们设$\beta \in W$是使得向量$\beta-\alpha$垂直于$W$的向量,$\beta' \in W$是空间$W$中任意的另外一个向量,那么$\beta'-\alpha, \beta-\alpha, \beta'-\beta$组成了一个直角三角形,以$\beta'-\alpha$为斜边,垂线段$\beta-\alpha$是一条直角边,必然比斜边短。

\begin{figure}[H]
\centering
\begin{tikzpicture}
\path[-, thin] (0,0) edge (7,0);
\path[-, thin] (0,0) edge (-2,-2);
\path[-, thin] (-2,-2) edge (5,-2);
\path[-, thin] (7,0) edge (5,-2);
\path[-Stealth, very thick] (0,0) edge (4,2);
\path[-Stealth, very thick, dashed] (4,2) edge (4,-1);
\path[-Stealth, very thick] (0,0) edge (4,-1);
\path[-Stealth, very thick] (0,0) edge (2.5,-1.5);
\path[-Stealth, very thick, dashed] (4,2) edge (2.5,-1.5);
\path[-Stealth, dotted] (4,-1) edge (2.5,-1.5);
\node at (4.7, 0.5) {$\beta-\alpha$};
\node at (2.6, 0.5) {$\beta'-\alpha$};
\node at (1, 0.7) {$\alpha$};
\node at (2.3, -0.28) {$\beta$};
\node at (0.6, -0.7) {$\beta'$};
\node at (-1, -1.6) {$W$};
\node at (4.1, -1.4) {$\beta'-\beta$};
\draw[-, very thick] (4,-0.6) -- (3.6, -0.5) -- (3.6, -0.9);
\end{tikzpicture}
\end{figure}

下面我们用向量的内积来证明这题的结论。以下,我们用$\langle \cdot, \cdot \rangle$来表示向量内积。

(1) 假设向量$\beta\in W$使得$\alpha - \beta \in W^{\perp}$, $\beta'\in W$是$W$中任意一个向量。那么
\begin{align*}
\lvert \alpha - \beta' \rvert^2 & = \lvert (\alpha - \beta) + (\beta - \beta') \rvert^2 \\
& = \langle (\alpha - \beta) + (\beta - \beta'), (\alpha - \beta) + (\beta - \beta') \rangle \\
& = \lvert \alpha - \beta \rvert^2 + 2\langle \alpha-\beta, \beta-\beta' \rangle + \lvert \beta - \beta' \rvert^2 \\
& = \lvert \alpha - \beta \rvert^2 + \lvert \beta - \beta' \rvert^2 \\
& \geqslant \lvert \alpha - \beta \rvert^2
\end{align*}
上式中的$\geqslant$是相等当且仅当$\beta = \beta'$. 这就证明了当$\alpha - \beta \in W^{\perp}$时$\lvert \alpha - \beta \rvert$取最小值。

(2) 和第(1)问的证明相同,因为(1)的证明与$W$的维数是无关的。

(3) 由第(2)问知需要满足$\alpha - (x_1\alpha_1 + \cdots + x_k\alpha_k) \in W^{\perp}$. 由于$W$是由$\alpha_1, \ldots, \alpha_k \in E(\mathbb{R})$生成的,这又等价于$\forall 1 \leqslant i \leqslant k$, 有
$$0 = \langle \alpha_i, \alpha - (x_1\alpha_1 + \cdots + x_k\alpha_k) \rangle,$$
即
$$\sum\limits_{j=1}^k \langle \alpha_i, \alpha_j \rangle x_j = \langle \alpha_i, \alpha \rangle,$$
写成矩阵形式即
$$
\begin{pmatrix}
\langle \alpha_1, \alpha_1 \rangle & \cdots & \langle \alpha_1, \alpha_k \rangle \\
\vdots & & \vdots \\
\langle \alpha_k, \alpha_1 \rangle & \cdots & \langle \alpha_k, \alpha_k \rangle
\end{pmatrix}
\begin{pmatrix}
x_1 \\ \vdots \\ x_k
\end{pmatrix} = 
\begin{pmatrix}
\langle \alpha_1, \alpha \rangle \\
\vdots \\
\langle \alpha_k, \alpha \rangle
\end{pmatrix}
$$
当$x_1, \ldots, x_k \in \mathbb{R}$是上述线性方程组的解时,$\lvert \alpha - (x_1\alpha_1 + \cdots + x_k\alpha_k) \rvert$取最小值。

注意,假设我们在$E(\mathbb{R})$中取定了一组基,那么$\alpha, \alpha_1, \ldots, \alpha_k$就有(列向量形式的)坐标表示,这些坐标(列向量)我们还用$\alpha, \alpha_1, \ldots, \alpha_k$来记,并令矩阵$A = (\alpha_1, \ldots, \alpha_k)$, 那么上述方程就可以写为
$$A^TA \begin{pmatrix} x_1 \\ \vdots \\ x_k \end{pmatrix} = A^T \alpha.$$


\newpageorvspace


{\bf 第三题}. 习题9.2第9题. (1) 设$W$是$\mathbb{R}^3$中过点$(0,0,0), (1,2,2), (3,4,0)$的平面,求点$A (5,0,0)$到平面$W$的最短距离;

(2) 求方程组
$$
\begin{cases}
0.39x - 1.89y = 1 \\
0.61x - 1.80y = 1 \\
0.93x - 1.68y = 1 \\
1.35x - 1.50y = 1
\end{cases}
$$
的最小二乘解。

(3) 设$A\in\mathbb{R}^{m\times n}, X = (x_1,\cdots,x_n)^T, \beta\in\mathbb{R}^{m\times 1}$. 如果实系数线性方程组$AX = \beta$无解,我们可以求$X$使$\mathbb{R}^{m\times 1}$中的向量$\delta = AX - \beta$的长度$\lvert \delta \rvert$取最小值。满足这个条件的解$X$称为方程组$AX = \beta$的最小二乘解。设$A$的各列依次为$\alpha_1, \ldots, \alpha_n$, 则$AX = x_1\alpha_1 + \cdots + x_n\alpha_n$. 求证:
$$\lvert \delta \rvert \text{取最小值} \Longleftrightarrow (\delta, \alpha_i) = 0 (\forall 1\leqslant i \leqslant n) \Longleftrightarrow A^TAX = A^T\beta$$

% \newpageorvspace

{\bf 解}:设$O = (0,0,0), P_1 = (1,2,2), P_2 = (3,4,0)$, 那么$W$正好是过原点的由$\alpha_1 = \overrightarrow{OP_1} = (1,2,2), \alpha_2 = \overrightarrow{OP_2} = (3,4,0)$张成的线性空间。设$\beta' = \overrightarrow{OA'}$是$\beta = \overrightarrow{OA} = (5,0,0)$在$W$上的投影,则$\lvert \overrightarrow{AA'} \rvert$即是点$A$到平面$W$的最短距离。设$\beta' = x_1 \alpha_1 + x_2 \alpha_2 \in W$, 那么根据上一题的结论,$(x_1, x_2)$需要是如下的线性方程组的解
$$
\begin{pmatrix}
\langle \alpha_1, \alpha_1 \rangle & \langle \alpha_1, \alpha_2 \rangle \\
\langle \alpha_2, \alpha_1 \rangle & \langle \alpha_2, \alpha_2 \rangle
\end{pmatrix}
\begin{pmatrix}
x_1 \\ x_2
\end{pmatrix} = 
\begin{pmatrix}
\langle \alpha_1, \beta \rangle \\
\langle \alpha_2, \beta \rangle
\end{pmatrix}
$$
即
$$
\begin{pmatrix}
9 & 11 \\ 11 & 25
\end{pmatrix}
\begin{pmatrix}
x_1 \\ x_2
\end{pmatrix} = 
\begin{pmatrix}
5 \\ 15
\end{pmatrix}
$$
解得
$$
\begin{pmatrix}
x_1 \\ x_2
\end{pmatrix} = \dfrac{1}{104} \begin{pmatrix}
25 & -11 \\ -11 & 9
\end{pmatrix}
\begin{pmatrix}
5 \\ 15
\end{pmatrix} = \dfrac{1}{104} \begin{pmatrix}
-40 \\ 80
\end{pmatrix} = \dfrac{5}{13}\begin{pmatrix}
-1 \\ 2
\end{pmatrix}
$$
最短距离即等于$\lvert \beta - (x_1\alpha_1 + x_2\alpha_2) \rvert = \lvert (5,0,0) - \dfrac{5}{13} (5,6,-2) \rvert = \dfrac{5}{13} \lvert (8,-6,2) \rvert = 10 \sqrt{\dfrac{2}{13}}$.

(2) 将问题写成矩阵的形式
$$A\begin{pmatrix}
x \\ y
\end{pmatrix} = \begin{pmatrix}
0.39 & -1.89 \\ 0.61 & -1.80 \\ 0.93 & -1.68 \\ 1.35 & -1.50
\end{pmatrix}
\begin{pmatrix}
x \\ y
\end{pmatrix} = 
\begin{pmatrix}
1 \\ 1 \\ 1 \\ 1
\end{pmatrix} = \beta
$$
那么这个问题的最小二乘解为$(A^TA)^{-1}A^T\beta$, 编程解如下
\begin{lstlisting}
>>> import numpy as np

>>> A = np.array([[0.39,-1.89], [0.61,-1.80], [0.93,-1.68], [1.35,-1.50]])
>>> beta = np.array([1,1,1,1])

>>> sol = np.linalg.inv(A.T @ A) @ A.T @ beta
>>> sol
array([ 0.19722003, -0.48807896])
\end{lstlisting}

(3) 这问实际上我们在第一题中已经证明完毕。


\newpageorvspace


{\bf 第四题}. 习题9.3第6题. 给定$0 \neq \alpha \in E_n(\mathbb{R})$. 定义$E_n(\mathbb{R})$中的线性变换$\tau_{\alpha}: \beta \mapsto \beta - \dfrac{2\langle \beta, \alpha \rangle}{\langle \alpha, \alpha \rangle} \alpha$, 求证:

(1) $\tau_{\alpha}$是正交变换;

(2) $\tau_{\alpha}$在适当的标准正交基下的矩阵为$\operatorname{diag}(-1, 1, \cdots, 1)$.

% \newpageorvspace

{\bf 证明}. (1). 要证$\tau_{\alpha}$是正交变换,我们只要证明任取$\beta_1, \beta_2 \in E_n(\mathbb{R})$都有$\langle \tau_{\alpha}(\beta_1), \tau_{\alpha}(\beta_2) \rangle = \langle \beta_1, \beta_2 \rangle$. 我们来验证这个结论。
\begin{align*}
& \langle \tau_{\alpha}(\beta_1), \tau_{\alpha}(\beta_2) \rangle \\
= & \langle \beta_1 - \dfrac{2\langle \beta_1, \alpha \rangle}{\langle \alpha, \alpha \rangle} \alpha, \beta_2 - \dfrac{2\langle \beta_2, \alpha \rangle}{\langle \alpha, \alpha \rangle} \alpha  \rangle \\
= & \langle \beta_1, \beta_2 \rangle - \dfrac{2\langle \beta_2, \alpha \rangle \langle \beta_1, \alpha \rangle}{\langle \alpha, \alpha \rangle}- \dfrac{2\langle \beta_1, \alpha \rangle \langle \alpha, \beta_2 \rangle}{\langle \alpha, \alpha \rangle} + \dfrac{4\langle \beta_1, \alpha \rangle \langle \beta_2, \alpha \rangle \langle \alpha, \alpha \rangle}{\langle \alpha, \alpha \rangle^2} \\
= & \langle \beta_1, \beta_2 \rangle
\end{align*}

(2) 对于线性变换$\tau_{\alpha}: \beta \mapsto \beta - \dfrac{2\langle \beta, \alpha \rangle}{\langle \alpha, \alpha \rangle} \alpha$, 我们容易发现如下结论
\begin{itemize}
\item $\tau_{\alpha}(\alpha) = \alpha - \dfrac{2\langle \alpha, \alpha \rangle}{\langle \alpha, \alpha \rangle} \alpha = -\alpha$.
\item 任取$\beta \in \operatorname{span}\{\alpha\}^{\perp}$, 有$\tau_{\alpha}(\beta) = \beta$.
\end{itemize}
那么我们从向量$\alpha_1 = \dfrac{\alpha}{\lvert \alpha \rvert}$出发,扩充成$E_n(\mathbb{R})$的一组标准正交基$\alpha_1, \alpha_2, \ldots, \alpha_n$, 那么
$$\tau_{\alpha}(\alpha_1) = \tau_{\alpha}(\dfrac{\alpha}{\lvert \alpha \rvert}) = \dfrac{1}{\lvert \alpha \rvert} \tau_{\alpha}(\alpha) = \dfrac{-\alpha}{\lvert \alpha \rvert} = -\alpha_1$$
此外,$\alpha_2, \ldots, \alpha_n \in \operatorname{span}\{\alpha\}^{\perp}$, 故有$\tau_{\alpha}(\alpha_i) = \alpha_i$, $\forall 2 \leqslant i \leqslant n$. 那么在标准正交基$\alpha_1, \alpha_2, \ldots, \alpha_n$下,$\tau_{\alpha}$的矩阵为$\operatorname{diag}(-1, 1, \cdots, 1)$。

这里我们容易看到,$\tau_{\alpha}$实际上就是关于以$\alpha$为法向量的超平面(由一个线性方程确定的$n-1$维子空间)的对称变换。

\end{document}
