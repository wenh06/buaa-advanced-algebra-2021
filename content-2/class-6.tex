\usepackage{relsize}

\renewcommand{\newpageorvspace}{\newpage}
% \renewcommand{\newpageorvspace}{\vspace{2em}}

\date{2022-5-20  第六次习题课}

\begin{document}

\maketitle


\larger[2]


{\bf 第一题}. 9.2例3(最小二乘法)求直线$y = kx + b$尽可能接近已知数据点$(x_i, y_i)$ $(1 \leqslant i \leqslant n)$, 也就是使$d(k, b) = \sum\limits_{i=1}^n (kx_i + b - y_i)^2$达到最小值。

\newpageorvspace

{\bf 解}. 我们首先从分析的角度来看这个问题。$d(k, b) = \sum\limits_{i=1}^n (kx_i + b - y_i)^2$关于$k, b$都是(下)凸的,所以满足$\operatorname{grad} d(k, b) = (0,0)^T$的点都是使$d(k, b)$达到最小值的点。计算梯度得
$$
\operatorname{grad} d(k, b) = \begin{pmatrix} 2 \sum\limits_{i=1}^n x_i (kx_i + b - y_i) \\ 2 \sum\limits_{i=1}^n (kx_i + b - y_i) \end{pmatrix}
$$
得方程组
$$
\begin{cases}
k \sum\limits_{i=1}^n x_i^2 + b \sum\limits_{i=1}^n x_i - \sum\limits_{i=1}^n x_i y_i = 0 \\
k \sum\limits_{i=1}^n x_i + nb - \sum\limits_{i=1}^n y_i = 0
\end{cases}
$$
或者写成矩阵的形式
$$
M \begin{pmatrix} k \\ b \end{pmatrix} = \begin{pmatrix} \sum\limits_{i=1}^n x_i^2 & \sum\limits_{i=1}^n x_i \\ \sum\limits_{i=1}^n x_i & n \end{pmatrix} \begin{pmatrix} k \\ b \end{pmatrix} = \begin{pmatrix} \sum\limits_{i=1}^n x_i y_i \\ \sum\limits_{i=1}^n y_i \end{pmatrix}
$$
只要满足$x_i$不全相等,系数矩阵就非奇异,$(k, b)$有唯一解
$$\begin{pmatrix} k \\ b \end{pmatrix} = M^{-1}\begin{pmatrix} \sum\limits_{i=1}^n x_i y_i \\ \sum\limits_{i=1}^n y_i \end{pmatrix}$$
令$A = \begin{pmatrix} x_1 & 1 \\ \vdots & \vdots \\ x_n & 1 \end{pmatrix}$, $\mathbf{y} = \begin{pmatrix} y_1 \\ \vdots \\ y_n \end{pmatrix}$, 那么$M = A^TA$, 上述解又可以写为
$$(A^TA)^{-1}A^T\mathbf{y}$$

下面,我们从代数的角度来解这个问题。在最理想的情况下,所有数据点都满足某个线性方程$y = kx + b$, 写成矩阵的形式即为
$$
\begin{pmatrix} x_1 & 1 \\ \vdots & \vdots \\ x_n & 1 \end{pmatrix} \begin{pmatrix} k \\ b \end{pmatrix} = \begin{pmatrix} y_1 \\ \vdots \\ y_n \end{pmatrix}
$$
一般情况下,满足上式的$(k,b)$不存在。我们考虑更一般的形式$A\mathbf{x} = \mathbf{b}$. 


\newpageorvspace


{\bf 第二题}. 习题9.2第8题. 用向量的内积证明平面外一点到平面的线段长以垂线段最短,并推广到$n$维欧氏空间:

(1) 取平面上任一点为原点$O$, 将空间$V$每一点$P$用向量$\overrightarrow{OP}$表示,则平面是一个2维子空间$W$. 设$A$是空间中给定的任一点,$B$是平面内任一点,分别对应于向量$\alpha = \overrightarrow{OA}, \beta = \overrightarrow{OB}$, 则$\lvert AB \rvert = \lvert \alpha - \beta \rvert$. 求证:当$\alpha - \beta \in W^{\perp}$时$\lvert \alpha - \beta \rvert$取最小值;

(2) 设$E(\mathbb{R})$是欧氏空间,$W$是它的任意子空间,$\alpha$是$E(\mathbb{R})$任意给定的向量。求证:当$\alpha - \beta \in W^{\perp}$时,$\lvert \alpha - \beta \rvert$ $(\beta \in W)$取最小值;

(3) 设$E(\mathbb{R})$是欧氏空间,$\alpha \in E(\mathbb{R})$, $W$是由$\alpha_1, \ldots, \alpha_k \in E(\mathbb{R})$生成的子空间。当$x_1, \ldots, x_k \in \mathbb{R}$满足什么条件时,$\lvert \alpha - (x_1\alpha_1 + \cdots + x_n\alpha_n) \rvert$取最小值?

\newpageorvspace

{\bf 证明}:


\newpageorvspace


{\bf 第三题}. 习题9.2第9题. (1) 设$W$是$\mathbb{R}^3$中过点$(0,0,0), (1,2,2), (3,4,0)$的平面,求点$A (5,0,0)$到平面$W$的最短距离;

(2) 求方程组
$$
\begin{cases}
0.39x - 1.89y = 1 \\
0.61x - 1.80y = 1 \\
0.93x - 1.68y = 1 \\
1.35x - 1.50y = 1
\end{cases}
$$
的最小二乘解。

(3) 设$A\in\mathbb{R}^{m\times n}, X = (x_1,\cdots,x_n)^T, \beta\in\mathbb{R}^{m\times 1}$. 如果实系数线性方程组$AX = \beta$无解,我们可以求$X$使$\mathbb{R}^{m\times 1}$中的向量$\delta = AX - \beta$的长度$\lvert \delta \rvert$取最小值。满足这个条件的解$X$称为方程组$AX = \beta$的最小二乘解。设$A$的各列依次为$\alpha_1, \ldots, \alpha_n$, 则$AX = x_1\alpha_1 + \cdots + x_n\alpha_n$. 求证:
$$\lvert \delta \rvert \text{取最小值} \Longleftrightarrow (\delta, \alpha_i) = 0 (\forall 1\leqslant i \leqslant n) \Longleftrightarrow A^TAX = A^T\beta$$

\newpageorvspace

{\bf 证明}:


\newpageorvspace


{\bf 第四题}. 习题9.3第6题. 给定$0 \neq \alpha \in E_n(\mathbb{R})$. 定义$E_n(\mathbb{R})$中的线性变换$\tau_{\alpha}: \beta \mapsto \beta - \dfrac{2(\beta, \alpha)}{(\alpha, \alpha)} \alpha$, 求证:

(1) $\tau_{\alpha}$是正交变换;

(2) $\tau_{\alpha}$在适当的标准正交基下的矩阵为$\operatorname{diag}(-1, 1, \cdots, 1)$.

\newpageorvspace

{\bf 证明}. 

\end{document}
