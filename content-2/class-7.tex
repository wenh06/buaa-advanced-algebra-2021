\usepackage{relsize}

\renewcommand{\newpageorvspace}{\newpage}
% \renewcommand{\newpageorvspace}{\vspace{2em}}

\date{2022-6-17  第七次习题课}

\begin{document}


\maketitle


\larger[2]


{\bf 第一题}. 习题 9.5 第 9 题. 实数域$\mathbb{R}$上全体$n$阶方阵构成$\mathbb{R}$上$n^2$维线性空间$V = \mathbb{R}^{n\times n}$, 在$V$中定义了内积$\langle X, Y \rangle = \operatorname{tr} \left( XY^T \right)$之后称为欧氏空间。对$V$上如下的线性函数$f$, 求$B\in V$使$f(X) = \langle X, B \rangle.$

(1) $f(X) = \operatorname{tr} X$;

(2) 对给定的$A, D \in \mathbb{R}^{n\times n}$, $f(X) = \operatorname{tr}(AXD)$;

(3) 对给定的$A, D \in \mathbb{R}^{n\times n}$, $f(X) = \operatorname{tr}(AX - XD)$;

(4) 对给定的$\alpha, \beta \in \mathbb{R}^{1\times n}$, $f(X) = \alpha X \beta^T$.

\newpageorvspace

{\bf 解}. (1). 要满足$\operatorname{tr} \left( XY^T \right) = \langle X, B \rangle = f(X) = \operatorname{tr} X$, 只要令$B = I$即可。

(2). 要满足$\operatorname{tr} \left( XY^T \right) = \langle X, B \rangle = f(X) = \operatorname{tr} (AXD) = {\color{red} \operatorname{tr} (XDA)}$, 只要令$B^T = DA$, 即$B = (DA)^T = A^T D^T$即可。

(3). 要满足
\begin{align*}
\operatorname{tr} \left( XY^T \right) & = \langle X, B \rangle = f(X) = \operatorname{tr} (AX - XD) \\
& = {\color{red} \operatorname{tr} (AX) - \operatorname{tr} (XD) = \operatorname{tr} (XA) - \operatorname{tr} (XD)} \\
& = {\color{red} \operatorname{tr} \left( X(A-D) \right)}
\end{align*}
只要令$B^T = A-D$, 即$A = (A-D)^T$即可。

(4). 要满足$\operatorname{tr} \left( XY^T \right) = \langle X, B \rangle = f(X) = \operatorname{tr} \left( \alpha X \beta^T \right) = {\color{red} \operatorname{tr} \left( X \beta^T \alpha \right)}$, 只要令$B^T = \beta^T \alpha$, 即$B = (\beta^T \alpha)^T = \alpha^T \beta$即可。

\vspace{1em}

这题主要利用了方阵迹的性质$\operatorname{tr}(AB) = \operatorname{tr}(BA)$.


\newpageorvspace


{\bf 第二题}. 习题 9.7 第 8 题. 设$U$为酉方阵,且$U^{-1}AU = B$, 证明:$\operatorname{tr} \left( A^* A \right) = \operatorname{tr} \left( B^* B \right).$

\newpageorvspace

{\bf 证明}:因为$U$为酉方阵,所以$U^{-1} = U^*$, 那么
\begin{align*}
\operatorname{tr} \left( B^* B \right) & = \operatorname{tr} \left( \left( U^{-1} A U \right)^* B \right) = \operatorname{tr} \left( \left( U^* A U \right)^* B \right) \\
& = \operatorname{tr} \left( U^* A^* U^{**} B \right) = \operatorname{tr} \left( U^* A^* U B \right) \\
& = \operatorname{tr} \left( A^* U B U^* \right) = \operatorname{tr} \left( A^* U U^* A U U^* \right) \\
& = \operatorname{tr} \left( A^* A \right).
\end{align*}


\newpageorvspace


{\bf 第三题}. 习题 9.7 第 9 题. 设$H_1, H_2$都是$n$阶正定Hermite方阵,且$H_1 - H_2$正定,求证:$H_2^{-1} - H_1^{-1}$正定。

\newpageorvspace

{\bf 证明}:由于$H_1$是$n$阶正定Hermite方阵,那么它共轭相合于$n$阶单位阵,即存在可逆的$n$阶复方阵$P$, 使得$P^* H_1 P = I_n$. 令$\widetilde{H}_2 = P^* H_2 P$, 由于$H_2$是正定Hermite方阵,那么
\begin{itemize}
\item $\widetilde{H}_2^* = \left( P^* H_2 P \right)^* = P^* H_2^* P = P^* H_2 P = \widetilde{H}_2^*$;
\item 任取非零$x \in \mathbb{C}^n$, $\langle x, \widetilde{H}_2 x \rangle = \langle x, P^* H_2 P x \rangle = \langle Px, H_2 P x \rangle > 0$.
\end{itemize}
所以$\widetilde{H}_2$也是正定Hermite方阵,酉相似于对角线全为正实数的对角阵,即存在酉方阵$U$, 使得
$$U^* \widetilde{H}_2 U = \operatorname{diag} (\lambda_1, \ldots, \lambda_n),$$
$\lambda_1, \ldots, \lambda_n > 0$为$\widetilde{H}_2$的特征值。

令$Q = PU$, 那么
\begin{align*}
Q^* H_1 Q & = U^* \left( P^* H_1 P \right) U = U^* I_n U = U^* U = I_n, \\
Q^* H_2 Q & = U^* \left( P^* H_2 P \right) U = U^* \widetilde{H}_2 U = \operatorname{diag} (\lambda_1, \ldots, \lambda_n).
\end{align*}
从而有
$$Q^* (H_1 - H_2) Q = I_n - \operatorname{diag} (\lambda_1, \ldots, \lambda_n) = \operatorname{diag} (1-\lambda_1, \ldots, 1-\lambda_n).$$
因为$H_1 - H_2$正定,所以$Q^* (H_1 - H_2) Q$也是正定的,所以
$$\lambda_1, \ldots, \lambda_n < 1.$$
那么有
\begin{align*}
 Q^{-1} \left( H_2^{-1} - H_1^{-1} \right) \left( Q^* \right)^{-1} & = Q^{-1} H_2^{-1} \left( Q^* \right)^{-1} - Q^{-1} H_1^{-1} \left( Q^* \right)^{-1} \\
& = \left( Q^* H_2 Q \right)^{-1} - \left( Q^* H_1 Q \right)^{-1} \\
& = \left( \operatorname{diag} (\lambda_1, \ldots, \lambda_n) \right)^{-1} - \left( I_n \right)^{-1} \\
& = \operatorname{diag} (\lambda_1^{-1} - 1, \ldots, \lambda_n^{-1} - 1)
\end{align*}
也是正定阵。


\newpageorvspace


{\bf 第四题}. 习题 9.8 第 7 题. 设数域$\mathbb{F}$上$n$维线性空间$V$上定义了非退化斜对称双线性函数$f$.

(1) 证明:$n$是偶数。

(2) 证明:$f$在$V$的适当的基$M$下的矩阵是$H = \begin{pmatrix} O & I_{(m)} \\ -I_{(m)} & O \end{pmatrix}$, 其中$m = \dfrac{n}{2}$.

(3) 证明:$\mathcal{A}$是$V$上的辛变换 $\Leftrightarrow$ $\mathcal{A}$在基$M$下的矩阵$A$满足条件$A^THA = H$.

(4) 满足条件$A^THA = H$的方阵称为辛方阵。要使以下方阵:
$$
\begin{pmatrix} P & O \\ O & Q \end{pmatrix}, \quad
\begin{pmatrix} I & X \\ O & I \end{pmatrix}, \quad
\begin{pmatrix} I & O \\ X & I \end{pmatrix}
$$
是辛方阵,其中的$m$阶块$P, Q, X$应当满足什么样的充分必要条件?

\newpageorvspace

{\bf 证明}. (1) $f$是斜对称双线性函数,那么任取$u, v \in \mathbb{F}^n$, $f(u, v) = - f(v, u)$. 任取$\mathbb{F}^n$的一组基$\varepsilon_1, \ldots, \varepsilon_n$, 令
\begin{align*}
u = a_1 \varepsilon_1 + \cdots + a_n \varepsilon_n, \\
v = b_1 \varepsilon_1 + \cdots + b_n \varepsilon_n,
\end{align*}
那么
\begin{align*}
f(u, v) & = f(a_1 \varepsilon_1 + \cdots + a_n \varepsilon_n, b_1 \varepsilon_1 + \cdots + b_n \varepsilon_n) \\
& = \sum\limits_{1 \leqslant i,j \leqslant n} a_i b_j f(\varepsilon_i, \varepsilon_j) = \alpha^T H \beta
\end{align*}
其中$\alpha = (a_1, \ldots, a_n)^T, \beta = (b_1, \cdots, b_n)^T, H = \begin{pmatrix} f(\varepsilon_1, \varepsilon_1) & \cdots & f(\varepsilon_1, \varepsilon_n) \\ \vdots & & \vdots \\ f(\varepsilon_n, \varepsilon_1) & \cdots & f(\varepsilon_n, \varepsilon_n) \end{pmatrix}.$ 由于$f(\varepsilon_i, \varepsilon_j) = - f(\varepsilon_j, \varepsilon_i)$, 所以$H$是反对称阵。

$f$是非退化的,也就是说任取非零的$u \in \mathbb{F}^n$, 都存在$v \in \mathbb{F}^n$, 使得$f(u, v) \neq 0$. 假设$\det H = 0$, 那么$H$列不满秩,即存在不全为$0$的$t_1, \ldots, t_n$, 使得
\begin{align*}
0 & = t_1 \begin{pmatrix} f(\varepsilon_1, \varepsilon_1) \\ \vdots \\ f(\varepsilon_n, \varepsilon_1) \end{pmatrix} + \cdots + t_n \begin{pmatrix} f(\varepsilon_1, \varepsilon_n) \\ \vdots \\ f(\varepsilon_n, \varepsilon_n) \end{pmatrix} \\
& = \begin{pmatrix} f(\varepsilon_1, t_1 \varepsilon_1 + \cdots + t_n \varepsilon_n) \\ \vdots \\ f(\varepsilon_n, t_1 \varepsilon_1 + \cdots + t_n \varepsilon_n) \end{pmatrix},
\end{align*}
那么对非零向量$u = t_1 \varepsilon_1 + \cdots + t_n \varepsilon_n$, 不存在向量$v$使得$f(u, v) \neq 0$, 这与$f$非退化矛盾。

综上,$H$是一个可逆的反对称阵,于是
$$\det H = \det (-H) = (-1)^n \det H,$$
故$n$为偶数。

(2) 根据(1)小问,要证明$f$在$V$的适当的基$M = \{ \varepsilon_1, \ldots, \varepsilon_n \}$下的矩阵是$H = \begin{pmatrix} O & I_{(m)} \\ -I_{(m)} & O \end{pmatrix} = \begin{pmatrix} f(\varepsilon_1, \varepsilon_1) & \cdots & f(\varepsilon_1, \varepsilon_n) \\ \vdots & & \vdots \\ f(\varepsilon_n, \varepsilon_1) & \cdots & f(\varepsilon_n, \varepsilon_n) \end{pmatrix}$, 只要证明
\begin{equation*}
f(\varepsilon_i, \varepsilon_j) = \begin{cases}
1, & j = i + m, \\
-1, & j = i - m \text{ (由上一条件,自动满足) }, \\
0, & \text{其余情况}
\end{cases}
\end{equation*}

我们任取一个非零向量$\alpha_1$, 由于$f$非退化,则存在非零向量$\alpha_2$, 使得$f(\alpha_1, \alpha_2) = a \neq 0$. 那么我们可以考虑令
$$\varepsilon_1 = \alpha_1, \quad \varepsilon_{1+m} = \dfrac{1}{a} \varepsilon_2,$$
即可以满足$f(\varepsilon_1, \varepsilon_{1+m}) = 1$. 考虑$\varepsilon_1, \varepsilon_{1+m}$生成的子空间$W = \operatorname{span} \{\varepsilon_1, \varepsilon_{1+m}\}$. 令
\begin{align*}
W' & = \{ \alpha \in V \ |\ f(\alpha, \varepsilon_1) = f(\alpha, \varepsilon_{1+m}) = 0 \} \\
& = \ker(f(\varepsilon_1, \cdot)) \cap \ker(f(\varepsilon_{1+m}, \cdot)).
\end{align*}

对$V$的任一组基$M_0$, 以及$f$在这组基下的矩阵$H_0$, 在(1)小问中已经证明了若$f$非退化,则$H_0$可逆。对$\alpha \in V$, 令$\alpha_{M_0}$为$\alpha$在这组基下的坐标表示。那么
$$\ker(f(\alpha, \cdot)) \cong \left\{x \in \mathbb{F}^n \ |\ \left( \alpha_{M_0}^T H_0 \right) x = 0 \right\}$$
维数等于$n - 1.$ 从而知$\dim W' = n-2 = 2(m-1)$. 注意,此时也会有$V = W {\color{red}\oplus} W'.$

于是,可以考虑对$m$进行归纳证明。$m = 1$时,可以类似$W$进行证明。若对所有的$n-2 = 2(m-1)$维空间都证明了题干中的结论,那么存在$W'$的一组基$M' = \{ \varepsilon'_1, \ldots, \varepsilon'_{n-2}  \}$使得$f'$在这组基下矩阵是$H' = \begin{pmatrix} O & I_{(m-1)} \\ -I_{(m-1)} & O \end{pmatrix} = \begin{pmatrix} f(\varepsilon'_1, \varepsilon'_1) & \cdots & f(\varepsilon'_1, \varepsilon'_{n-2}) \\ \vdots & & \vdots \\ f(\varepsilon'_{n-2}, \varepsilon'_1) & \cdots & f(\varepsilon'_{n-2}, \varepsilon'_{n-2}) \end{pmatrix},$ 其中$f'$为$f$在$W'$上的限制。那么令
$$M = \left\{ \varepsilon_1, \varepsilon'_1, \ldots, \varepsilon'_{m-1}, \varepsilon_{m+1}, \varepsilon'_m, \ldots, \varepsilon'_{n-2}  \right\},$$
$f$在这组基下的矩阵即为$H = \begin{pmatrix} O & I_{(m)} \\ -I_{(m)} & O \end{pmatrix}.$

(3) 辛空间$(V, f)$上的一个辛变换$\mathcal{A}$换指的是保这种辛结构的线性变换,即
$$f(\mathcal{A} \alpha, \mathcal{A} \beta) = f(\alpha, \beta), \quad \forall \alpha, \beta \in V.$$
对$\alpha \in V$, 令$\alpha_M$为$\alpha$在这组基下的坐标表示。

$\Longrightarrow$: 假设$\mathcal{A}$是$(V, f)$上的辛变换,那么任取$\alpha, \beta \in V,$ 有
$$
\alpha_M^T A^THA \beta_M = (A\alpha_M)^T H (A\beta_M) = f(\mathcal{A} \alpha, \mathcal{A} \beta) = f(\alpha, \beta) = \alpha_M^T H \beta_M
$$
因为$\alpha, \beta$是任取的,所以有$A^THA = H$.

$\Longleftarrow$: 若$A^THA = H$, 那么任取$\alpha, \beta \in V,$ 有
$$f(\mathcal{A} \alpha, \mathcal{A} \beta) = (A\alpha_M)^T H (A\beta_M) = \alpha_M^T A^THA \beta_M = \alpha_M^T H \beta_M = f(\alpha, \beta).$$
因为$\alpha, \beta$是任取的,所以$\mathcal{A}$是辛变换。

(4) 要使得$\begin{pmatrix} P & O \\ O & Q \end{pmatrix}$是辛方阵,依定义,需要满足
$$\begin{pmatrix} O & I_{(m)} \\ -I_{(m)} & O \end{pmatrix} = H = \begin{pmatrix} P & O \\ O & Q \end{pmatrix}^T H \begin{pmatrix} P & O \\ O & Q \end{pmatrix} = \begin{pmatrix} O & P^TQ \\ -Q^TP & O \end{pmatrix},$$
即满足$P^TQ = I_{(m)}$即可。

要使得$\begin{pmatrix} I & X \\ O & I \end{pmatrix}$是辛方阵,依定义,需要满足
$$\begin{pmatrix} O & I_{(m)} \\ -I_{(m)} & O \end{pmatrix} = H = \begin{pmatrix} I & X \\ O & I \end{pmatrix}^T H \begin{pmatrix} I & X \\ O & I \end{pmatrix} = \begin{pmatrix} O & I \\ -I & X^T - X \end{pmatrix},$$
于是,需要满足$X^T = X$.

要使得$\begin{pmatrix} I & O \\ X & I \end{pmatrix}$是辛方阵,依定义,需要满足
$$\begin{pmatrix} O & I_{(m)} \\ -I_{(m)} & O \end{pmatrix} = H = \begin{pmatrix} I & O \\ X & I \end{pmatrix}^T H \begin{pmatrix} I & O \\ X & I \end{pmatrix} = \begin{pmatrix} X - X^T & I \\ -I & O \end{pmatrix},$$
于是,需要满足$X^T = X$.

\newpage

\begin{center}
{\larger 补充内容}
\end{center}

1. 多线性映射,Multilinear Mapping:

设$V_1,\ldots,V_s,W$为实数$\mathbb{R}$上的线性空间。称一个映射
$$f: \prod_{i=1}^s V_i \rightarrow W$$
为一个多线性映射,如果$f$对每一个分量都具有线性性,即对任意的$1\leqslant i \leqslant s$都有
\begin{itemize}
\item $f(\alpha_1,\ldots,\alpha_i+\alpha_i',\ldots,\alpha_s) = f(\alpha_1,\ldots,\alpha_i,\ldots,\alpha_s) + f(\alpha_1,\ldots,\alpha_i',\ldots,\alpha_s)$
\item $f(\alpha_1,\ldots,t\alpha_i,\ldots,\alpha_s) = tf(\alpha_1,\ldots,\alpha_i,\ldots,\alpha_s)$
\end{itemize}

类似于线性映射的情形,我们可以定义多线性映射的加法以及数乘,于是从$\prod\limits_{i=1}^s V_i$到$W$的多线性映射的全体成为了一个线性空间,被记作$\mathcal{L}(V_1,\ldots,V_s;W)$.

多线性映射的例子:

\begin{itemize}
\item 之前在学习方阵行列式的时候,已经提到过行列式$\det$对于矩阵的列(以及行)有多线性性,于是我们可以把$\det$视作是如下的多线性映射(函数)
$$\det: \underbrace{\mathbb{R}^n \times \cdots \times \mathbb{R}^n}_{\text{$n$个}} \rightarrow \mathbb{R}$$
\item 内积也是多线性函数的一个例子
$$\langle \;,\; \rangle: V\times V \rightarrow \mathbb{R},$$
\end{itemize}

\vspace{2em}

2. 张量积:

设$V_1,V_2,\ldots,V_s$为线性空间。如果一个线性空间$T$,以及多线性映射$\tau: \prod\limits_{i=1}^s V_i \to T$ 满足如下的泛性质(Universal Property):
\begin{itemize}
\item 对任意线性空间$W$以及任意多线性映射$f: \prod\limits_{i=1}^s V_i \to W$, $f$都能唯一地通过$\tau$分解:
\[
\begin{tikzcd}
V_1 \times \cdots \times V_s \ar[r, "f"] \ar[d, "\tau"'] & W \\
T \ar[ur, dashrightarrow, "\exists! \widetilde{f}"'] &
\end{tikzcd}
\]
\end{itemize}
那么$T$被称作$V_1,V_2,\ldots,V_s$的张量积(Tensor Product),记作$V_1\otimes V_2\otimes \cdots \otimes V_s$. 直积空间$\prod\limits_{i=1}^s V_i$中的元素$(\alpha_1,\ldots,\alpha_s)$在$\tau$下的像被记作$\alpha_1\otimes\cdots\otimes\alpha_s.$

张量有如下的性质:
\begin{itemize}
\item $\alpha_1\otimes\cdots\otimes(\alpha_i+\alpha_i')\otimes\cdots\otimes\alpha_s = \alpha_1\otimes\cdots\otimes\alpha_i\otimes\cdots\otimes\alpha_s + \alpha_1\otimes\cdots\otimes\alpha_i'\otimes\cdots\otimes\alpha_s$
\item $\alpha_1\otimes\cdots\otimes(k\alpha_i)\otimes\cdots\otimes\alpha_s = k \alpha_1\otimes\cdots\otimes\alpha_s$
\end{itemize}

为了记号方便,考虑2个线性空间的张量积$V_1\otimes V_2$.

在分别取定基$\alpha_1,\ldots,\alpha_n$以及$\beta_1,\ldots,\beta_m$之后,$V_1\otimes V_2$的一组基就可以取作
$$\alpha_i\otimes \beta_j, \quad 1\leqslant i\leqslant n, 1\leqslant j\leqslant m$$
于是,$V_1\otimes V_2$中的所有元素都能表示成有限和
$$\sum\limits_{i,j} k_{ij} \alpha_i\otimes \beta_j.$$
于是,$\dim V_1\otimes V_2 = \dim V_1 \times \dim V_2.$

{\color{red}注意}!一般来说,$V_1\otimes V_2$中元素并不能表示成$\alpha\otimes\beta$, $\alpha\in V_1$, $\beta\in V_2$, 这种形式,例如$\alpha_1\otimes \beta_1 + \alpha_2\otimes\beta_2$.

在取定了$V_i (i=1,2,\ldots,s)$的一组基
$$\varepsilon_{i1},\ldots,\varepsilon_{in_i}$$
之后,$\bigotimes\limits_{i=1}^s V_i$中所有元素都能表示为一个和式
$$\sum\limits_{i_1,i_2,\ldots,i_s} k_{i_1i_2\ldots i_s} \varepsilon_{1i_1} \otimes \cdots \otimes \varepsilon_{si_s}$$
其中$1\leqslant i_1\leqslant \dim V_1 = n_1, \ldots, 1\leqslant i_s\leqslant \dim V_s = n_s$.

张量除了加法与数乘,还可以定义他们的积。为了书写方便,我们任取两个张量空间$V_1\otimes\cdots\otimes V_s$与$W_1\otimes\cdots\otimes W_t$中的两个张量
$$v = \alpha_1\otimes\cdots\otimes \alpha_s, \quad w = \beta_1\otimes\cdots\otimes \beta_t$$
那么可以定义$v$与$w$的积为
$$v\otimes w := \alpha_1\otimes\cdots\otimes \alpha_s \otimes \beta_1\otimes\cdots\otimes \beta_t$$
为张量空间$V_1\otimes\cdots\otimes V_s\otimes W_1\otimes\cdots\otimes W_t$中的元素。

大家可以自行写一写以上两个张量空间中的一般元素
$$\sum k_{i_1,i_2,\ldots,i_s}\alpha_{1i_1} \otimes \cdots \otimes \alpha_{si_s}, \quad \sum h_{j_1,j_2,\ldots,j_t}\beta_{1j_1} \otimes \cdots \otimes \beta_{tj_t}$$
的积的表达式。

作为多维数组的张量:

考虑张量空间$\bigotimes\limits_{i=1}^s V_i$中一般的一个元素
$$\sum\limits_{i_1,i_2,\ldots,i_s} k_{i_1i_2\ldots i_s} \varepsilon_{1i_1} \otimes \cdots \otimes \varepsilon_{si_s}$$
其中$\varepsilon_{i1},\ldots,\varepsilon_{in_i}, i=1,2,\ldots,s,$为线性空间$V_i $的一组基,且令$n_i = \dim V_i$. 那么该元素能够被视作一个$s$维数组
$$
\texttt{double} \quad k[n_1][n_2]...[n_s];
$$
使得数组中$k[i_1][i_2]...[i_s]$的值就是$k_{i_1i_2\ldots i_s}$.

\end{document}
