
% \renewcommand{\newpageorvspace}{\newpage}
\renewcommand{\newpageorvspace}{\vspace{2em}}

\date{2022-09-09  第一次习题课}
\author{}

\begin{document}

\maketitle

% \larger[2]

{\bf 习题1.2 第3题} 已知两个变量$x,y$之间有某种函数关系$y=f(x)$,并且有如下对应值
\begin{table}[H]
    \centering
    \begin{tabular}{|c|c|c|c|c|}
    \toprule
         x & 1 & 2 & 3 & 4 \\ \hline
         y & 2 & 7 & 16 & 29 \\
    \bottomrule
    \end{tabular}
\end{table}
问:$y$是否可能是$x$的二次函数?如果可能,试求出满足要求的二次函数。

\newpageorvspace

{\bf 解}:假设$y$是$x$的二次函数,即存在实数$a,b,c$使得
$$y = a + bx + cx^2$$
我们有
\begin{table}[H]
    \centering
    \begin{tabular}{|c|c|c|c|c|}
    \toprule
         x & 1 & 2 & 3 & 4 \\ \hline
         $x^2$ & 1 & 4 & 9 & 16 \\ \hline
         y & 2 & 7 & 16 & 29 \\
    \bottomrule
    \end{tabular}
\end{table}
上式需要满足
$$\begin{pmatrix} 2 \\ 7 \\ 16 \\ 29 \end{pmatrix} = \begin{pmatrix} 1 & 1 & 1 \\ 1 & 2 & 4 \\ 1 & 3 & 9 \\ 1 & 4 & 16 \end{pmatrix} \begin{pmatrix} a \\ b \\ c \end{pmatrix}$$
高斯消元法化简
\begin{multline*}
\begin{pmatrix}[ccc|c] 1 & 1 & 1 & 2 \\ 1 & 2 & 4 & 7 \\ 1 & 3 & 9 & 16 \\ 1 & 4 & 16 & 29 \end{pmatrix}
\longrightarrow \begin{pmatrix}[ccc|c] 1 & 1 & 1 & 2 \\ 0 & 1 & 3 & 5 \\ 0 & 2 & 8 & 14 \\ 0 & 3 & 15 & 27 \end{pmatrix}
\longrightarrow \begin{pmatrix}[ccc|c] 1 & 1 & 1 & 2 \\ 0 & 1 & 3 & 5 \\ 0 & 1 & 4 & 7 \\ 0 & 1 & 5 & 9 \end{pmatrix} \\
\longrightarrow \begin{pmatrix}[ccc|c] 1 & 1 & 1 & 2 \\ 0 & 1 & 3 & 5 \\ 0 & 0 & 1 & 2 \\ 0 & 0 & 1 & 2 \end{pmatrix}
\longrightarrow \begin{pmatrix}[ccc|c] 1 & 1 & 0 & 0 \\ 0 & 1 & 0 & -1 \\ 0 & 0 & 1 & 2 \\ 0 & 0 & 1 & 2 \end{pmatrix}
\longrightarrow \begin{pmatrix}[ccc|c] 1 & 0 & 0 & 1 \\ 0 & 1 & 0 & -1 \\ 0 & 0 & 1 & 2 \\ 0 & 0 & 1 & 2 \end{pmatrix}
\end{multline*}
得
$$\begin{pmatrix} 1 & 0 & 0 \\ 0 & 1 & 0 \\ 0 & 0 & 1 \\ 0 & 0 & 1 \end{pmatrix} \begin{pmatrix} a \\ b \\ c \end{pmatrix} = \begin{pmatrix} 1 \\ -1 \\ 2 \\ 2 \end{pmatrix}$$
得$(a,b,c) = (1,-1,2)$, 即$y = 1 - x + 2x^2$。

这题也可以用拉格朗日插值多项式来做。一般来说,过$n+1$个点$(x_0, y_0), \ldots, (x_n, y_n)$的$n$次插值多项式为
$$L(x) = \sum\limits_{i=0}^n y_i \prod\limits_{j \neq i}  \dfrac{x-x_j}{x_i-x_j}.$$
一般来说这是一个$n$次多项式,它也有可能退化为更低次的多项式。在这题中,我们有4个点,会得到次数$\leqslant 3$次的多项式。大家可以具体去算一算,最后答案还是$y = 1 - x + 2x^2$. 我们可以用\href{https://gitee.com/wenh06/buaa-advanced-algebra-2021/blob/master/notebooks/class-1.ipynb}{程序}验证这个结果:
\begin{center}
\begin{lstlisting}[language=Python]
import numpy as np
import sympy as sp

def LagrangePolynomial(xs, ys):
    X = sp.symbols("x")
    assert len(xs) == len(ys)
    y = 0
    for k in range(len(xs)):
        t = 1
        for j in range(len(xs)):
            if j != k:
                t = t * ( (X - xs[j]) / (xs[k] - xs[j]) )
        y += t * ys[k]
    return sp.simplify(y)

xs = np.array([1, 2, 3, 4])
ys = np.array([2, 7, 16, 29])

LagrangePolynomial(xs, ys)
\end{lstlisting}
\end{center}

\newpageorvspace

{\bf 习题1.2 第4题} 在实数范围内解线性方程组
$$\begin{cases}
x + 3y + 2z = 4 \\
2x + 5y -3z = -1 \\
4x + 11y + z = 7
\end{cases}$$

这个方程组的解集在3维空间中的图像$\Pi$是什么?

将这个方程组的常数项全部变成0,得到的方程组的解集在3维空间中的图像$\Pi_0$是什么?$\Pi_0$与$\Pi$有什么关系?

\newpageorvspace

{\bf 解}:该线性方程组的增广系数矩阵为
$$\begin{pmatrix}[ccc|c]
  1 & 3 & 2 & 4\\
  2 & 5 & -3 & -1 \\
  4 & 11 & 1 & 7
\end{pmatrix}$$
通过高斯消元法(具体过程类似上一题,这里略去)化为阶梯形
$$\begin{pmatrix}[ccc|c]
  1 & 0 & -19 & -23\\
  0 & 1 & 7 & 9
\end{pmatrix}$$
所以解集为$(x,y,z) = (19t-23,-7t+9,t)$, $t\in\mathbb{R}$. 它在3维空间中的图像$\Pi$是一条直线。

将这个方程组的常数项全部变成0,得到的方程组的解集在3维空间中的图像$\Pi_0$一条过原点的直线。$\Pi_0$与$\Pi$之间可以通过平移相互得到,即$\Pi = \Pi_0 + \begin{pmatrix} -23 \\ 9 \\ 0 \end{pmatrix}$,$\begin{pmatrix} -23 \\ 9 \\ 0 \end{pmatrix}$可以换为原非齐次线性方程组的任意一个特解。

\newpageorvspace

{\bf 习题1.3 第2题} 讨论当$\lambda$取什么值时下面的方程组有解
$$\begin{cases}
\lambda x_1 + x_2 + x_3 = 1 \\
x_1 + \lambda x_2 + x_3 = \lambda \\
x_1 + x_2 + \lambda x_3 = \lambda^2 \\
\end{cases}$$
当方程组有解时求出解来,并讨论$\lambda$取什么值时方程组有唯一解,什么时候有无穷多组解。

\newpageorvspace

{\bf 解}:对增广系数矩阵做行变换
\begin{align*}
& \begin{pmatrix}[ccc|c]
  \lambda & 1 & 1 & 1\\
  1 & \lambda & 1 & \lambda \\
  1 & 1 & \lambda & \lambda^2
\end{pmatrix}
\to
\begin{pmatrix}[ccc|c]
  1 & 1 & \lambda & \lambda^2 \\
  1 & \lambda & 1 & \lambda \\
  \lambda & 1 & 1 & 1
\end{pmatrix} \\
\to &
\begin{pmatrix}[ccc|c]
  1 & 1 & \lambda & \lambda^2 \\
  0 & \lambda-1 & -(\lambda-1) & -\lambda(\lambda-1) \\
  0 & -(\lambda-1) & -(\lambda-1)(\lambda+1) & -(\lambda-1)(\lambda^2+\lambda+1)
\end{pmatrix} \\
\to &
\begin{pmatrix}[ccc|c]
  1 & 1 & \lambda & \lambda^2 \\
  0 & \lambda-1 & -(\lambda-1) & -\lambda(\lambda-1) \\
  0 & 0 & (\lambda-1)(\lambda+2) & (\lambda-1)(\lambda+1)^2
\end{pmatrix}
\end{align*}
所以
\begin{itemize}
\item 当$\lambda = 1$时,增广系数矩阵化为$\begin{pmatrix}[ccc|c] 1 & 1 & 1 & 1 \end{pmatrix}$,原线性方程组有无穷多组解
$$\begin{pmatrix}
  1 - t_1 - t_2 \\ t_1 \\ t_2
\end{pmatrix}, \quad t_1, t_2 \in \mathbb{R}.$$
\item 当$\lambda = -2$时,增广系数矩阵化为
$$\begin{pmatrix}[ccc|c] 1 & 1 & -2 & 4 \\ 0 & -3 & 3 & -6 \\ 0 & 0 & 0 & -3 \end{pmatrix}$$
此时原线性方程组无解。
\item 其余情况,增广系数矩阵可进一步约化
\begin{align*}
& \to
\begin{pmatrix}[ccc|c]
  1 & 1 & \lambda & \lambda^2 \\
  0 & 1 & -1 & -\lambda \\
  0 & 0 & 1 & \frac{(\lambda+1)^2}{\lambda+2}
\end{pmatrix} \to
\begin{pmatrix}[ccc|c]
  1 & 0 & 0 & -\frac{\lambda+1}{\lambda+2} \\
  0 & 1 & 0 & \frac{1}{\lambda+2} \\
  0 & 0 & 1 & \frac{(\lambda+1)^2}{\lambda+2}
\end{pmatrix}
\end{align*}
此时原线性方程组有唯一解
$$
\begin{pmatrix} x_1 \\ x_2 \\ x_3 \end{pmatrix} = \begin{pmatrix} -\frac{\lambda+1}{\lambda+2} \\ \frac{1}{\lambda+2} \\ \frac{(\lambda+1)^2}{\lambda+2} \end{pmatrix}
$$
\end{itemize}

可以用\href{https://gitee.com/wenh06/buaa-advanced-algebra-2021/blob/master/notebooks/class-1.ipynb}{程序}验证答案(在$\mathbb{Q}(\lambda)$中的解,适合一般情况,不适合$\lambda = 1,-2$这样的退化的情况):
\begin{center}
\begin{lstlisting}[language=Python]
import sympy as sp
from sympy.solvers.solveset import linsolve
x, y, z, u = sp.symbols("x,y,z,u")
linsolve(
    [u*x + y + z - 1, x + u*y + z - u, x + y + u*z - u**2],
    (x, y, z)
)
\end{lstlisting}
\end{center}

\newpageorvspace

{\bf 习题1.1 第2题} (1). 求证:如果复数集合的子集$P$包含至少一个非零数,并且对加、减、乘、除(除数不为$0$)封闭,则$P$包含$0,1$,从而是数域。

(2). 求证:所有的数域都包含有理数域。

(3). 求证:集合$F = \{ a+b\sqrt{2} \ |\ a,b\in\mathbb{Q} \}$ 是数域。(其中$\mathbb{Q}$是有理数域。)

(4). 试求包含$\sqrt[3]{2}$ 的最小的数域。

\newpageorvspace

{\bf 解}:(1). 设$P$包含非零数$a$,则$0 = a-a \in P$, $1 = a/a \in P$,所以$P$是数域。

(2). 任意一个数域$F$都包含$0,1$,且对加、减、乘、除封闭,故$F$包含整数$\mathbb{Z}$(包含$0,1$且对加、减封闭),进而包含有理数域$\mathbb{Q}$(对乘、除封闭)。有理数域$\mathbb{Q}$与有限域$\mathbb{F}_p$,$p$为素数,为最小的域,即不真包含更小的域,称为素域。

(3) 集合$F = \{ a+b\sqrt{2} \ |\ a,b\in\mathbb{Q} \}$包含$0,1$ (分别令$(a,b) = (0,0)$与$(a,b) = (1,0)$)。设$a_1, b_1, a_2, b_2 \in \mathbb{Q}$,下面验证
\begin{itemize}
    \item 加、减法封闭:
    $$(a_1+b_1\sqrt{2}) \pm (a_2+b_2\sqrt{2}) = (a_1 \pm a_2) + (b_1 \pm b_2)\sqrt{2}$$
    \item 乘法封闭:
    $$(a_1+b_1\sqrt{2}) \cdot (a_2+b_2\sqrt{2}) = (a_1a_2+2b_1b_2) + (a_1b_2 + a_2b_1)\sqrt{2}$$
    \item 除法封闭($a_2, b_2$不同时为零):
    $$(a_1+b_1\sqrt{2}) / (a_2+b_2\sqrt{2}) = \dfrac{a_1a_2-2b_1b_2}{a_2^2-2b_2^2} + \dfrac{a_2b_1-a_1b_2}{a_2^2-2b_2^2}\sqrt{2}$$
\end{itemize}

(4). 包含$\sqrt[3]{2}$的最小的数域为
$$\mathbb{Q}(\sqrt[3]{2}) = \{ a + b\sqrt[3]{2} + c\sqrt[3]{4} \ |\ a,b,c\in\mathbb{Q} \}$$

很容易看到任何包含$\sqrt[3]{2}$的数域都必须包含$\sqrt[3]{2}, \sqrt[3]{4}$,从而包含$\mathbb{Q}(\sqrt[3]{2})$,那么只要证明$\mathbb{Q}(\sqrt[3]{2})$是一个数域。

解法一:直接验证$\mathbb{Q}(\sqrt[3]{2})$是对加、减、乘、除封闭。其中对加、减、乘封闭好验证。对于除法封闭,只要验证$1/(a + b\sqrt[3]{2} + c\sqrt[3]{4}) \in \mathbb{Q}(\sqrt[3]{2})$,$a,b,c$不全为零,即可。不妨设$a,b,c$都不为零(其余情况更简单)。这种情况下,又不妨设$c=1$,记$\theta = \sqrt[3]{2}$,要证明
$$1/(a + b\theta + \theta^2) \in \mathbb{Q}(\theta)$$
利用带余除法,有
$$\theta^3-2 = (\theta^2 + b\theta + a) (\theta-b) + ((b^2-a)\theta + ab-2)$$
若$b^2-a=0$,则$ab-2\neq 0$,此时有
$$1/(\theta^2 + b\theta + a) = (\theta-b)/(2-ab).$$
若$b^2-a\neq 0$,则
\begin{multline*}
(\theta^2 + b\theta + a) \\
= ((b^2-a)\theta + ab-2) \left(\frac{1}{b^2-a}\theta + \frac{b^3 - 2ab + 2}{(b^2-a)^2} \right) + a - (ab-2)\frac{b^3 - 2ab + 2}{(b^2-a)^2}
\end{multline*}
% $$(\theta^2 + b\theta + a) = ((b^2-a)\theta + ab-2) \left(\frac{1}{b^2-a}\theta + \frac{b^3 - 2ab + 2}{(b^2-a)^2} \right) + a - (ab-2)\frac{b^3 - 2ab + 2}{(b^2-a)^2}$$
将上式回代,有
\begin{multline*}
(\theta^3-2) \left(\frac{1}{b^2-a}\theta + \frac{b^3 - 2ab + 2}{(b^2-a)^2} \right) \\
= (\theta^2 + b\theta + a) \left[ (\theta-b) \left(\frac{1}{b^2-a}\theta + \frac{b^3 - 2ab + 2}{(b^2-a)^2} \right) + 1 \right] - \gamma
\end{multline*}
% $$(\theta^3-2) \left(\frac{1}{b^2-a}\theta + \frac{b^3 - 2ab + 2}{(b^2-a)^2} \right) = (\theta^2 + b\theta + a) \left[ (\theta-b) \left(\frac{1}{b^2-a}\theta + \frac{b^3 - 2ab + 2}{(b^2-a)^2} \right) + 1 \right] - \gamma$$
其中$\gamma = a - (ab-2)\frac{b^3 - 2ab + 2}{(b^2-a)^2}$,只要验证$\gamma\neq 0$,即有
$$1/(\theta^2 + b\theta + a) = \left[ (\theta-b) \left(\frac{1}{b^2-a}\theta + \frac{b^3 - 2ab + 2}{(b^2-a)^2} \right) + 1 \right] / \gamma$$

解法二:

设$m(x) = x^3 - 2$,这是$\sqrt[3]{2}$的所谓的(首一的, monic)极小多项式\footnote{使得满足$f(\sqrt[3]{2})=0$的$\mathbb{Q}$系数多项式中次数最低的,如果要求是首一的,即最高次项系数为1,则这样的多项式是唯一的。}(minimal polynomial over $\mathbb{Q}$)。令$F = \mathbb{Q}[x] / (m(x))$为$\mathbb{Q}$系数多项式全体的等价类组成的集合,其中的等价关系为
$$f_1(x) \sim f_2(x) \quad \Longleftrightarrow \quad \exists g(x) \ s.t.\ f_1(x)-f_2(x) = g(x)m(x)$$
作为一个$\mathbb{Q}$线性空间,$F$的一组基可以取作$1, \overline{x}, \overline{x}^2$。这是因为对于任意一个$f(x) \in \mathbb{Q}[x]$,都存在$g(x), r(x)$,$\deg r(x) < \deg m(x)$,使得
$$f(x) = g(x)m(x) + r(x)$$
于是
$$F = \{ a + b\overline{x} + c\overline{x}^2 \ |\ a,b,c\in\mathbb{Q} \}$$
定义
$$\varphi : F \to \mathbb{Q}(\sqrt[3]{2}), \overline{x} \mapsto \sqrt[3]{2}$$
可以验证
\begin{itemize}
    \item $F$关于加、减、乘、除封闭。加、减、乘封闭很容易验证。任取$f(x) \in \mathbb{Q}[x] \setminus m(x)\cdot\mathbb{Q}[x]$,即$\overline{f} \neq 0 \in F$,那么$f(x)$与$m(x)$互素,即他们的最大公因子为$1$,记作$(f(x), m(x)) = 1$。于是(通过辗转相除法)存在$g_1(x), g_2(x) \in \mathbb{Q}[x]$,使得$g_1(x)f(x) + g_2(x)m(x) = 1$,于是$\overline{f}$在$F$中的逆元即为$\overline{g}_1$。
    \item $\varphi$是一个一一对应(且保运算,即是一个域同构)。
\end{itemize}

\end{document}
