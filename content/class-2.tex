
% \renewcommand{\newpageorvspace}{\newpage}
\renewcommand{\newpageorvspace}{\vspace{2em}}

\date{2022-09-23  第二次习题课}
\author{}

\begin{document}

\maketitle

% \larger[2]

{\bf 第一题}. 设$\alpha_1, \ldots, \alpha_n$线性无关。令$\beta_j = a_{j1}\alpha_1 + \cdots + a_{jn}\alpha_n,$ $j = 1, \ldots, n.$ 令$\gamma_j = \begin{pmatrix} a_{j1} \\ \vdots \\ a_{jn} \end{pmatrix},$ $j = 1, \ldots, n.$ 证明$\beta_1, \ldots, \beta_n$线性无关 $\Longleftrightarrow$ $\gamma_1, \ldots, \gamma_n$ 线性无关。

\newpageorvspace

{\bf 证明}:

\newpageorvspace

{\bf 习题2.2 第6题} 

\newpageorvspace

{\bf 解}:

\newpageorvspace

{\bf 习题2.2 第7题} 

\newpageorvspace

{\bf 解}:

\newpageorvspace

{\bf 习题2.3 第6题} 

\newpageorvspace

{\bf 解}:

{\bf 第五题}. 设$A$是$\mathbb{R}$上$n$阶矩阵。若$\lvert a_{ii} \rvert > \sum\limits_{j\neq i} \lvert a_{ij} \rvert,$ $i = 1, \ldots, n,$ 求矩阵$A$的秩$r(A)$.

\newpageorvspace

{\bf 解}:

\end{document}
