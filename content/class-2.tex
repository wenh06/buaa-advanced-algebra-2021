
% \renewcommand{\newpageorvspace}{\newpage}
\renewcommand{\newpageorvspace}{\vspace{2em}}

\date{2022-09-23  第二次习题课}
\author{}

\begin{document}

\maketitle

% \larger[2]

{\bf 第一题}. 设$\alpha_1, \ldots, \alpha_n$线性无关。令$\beta_j = a_{j1}\alpha_1 + \cdots + a_{jn}\alpha_n,$ $j = 1, \ldots, n.$ 令$\gamma_j = \begin{pmatrix} a_{j1} \\ \vdots \\ a_{jn} \end{pmatrix},$ $j = 1, \ldots, n.$ 证明$\beta_1, \ldots, \beta_n$线性无关 $\Longleftrightarrow$ $\gamma_1, \ldots, \gamma_n$ 线性无关。

\newpageorvspace

{\bf 证明}:$\Longrightarrow$: 假设$\gamma_1, \ldots, \gamma_n$ 线性相关,那么存在不全为0的数$\lambda_1, \ldots, \lambda_n$使得
$$\lambda_1 \gamma_1 + \cdots + \lambda_n \gamma_n = 0.$$
那么有
\begin{align*}
& ~~ \lambda_1 \beta_1 + \cdots + \lambda_n \beta_n \\
= & ~~ \lambda_1 (a_{11}\alpha_1 + \cdots + a_{1n}\alpha_n) + \cdots + \lambda_n (a_{n1}\alpha_1 + \cdots + a_{nn}\alpha_n) \\
= & ~~ (\lambda_1 a_{11} + \cdots + \lambda_n a_{n1}) \alpha_1 + \cdots + (\lambda_1 a_{1n} + \cdots + \lambda_n a_{nn}) \alpha_n \\
= & ~~ (\alpha_1, \cdots, \alpha_n) \begin{pmatrix} \lambda_1 a_{11} + \cdots + \lambda_n a_{n1} \\ \vdots \\ \lambda_1 a_{1n} + \cdots + \lambda_n a_{nn} \end{pmatrix} \\
= & ~~ (\alpha_1, \cdots, \alpha_n) \left( \begin{pmatrix} \lambda_1 a_{11} \\ \vdots \\ \lambda_1 a_{1n} \end{pmatrix} + \cdots + \begin{pmatrix} \lambda_n a_{n1} \\ \vdots \\ \lambda_n a_{nn} \end{pmatrix} \right) \\
= & ~~ (\alpha_1, \cdots, \alpha_n) (\lambda_1 \gamma_1 + \cdots + \lambda_n \gamma_n) \\
= & ~~ 0.
\end{align*}
这与$\beta_1, \ldots, \beta_n$线性无关的条件矛盾。

$\Longleftarrow$: 假设$\beta_1, \ldots, \beta_n$ 线性相关,那么存在不全为0的数$\lambda_1, \ldots, \lambda_n$使得
$$\lambda_1 \beta_1 + \cdots + \lambda_n \beta_n = 0.$$
上式左边有
$$\lambda_1 \beta_1 + \cdots + \lambda_n \beta_n = (\lambda_1 a_{11} + \cdots + \lambda_n a_{n1}) \alpha_1 + \cdots + (\lambda_1 a_{1n} + \cdots + \lambda_n a_{nn}) \alpha_n$$
由于$\alpha_1, \ldots, \alpha_n$线性无关,所以有
$$
\begin{cases}
\lambda_1 a_{11} + \cdots + \lambda_n a_{n1} = 0\\
\hspace{3em} \vdots \\
\lambda_1 a_{1n} + \cdots + \lambda_n a_{nn} = 0
\end{cases}
$$
排成列向量的形式,即有
$$0 = \begin{pmatrix} 0 \\ \vdots \\ 0 \end{pmatrix} = \begin{pmatrix} \lambda_1 a_{11} + \cdots + \lambda_n a_{n1} \\ \vdots \\ \lambda_1 a_{1n} + \cdots + \lambda_n a_{nn} \end{pmatrix} = \lambda_1 \begin{pmatrix} a_{11} \\ \vdots \\ a_{1n} \end{pmatrix} + \cdots + \lambda_n \begin{pmatrix} a_{n1} \\ \vdots \\ a_{nn} \end{pmatrix} = \lambda_1 \gamma_1 + \cdots + \lambda_n \gamma_n.$$
这与$\gamma_1, \ldots, \gamma_n$ 线性无关矛盾。

\newpageorvspace

{\bf 习题2.2 第6题}. 设向量组$\alpha_1, \ldots, \alpha_n$的秩是$r.$ 求证:

(1). $\alpha_1, \ldots, \alpha_n$中任意$r$个线性无关向量都是极大线性无关组。

(2). 设$\alpha_1, \ldots, \alpha_n$能被其中某$r$个向量$\beta_1, \ldots, \beta_r$线性表出,则$\beta_1, \ldots, \beta_r$线性无关。

\newpageorvspace

{\bf 证明}: (1). 任取$\alpha_1, \ldots, \alpha_n$中$r$个线性无关向量$\alpha_{i_1}, \ldots, \alpha_{i_r}$, 假设它不是极大线性无关组,那么存在$j, 1\leqslant j \leqslant n, j\not\in \{i_1, \ldots, i_r \},$ 使得$\alpha_{i_1}, \ldots, \alpha_{i_r}, \alpha_j$线性无关。这与$\alpha_1, \ldots, \alpha_n$的秩为$r$是矛盾的。

(2). 假设$\beta_1, \ldots, \beta_r$线性相关,那么存在不全为0的数$\lambda_1, \ldots, \lambda_r$使得$\lambda_1\beta_1 + \cdots + \lambda_r\beta_r = 0.$ 任取$\alpha_1, \ldots, \alpha_n$的一个极大线性无关组$\alpha_{i_1}, \ldots, \alpha_{i_r}$. 由于$\alpha_1, \ldots, \alpha_n$能被$\beta_1, \ldots, \beta_r$线性表出,所以存在系数$a_{11}, \ldots, a_{rr},$ 使得
$$
\begin{cases}
\alpha_{i_1} = a_{11}\beta_1 + \cdots + a_{r1}\beta_r \\
\hspace{3em} \vdots \\
\alpha_{i_r} = a_{1r}\beta_1 + \cdots + a_{rr}\beta_r,
\end{cases}
$$
写成矩阵形式$(\alpha_{i_1}, \ldots, \alpha_{i_r}) = (\beta_1, \ldots, \beta_r) A$, 其中$A = (a_{ij})_{1\leqslant i,j \leqslant r}.$ 不妨设方阵$A$是满秩的,否则齐次线性方程组$Ax = 0$有非零解,此时任取其中一个非零解$x = (x_1, \ldots, x_r),$ 即有
$$x_1\alpha_{i_1} + \cdots + x_r\alpha_{i_r} = (\beta_1, \ldots, \beta_r) Ax = 0,$$
这与$\alpha_{i_1}, \ldots, \alpha_{i_r}$是线性无关组是矛盾的。

令$t_1, \ldots, t_r$为未知数,考虑
\begin{align*}
t_1 \alpha_{i_1} + \cdots + t_r \alpha_{i_r} & = t_1 (a_{11}\beta_1 + \cdots + a_{r1}\beta_r) + \cdots + t_r (a_{1r}\beta_1 + \cdots + a_{rr}\beta_r) \\
& = (t_1 a_{11} + \cdots + t_r a_{1r}) \beta_1 + \cdots + (t_1 a_{r1} + \cdots + t_r a_{rr}) \beta_r \\
& = (\beta_1, \ldots, \beta_r) A \begin{pmatrix} t_1 \\ \vdots \\ t_r \end{pmatrix}
\end{align*}
由于$A$是满秩方阵,那么非齐次线性方程组$A \begin{pmatrix} t_1 \\ \vdots \\ t_r \end{pmatrix} = \begin{pmatrix} \lambda_1 \\ \vdots \\ \lambda_r \end{pmatrix}$有唯一解,记为$(x_1, \ldots, x_r).$ 由于$\lambda_1, \ldots, \lambda_r$不全为0,所以$x_1, \ldots, x_r$不全为0. 那么有
$$x_1 \alpha_{i_1} + \cdots + x_r \alpha_{i_r} = (\beta_1, \ldots, \beta_r) A \begin{pmatrix} x_1 \\ \vdots \\ x_r \end{pmatrix} = (\beta_1, \ldots, \beta_r) \begin{pmatrix} \lambda_1 \\ \vdots \\ \lambda_r \end{pmatrix} = \lambda_1\beta_1 + \cdots + \lambda_r\beta_r = 0$$
这还是与$\alpha_{i_1}, \ldots, \alpha_{i_r}$是线性无关组矛盾。故假设不成立,$\beta_1, \ldots, \beta_r$必定是线性无关的。

\newpageorvspace

{\bf 习题2.2 第7题}. 证明:若向量组(I)可以由向量组(II)线性表出,则(I)的秩不超过(II)的秩。

\newpageorvspace

{\bf 解}: 这其实已经在上一题(习题2.2 第6题第(2)问)中证明了,因为在证明过程中,我们并没有假设$\beta_1, \cdots, \beta_r$是取自$\alpha_1, \ldots, \alpha_n$中的。事实上,假设向量组(I)秩为$r$,可以由秩为$s$的向量组(II)线性表出,且$r > s$, 那么取向量组(II)的一个极大线性无关组,再从向量组(II)可重复地随机添加$r-s$个向量,构成一个数量为$r$的向量组,那么这个向量组可以线性表出向量组(I)可以由这个向量组线性表出。根据上一题结论,这个向量组秩为$r$,矛盾。

\newpageorvspace

{\bf 第四题}. 设$A$是$\mathbb{R}$上$n$阶方阵。若$\lvert a_{ii} \rvert > \sum\limits_{j\neq i} \lvert a_{ij} \rvert,$ $i = 1, \ldots, n,$ 求矩阵$A$的秩$r(A)$.

\newpageorvspace

{\bf 解}: 假设$A$不满秩,那么$A = (\alpha_1, \cdots, \alpha_n)$列不满秩,其中$\alpha_1, \ldots, \alpha_n$是$A$的列。那么存在不全为0的实数$\lambda_1, \ldots, \lambda_n$, 使得
$$0 = \lambda_1 \alpha_1 + \cdots + \lambda_n \alpha_n = \begin{pmatrix} \lambda_1 a_{11} + \cdots + \lambda_n a_{1n} \\ \vdots \\ \lambda_1 a_{n1} + \cdots + \lambda_n a_{nn} \end{pmatrix}.$$
令$i_0 = \operatorname{argmax} \{ \lvert \lambda_i \rvert \ |\ i = 1, \ldots, n \}$为满足$\lambda_i$中绝对值最大者的下标(如果有不止一个则任取一个即可)。考察上式右边第$i_0$位的元素
$$\lambda_{1} a_{i_01} + \cdots + \lambda_{i_0} a_{i_0i_0} + \cdots \lambda_n a_{i_0n} = 0.$$
那么会有
\begin{align*}
& ~~ \lvert \lambda_{1} a_{i_01} + \cdots + \lambda_{i_0-1} a_{i_0,i_0-1} + \lambda_{i_0+1} a_{i_0,i_0+1} + \cdots + \lambda_n a_{i_0n} \rvert \\
= & ~~ \lvert \lambda_{i_0} a_{i_0i_0} \rvert = \lvert \lambda_{i_0} \rvert \cdot \lvert a_{i_0i_0} \rvert ~~ {\color{red} >} ~~ \lvert \lambda_{i_0} \rvert \sum\limits_{j\neq i_0} \lvert a_{i_0j} \rvert \\
\geqslant & ~~ \lvert \lambda_{1} a_{i_01} \rvert + \cdots + \lvert \lambda_{i_0-1} a_{i_0,i_0-1} \rvert + \lvert \lambda_{i_0+1} a_{i_0,i_0+1} \rvert + \cdots + \lvert \lambda_n a_{i_0n} \rvert.
\end{align*}
上述不等式是不可能成立的。所以假设不成立,级$A$是满秩的,$r(A) = n.$

满足题设条件的方阵被称为(强)对角优势矩阵(diagonally dominant matrix)。这样的矩阵都是非奇异的。如果学了方阵{\color{red}特征值}的知识,此题还可以利用如下结论证明:

\begin{quote}
设$A$是复数域$\mathbb{C}$上$n$阶方阵,对$i = 1, \ldots, n$, 令
\begin{align*}
R_{i} & = \sum_{j\neq {i}}\lvert a_{ij} \rvert, \\
D(a_{ii}, R_{i}) & = \left\{ x \in \mathbb{C} \ |\ \lvert x - a_{ii} \rvert \leqslant R_{i} \right\}.
\end{align*}
那么$A$的每个{\color{red}特征值}都落在某个$D(a_{ii}, R_{i})$中。
\end{quote}

\newpageorvspace

{\bf 习题2.3 第6题}. 设$S, T$是向量组。求证:$S$与$T$等价$\Longleftrightarrow$ $\operatorname{rank} S = \operatorname{rank} (S\cup T) = \operatorname{rank} T.$

\newpageorvspace

{\bf 证明}: $\Longrightarrow$: 若$S$与$T$等价,则他们可以相互线性表出,那么根据上一题(习题2.2 第7题),有$\operatorname{rank} S \leqslant \operatorname{rank} T$, 以及$\operatorname{rank} T \leqslant \operatorname{rank} S$. 所以必须有$\operatorname{rank} S = \operatorname{rank} T.$ 对于向量组$S\cup T,$ 它里面的每一个向量,要么属于$S$, 要么属于$T$, 都可以用$S$的向量线性表出。反之,$S$中每一个向量显然可以用向量组$S\cup T$的向量线性表出。所以向量组$S$与$S\cup T$线性等价,从而有$\operatorname{rank} S = \operatorname{rank} (S\cup T).$

$\Longleftarrow$: 令向量组$S\cup T$的秩为$r$. 任取$S$的一个极大线性无关组,记为$\alpha_1, \ldots, \alpha_r$. 那么它们也是向量组$S\cup T$中的$r$个线性无关的的向量,根据习题2.2 第6题第(1)小题的结论,$\alpha_1, \ldots, \alpha_r$是向量组$S\cup T$的一个极大线性无关组,从而可以线性表出其中每一个向量。$T$作为$S\cup T$的一个子集,其中每一个向量也能被$\alpha_1, \ldots, \alpha_r$线性表出。所以$T$能被$S$线性表出。

同样地可以证明$S$能被$T$线性表出。二者可以相互线性表出,所以它们是线性等价的。

\newpageorvspace

{\bf 习题2.3 第7题}. 求证:两个齐次线性方程组(I), (II)同解的充分必要条件是他们互为线性组合。

\newpageorvspace

{\bf 证明}: 考虑齐次线性方程组$A_1X = 0$与$A_2X = 0,$ 其中$A_1, A_2$分别是$m_1\times n$与$m_2\times n$的矩阵。将$A_1, A_2$的行向量组分别记作$S_1, S_2$, 那么线性方程组$\begin{pmatrix}A_1 \\ A_2\end{pmatrix} X = 0$的行向量组即为$S_1 \cup S_2$. 对于齐次线性方程组,我们知道
$$\left( \begin{pmatrix}A_1 \\ A_2\end{pmatrix} X = 0 \text{ 的解空间 } \right) = \left( A_1 X = 0 \text{ 的解空间 } \right) \cap \left( A_2X = 0 \text{ 的解空间 } \right)$$
那么
\begin{align*}
& ~~ A_1X = 0 \text{ 与 } A_2X = 0 \text{ 同解 } \\
\Longleftrightarrow & ~~ A_1X = 0, A_2X = 0, \begin{pmatrix}A_1 \\ A_2\end{pmatrix} X = 0 \text{ 同解 } \\
\Longleftrightarrow & ~~ \operatorname{rank} S_1 = \operatorname{rank} (S_1 \cup S_2) = \operatorname{rank} S_2 \\
\Longleftrightarrow & ~~ S_1, S_2 \text{ 线性等价 }
\end{align*}
最后一个等价是根据习题2.3 第6题。倒数第二个等价的解释如下:

$\begin{pmatrix}A_1 \\ A_2\end{pmatrix} X = 0$的解空间是$A_1X = 0$(或$A_2X = 0$)的子空间,其维数分别为$n - \operatorname{rank} (S_1\cup S_2)$以及$n - \operatorname{rank} S_1$(或者$n - \operatorname{rank} S_2$)。一个线性空间与它的一个子空间相等,当且仅当他们维数是相等的,即
$$n - \operatorname{rank} (S_1\cup S_2) = n - \operatorname{rank} S_1 ~~ (\text{或} ~~ n - \operatorname{rank} S_2),$$
即
$$\operatorname{rank} (S_1\cup S_2) = \operatorname{rank} S_1 ~~ (\text{或} ~~ \operatorname{rank} S_2).$$

\end{document}
