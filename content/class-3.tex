
% \renewcommand{\newpageorvspace}{\newpage}
\renewcommand{\newpageorvspace}{\vspace{2em}}

\date{2022-10-待定  第三次习题课}
\author{}

\begin{document}

\maketitle

% \larger[2]

{\bf 习题2.1 第7题}. (1). 若$\alpha_1, \ldots, \alpha_n$线性无关,问$\alpha_1 + \alpha_2, \alpha_2 + \alpha_3, \ldots, \alpha_{n-1} + \alpha_n, \alpha_n + \alpha_1$是否一定线性无关?为什么?

(2). 若$\alpha_1, \ldots, \alpha_n$线性相关,问$\alpha_1 + \alpha_2, \alpha_2 + \alpha_3, \ldots, \alpha_{n-1} + \alpha_n, \alpha_n + \alpha_1$是否一定线性相关?为什么?

\newpageorvspace

{\bf 解}: 考虑
$$\lambda_1 (\alpha_1+\alpha_2) + \lambda_2 (\alpha_2+\alpha_3) + \cdots + \lambda_n (\alpha_n+\alpha_1) = 0$$
$\lambda_1, \cdots, \lambda_n\in\mathbb{R}$, 变形为
$$(\alpha_1, \cdots, \alpha_n)
\underbrace{\begin{pmatrix}
1 & 0 & \cdots & \cdots & 0 & 1 \\
1 & 1 & 0 & & & 0 \\
0 & 1 & 1 & \ddots & & \vdots \\
\vdots & \ddots & \ddots & \ddots & \ddots & \vdots \\
\vdots & & \ddots & \ddots & \ddots & 0 \\
0 & \cdots & \cdots & 0 & 1 & 1
\end{pmatrix}
\begin{pmatrix}
\lambda_1 \\ \vdots \\ \lambda_n
\end{pmatrix}
}_{\text{记为} A \lambda}= 0
$$
我们考虑齐次线性方程组$A\lambda = 0$的解的情况。对系数矩阵$A$进行从上往下进行上一行乘以-1加到下一行的操作,有
$$
A = \begin{pmatrix}
1 & 0 & \cdots & \cdots & 0 & 1 \\
1 & 1 & 0 & & & 0 \\
0 & 1 & 1 & \ddots & & \vdots \\
\vdots & \ddots & \ddots & \ddots & \ddots & \vdots \\
\vdots & & \ddots & \ddots & \ddots & 0 \\
0 & \cdots & \cdots & 0 & 1 & 1
\end{pmatrix}
\to
\begin{pmatrix}
1 & 0 & \cdots & \cdots & 0 & 1 \\
0 & 1 & 0 & & & -1 \\
0 & 0 & 1 & \ddots & & 1 \\
\vdots & \ddots & \ddots & \ddots & \ddots & \vdots \\
\vdots & & \ddots & 0 & 1 & (-1)^{n-2} \\
0 & \cdots & \cdots & 0 & 0 & 1 + (-1)^{n-1}
\end{pmatrix}
$$
当$n$为偶数时,此时化简后的系数矩阵最后一行全为零,故$A\begin{pmatrix}
\lambda_1 \\ \vdots \\ \lambda_n
\end{pmatrix} = 0$有非零解,此时$\alpha_1+\alpha_2, \alpha_2+\alpha_3, \alpha_n+\alpha_1$一定线性相关。

当$n$为奇数时,化简后的系数矩阵满秩,此时$\alpha_1+\alpha_2, \alpha_2+\alpha_3, \cdots, \alpha_n+\alpha_1$线性相关当且仅当$\alpha_1, \alpha_2, \cdots, \alpha_n$线性相关。

\newpageorvspace

{\bf 习题2.1 第8题}. 设复数域上的向量$\alpha_1, \ldots, \alpha_n$线性无关。$\lambda$取什么复数值的时候,向量$\alpha_1 - \lambda \alpha_2, \alpha_2 - \lambda \alpha_3, \ldots, \alpha_{n-1} - \lambda \alpha_n, \alpha_n - \lambda \alpha_1$线性无关?

\newpageorvspace

{\bf 解}: 类似上一题习题2.1 第7题,将问题写为
$$(\alpha_1, \cdots, \alpha_n)
\underbrace{\begin{pmatrix}
1 & 0 & \cdots & \cdots & 0 & -\lambda \\
-\lambda & 1 & 0 & & & 0 \\
0 & -\lambda & 1 & \ddots & & \vdots \\
\vdots & \ddots & \ddots & \ddots & \ddots & \vdots \\
\vdots & & \ddots & \ddots & \ddots & 0 \\
0 & \cdots & \cdots & 0 & -\lambda & 1
\end{pmatrix}
\begin{pmatrix}
z_1 \\ \vdots \\ z_n
\end{pmatrix}
}_{\text{记为} Az}= 0
$$
$z_1, \cdots, z_n \in \mathbb{C}$.对系数矩阵$A$进行从上往下进行上一行乘以$\lambda$加到下一行的操作,有
$$
A = \begin{pmatrix}
1 & 0 & \cdots & \cdots & 0 & -\lambda \\
-\lambda & 1 & 0 & & & 0 \\
0 & -\lambda & 1 & \ddots & & \vdots \\
\vdots & \ddots & \ddots & \ddots & \ddots & \vdots \\
\vdots & & \ddots & \ddots & \ddots & 0 \\
0 & \cdots & \cdots & 0 & -\lambda & 1
\end{pmatrix}
\to
\begin{pmatrix}
1 & 0 & \cdots & \cdots & 0 & -\lambda \\
0 & 1 & 0 & & & -\lambda^2 \\
0 & 0 & 1 & \ddots & & 1 \\
\vdots & \ddots & \ddots & \ddots & \ddots & \vdots \\
\vdots & & \ddots & 0 & 1 & -\lambda\cdot\lambda^{n-2} \\
0 & \cdots & \cdots & 0 & 0 & 1-\lambda\cdot\lambda^{n-1}
\end{pmatrix}
$$
若$\alpha_1,\cdots,\alpha_n$线性无关,则只需要$1-\lambda^{n} \neq 0$,即$\lambda$不取$n$次单位根$e^{2k\pi i/n},$ $k = 0, \ldots, n-1,$ 即可使$\alpha_1-\lambda\alpha_2,\cdots,\alpha_n-\lambda\alpha_1$线性无关。

\end{document}
