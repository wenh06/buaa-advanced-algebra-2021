
% \renewcommand{\newpageorvspace}{\newpage}
\renewcommand{\newpageorvspace}{\vspace{2em}}

\date{2022-10-待定  第三次习题课}
\author{}

\begin{document}

\maketitle

% \larger[2]

{\bf 第一题}. 设$V_1, V_2$是$\mathbb{F}$上线性空间,$f$是$V_1$到$V_2$的一个同构映射,证明$f^{-1}$是$V_2$到$V_1$的同构映射。

\newpageorvspace

{\bf 证明}: 

\newpageorvspace

{\bf 习题2.8 第3题}. 求多项式$f(x)$使$f(1) = 1, f(2) = 2, f(3) = 4.$ 这样的多项式$f(x)$是否可能是整系数多项式?

\newpageorvspace

{\bf 解}: 一般给定$n$个点$(x_1,y_1), \dots, (x_n,y_n)$, 其中$x_1, \cdots, x_n$互不相同,则有拉格朗日插值多项式
\begin{gather*}
L(x) = \sum_{i=1}^{n}y_{i}\ell_{i}(x) \\
\text{其中, } \ell_{i}(x) = \prod_{\begin{smallmatrix}1\leq \leq k \\ j\neq i\end{smallmatrix}}{\frac {x-x_{j}}{x_{i}-x_{j}}}={\frac {(x-x_{0})}{(x_{i}-x_{0})}}\cdots {\frac {(x-x_{i-1})}{(x_{i}-x_{i-1})}}{\frac {(x-x_{i+1})}{(x_{i}-x_{i+1})}}\cdots {\frac {(x-x_{n})}{(x_{i}-x_{n})}},
\end{gather*}
满足$L(x_i) = y_i, i=1, \dots ,n$。本题对应的拉格朗日插值多项式为$f(x) = L(x) = \dfrac{1}{2}x^2 - \dfrac{1}{2}x + 1$. 这是之前习题课讲过的知识。当然也存在更高次多项式满足题目条件,例如$(x-1)^n(x-2)^m(x-3)^k + L(x)$.

假设存在整系数的多项式$f(x)$满足题设条件,那么考察$g(x) = f(x) - L(x).$ 由于$1, 2, 3$是$g(x)$的根,于是存在多项式$h(x)$使得
$$g(x) = f(x) - L(x) = (x-1)(x-2)(x-3) h(x)$$
于是在$\mathbb{R}[X]$中有
$$f(x) \equiv L(x) \mod (x-1)(x-2)(x-3)$$
由于$f(X)$在$\mathbb{R}[x]$中利用辗转相除法除以$(x-1)(x-2)(x-3)$所得的商多项式与余多项式都是整系数的,所以上式不可能成立。

\newpageorvspace

{\bf 第三题}. 已知$\mathbb{F}$上线性空间$V$的有限个真子空间之并不等于$V.$ 若$\dim(V) = n,$ $V_1, \ldots, V_s$是$V$的$s$个真子空间。证明$\displaystyle V \setminus \bigcup_{i=1}^s V_i$中包含$V$的一组基。

\newpageorvspace

{\bf 证明}: 由于$\displaystyle \bigcup_{i=1}^s V_i \subsetneq V$, 所以可以从$V \setminus \bigcup_{i=1}^s V_i \neq \emptyset$中取到至少一个(非零)向量,记作$\alpha_1.$ 令$W_1 = \operatorname{span}_{\mathbb{F}}\{\alpha_1\},$ 那么$W_1$是$V$的真子空间。于是,我们可以从$V \setminus \left( W_1 \cup \left( \bigcup_{i=1}^s V_i \right) \right) \neq \emptyset$中取到(非零)向量$\alpha_2,$ 并令$W_2 = \operatorname{span}_{\mathbb{F}}\{\alpha_1, \alpha_2\}$. 由于$\alpha_2 \not\in W_1$, 所以$\dim W_2 = 2$.

依此方法,我们可以取到$\alpha_1, \alpha_2, \ldots, \alpha_{n-1}$, 以及相应的$W_i = \operatorname{span}_{\mathbb{F}}\{\alpha_1, \ldots, \alpha_i\}, \dim (W_i) = i, i=1, \ldots, n-1$. 最后考察$V \setminus \left( W_{n-1} \cup \left( \bigcup_{i=1}^s V_i \right) \right) \neq \emptyset,$ 从这个集合中取(非零)向量$\alpha_n$. 这样我们就得到了$V$的一组基$\alpha_1, \ldots, \alpha_n.$

\newpageorvspace

{\bf 第四题}. 给定$n$元排列$(j_1j_2\cdots j_n),$ 若$\tau(j_1j_2\cdots j_n) = r,$ 求$n$元排列$(j_n\cdots j_2j_1)$的逆序数$\tau(j_n\cdots j_2j_1)$.

\newpageorvspace

{\bf 解}: 一个$n$元排列的序关系总共有$n(n-1)/2$对。排列$(j_1j_2\cdots j_n)$中的顺序关系是排列$(j_n\cdots j_2j_1)$中的逆序关系;排列$(j_1j_2\cdots j_n)$中的逆序关系是排列$(j_n\cdots j_2j_1)$中的顺序关系。因此
$$\tau(j_n\cdots j_2j_1) = n(n-1)/2 - \tau(j_1j_2\cdots j_n) = n(n-1)/2 - r.$$

\newpageorvspace

{\bf 习题2.1 第7题}. (1). 若$\alpha_1, \ldots, \alpha_n$线性无关,问$\alpha_1 + \alpha_2, \alpha_2 + \alpha_3, \ldots, \alpha_{n-1} + \alpha_n, \alpha_n + \alpha_1$是否一定线性无关?为什么?

(2). 若$\alpha_1, \ldots, \alpha_n$线性相关,问$\alpha_1 + \alpha_2, \alpha_2 + \alpha_3, \ldots, \alpha_{n-1} + \alpha_n, \alpha_n + \alpha_1$是否一定线性相关?为什么?

\newpageorvspace

{\bf 解}: 考虑
$$\lambda_1 (\alpha_1+\alpha_2) + \lambda_2 (\alpha_2+\alpha_3) + \cdots + \lambda_n (\alpha_n+\alpha_1) = 0$$
$\lambda_1, \cdots, \lambda_n\in\mathbb{R}$, 变形为
$$(\alpha_1, \cdots, \alpha_n)
\underbrace{\begin{pmatrix}
1 & 0 & \cdots & \cdots & 0 & 1 \\
1 & 1 & 0 & & & 0 \\
0 & 1 & 1 & \ddots & & \vdots \\
\vdots & \ddots & \ddots & \ddots & \ddots & \vdots \\
\vdots & & \ddots & \ddots & \ddots & 0 \\
0 & \cdots & \cdots & 0 & 1 & 1
\end{pmatrix}
\begin{pmatrix}
\lambda_1 \\ \vdots \\ \lambda_n
\end{pmatrix}
}_{\text{记为} A \lambda}= 0
$$
我们考虑齐次线性方程组$A\lambda = 0$的解的情况。对系数矩阵$A$进行从上往下进行上一行乘以-1加到下一行的操作,有
$$
A = \begin{pmatrix}
1 & 0 & \cdots & \cdots & 0 & 1 \\
1 & 1 & 0 & & & 0 \\
0 & 1 & 1 & \ddots & & \vdots \\
\vdots & \ddots & \ddots & \ddots & \ddots & \vdots \\
\vdots & & \ddots & \ddots & \ddots & 0 \\
0 & \cdots & \cdots & 0 & 1 & 1
\end{pmatrix}
\to
\begin{pmatrix}
1 & 0 & \cdots & \cdots & 0 & 1 \\
0 & 1 & 0 & & & -1 \\
0 & 0 & 1 & \ddots & & 1 \\
\vdots & \ddots & \ddots & \ddots & \ddots & \vdots \\
\vdots & & \ddots & 0 & 1 & (-1)^{n-2} \\
0 & \cdots & \cdots & 0 & 0 & 1 + (-1)^{n-1}
\end{pmatrix}
$$
当$n$为偶数时,此时化简后的系数矩阵最后一行全为零,故$A\begin{pmatrix}
\lambda_1 \\ \vdots \\ \lambda_n
\end{pmatrix} = 0$有非零解,此时$\alpha_1+\alpha_2, \alpha_2+\alpha_3, \alpha_n+\alpha_1$一定线性相关。

当$n$为奇数时,化简后的系数矩阵满秩,此时$\alpha_1+\alpha_2, \alpha_2+\alpha_3, \cdots, \alpha_n+\alpha_1$线性相关当且仅当$\alpha_1, \alpha_2, \cdots, \alpha_n$线性相关。

\newpageorvspace

{\bf 习题2.1 第8题}. 设复数域上的向量$\alpha_1, \ldots, \alpha_n$线性无关。$\lambda$取什么复数值的时候,向量$\alpha_1 - \lambda \alpha_2, \alpha_2 - \lambda \alpha_3, \ldots, \alpha_{n-1} - \lambda \alpha_n, \alpha_n - \lambda \alpha_1$线性无关?

\newpageorvspace

{\bf 解}: 类似上一题习题2.1 第7题,将问题写为
$$(\alpha_1, \cdots, \alpha_n)
\underbrace{\begin{pmatrix}
1 & 0 & \cdots & \cdots & 0 & -\lambda \\
-\lambda & 1 & 0 & & & 0 \\
0 & -\lambda & 1 & \ddots & & \vdots \\
\vdots & \ddots & \ddots & \ddots & \ddots & \vdots \\
\vdots & & \ddots & \ddots & \ddots & 0 \\
0 & \cdots & \cdots & 0 & -\lambda & 1
\end{pmatrix}
\begin{pmatrix}
z_1 \\ \vdots \\ z_n
\end{pmatrix}
}_{\text{记为} Az}= 0
$$
$z_1, \cdots, z_n \in \mathbb{C}$.对系数矩阵$A$进行从上往下进行上一行乘以$\lambda$加到下一行的操作,有
$$
A = \begin{pmatrix}
1 & 0 & \cdots & \cdots & 0 & -\lambda \\
-\lambda & 1 & 0 & & & 0 \\
0 & -\lambda & 1 & \ddots & & \vdots \\
\vdots & \ddots & \ddots & \ddots & \ddots & \vdots \\
\vdots & & \ddots & \ddots & \ddots & 0 \\
0 & \cdots & \cdots & 0 & -\lambda & 1
\end{pmatrix}
\to
\begin{pmatrix}
1 & 0 & \cdots & \cdots & 0 & -\lambda \\
0 & 1 & 0 & & & -\lambda^2 \\
0 & 0 & 1 & \ddots & & 1 \\
\vdots & \ddots & \ddots & \ddots & \ddots & \vdots \\
\vdots & & \ddots & 0 & 1 & -\lambda\cdot\lambda^{n-2} \\
0 & \cdots & \cdots & 0 & 0 & 1-\lambda\cdot\lambda^{n-1}
\end{pmatrix}
$$
若$\alpha_1,\cdots,\alpha_n$线性无关,则只需要$1-\lambda^{n} \neq 0$,即$\lambda$不取$n$次单位根$e^{2k\pi i/n},$ $k = 0, \ldots, n-1,$ 即可使$\alpha_1-\lambda\alpha_2,\cdots,\alpha_n-\lambda\alpha_1$线性无关。

\newpageorvspace

{\bf 习题2.5 第4题}. 设向量组$S = \{ \alpha_1, \ldots, \alpha_s \}$线性无关,并且可以由向量组$T = \{ \beta_1, \ldots, \beta_t \}$线性表出。求证:

(1) 向量组$T$与向量组$S \cup T$等价。

(2) 将$S$扩充为$S \cup T$的一个极大线性无关组$T_1 = \{ \alpha_1, \ldots, \alpha_s, \beta_{i_1}, \ldots, \beta_{i_k} \},$ 则$T_1$与$T$线性等价,且$s+k \leqslant t.$

(3) (Steinitz替换定理) 可以用向量$\alpha_1, \ldots, \alpha_s$替换向量$\beta_1, \ldots, \beta_t$中某$s$个向量$\beta_{i_1}, \ldots, \beta_{i_s},$ 使得得到的向量组$\{ \alpha_1, \ldots, \alpha_s, \beta_{i_{s+1}}, \ldots, \beta_{i_t} \}$与$\{ \beta_1, \ldots, \beta_t \}$等价。

\newpageorvspace

{\bf 证明}: (1) 首先,向量组$S\cup T$显然能线性表出它的子集$T$. 另一方面,由于向量组$T$能线性表出向量组$S,$ 所以向量组$T = T\cup T$能线性表出向量组$S\cup T.$ 于是向量组$T$与向量组$S \cup T$能够相互线性表出,从而是线性等价的。

(2) \textbf{\color{red}注意!这里$T_1$不是$T$的子集!}

由线性等价的传递性,我们只要证明向量组$T_1$与向量组$S \cup T$线性等价即可。由于$T_1$是$S \cup T$的子集,我们只要证明$S \cup T$能被$T_1$线性表出即可。更进一步,只要证明任取向量$\gamma \in (S \cup T) \setminus T_1,$ 有$\gamma$能被$T_1$中的向量线性表出即可。假设不然,则存在不全为0的数$\lambda_0, \lambda_1, \ldots, \lambda_{s+k},$ 使得
$$0 = \lambda_0 \gamma + \lambda_1 \alpha_1 + \cdots + \lambda_s \alpha_s + \lambda_{s+1} \beta_{i_1} + \cdots + \lambda_{s+k} \beta_{i_k}$$
$\lambda_0$不能为0,否则会与$T_1$是线性无关组矛盾,于是
$$\gamma = -\dfrac{\lambda_1}{\lambda_0} \alpha_1 - \cdots - -\dfrac{\lambda_s}{\lambda_0} \alpha_s - \dfrac{\lambda_{s+1}}{\lambda_0} \beta_{i_1} - \cdots - \dfrac{\lambda_{s+k}}{\lambda_0} \beta_{i_k}$$
于是我们就证明了$T_1$, $S \cup T$, $T$是线性等价的,那么它们的秩就都相等
$$s+k = \operatorname{rank}(T_1) = \operatorname{rank}(T) \leqslant \# T = t.$$

(3) 我们利用第(2)问中的结论,将$S$扩充为$S \cup T$的一个极大线性无关组$T_1$, 设其元素个数为$s+k.$ 我们从$T$中不在$T_1$中的向量(总共有$t - k \geqslant s$个)随意挑出$s$个,那么剩下的向量与$S$并在一起得到的向量组,记作$T_2,$就满足
$$T_1 \subseteq T_2 \subseteq S \cup T.$$
因为$T_1$是$S \cup T$的极大线性无关组,所以$T_1, T_2, S \cup T$三者线性等价,从而与$T$线性等价。

\end{document}
