
% \renewcommand{\newpageorvspace}{\newpage}
\renewcommand{\newpageorvspace}{\vspace{2em}}

\date{2021-11-12  第四次习题课}

\begin{document}

\maketitle

{\bf 第1题}. 讨论参数$\lambda$决定矩阵$A(\lambda) = \begin{pmatrix} 1 & \cdots & 1 & \lambda \\ 1 & \cdots & \lambda & 1 \\ \vdots & \reflectbox{$\ddots$} & \vdots & \vdots \\ \lambda & \cdots & 1 & 1 \end{pmatrix}$的秩

{\bf 解:}


\newpageorvspace

{\bf 第2题}. 设$V$是复数域$\mathbb{C}$上的$n$维线性空间,$\alpha_1,\cdots,\alpha_n$是一组基。证明$V$也是实数域$\mathbb{R}$上的$2n$维线性空间,并求出它的一组基。更一般地,设$V$是数域$\mathbb{F}$上的$n$维线性空间,$\mathbb{F}$是子域$\mathbb{F}_0$的$m$维线性空间,问$V$是域$\mathbb{F}_0$上多少维线性空间,如何确定它的一组基?

{\bf 解:}


\newpageorvspace

{\bf 第3题}. 设$V = \mathbb{F}^n$, $W = \left\{ \begin{pmatrix} a_1 \\ \vdots \\ a_n \end{pmatrix} \ \mid|\ a_1+a_2=a_2-a_3=0 \right\}$. 求商空间$V / W$的维数与一组基。

{\bf 解:}


\newpageorvspace

{\bf 第4题}. 设$\mathbb{R}^2$中三条不同直线的方程为
\begin{align*}
    \ell_1: & \quad ax+by+c=0 \\
    \ell_2: & \quad cx+ay+b=0 \\
    \ell_3: & \quad bx+cy+a=0 \\
\end{align*}
证明$\ell_1,\ell_2,\ell_3$交于一点 $\Longleftrightarrow$ $a+b+c=0$.

{\bf 解:}


\end{document}
