
\date{2022-11-18 第六次习题课}
\author{}

\begin{document}

\maketitle

% \larger[2]

{\bf 第一题} (习题4.3第2题(3)) 设$A$是方阵,$A^k = 0$对某个正整数$k$成立。求证下列方阵可逆,并分别求他们的逆
$$I + A + \dfrac{1}{2!}A^2 + \cdots + \dfrac{1}{(k-1)!}A^{k-1}.$$

\ifIncludeAnswer

\newpageorvspace

{\bf 解}: 我们知道函数$f(x) = e^x$在$x = 0$附近有展开式
$$f(x) = 1 + x + \dfrac{1}{2!}x^2 + \cdots + \dfrac{1}{(k-1)!}x^{k-1} + \cdots$$
那么我们可以类比考虑$f(-A) = I + (-A) + \dfrac{1}{2!}(-A)^2 + \cdots + \dfrac{1}{(k-1)!}(-A)^{k-1},$ 看它与$I + A + \dfrac{1}{2!}A^2 + \cdots + \dfrac{1}{(k-1)!}A^{k-1}$相乘是否等于单位阵$I.$ 我们有
\begin{align*}
\sum\limits_{i=0}^{k-1} \dfrac{1}{i!} A^i \sum\limits_{j=0}^{k-1} \dfrac{1}{j!} (-A)^j & = \dfrac{1}{0!0!} I + \dfrac{1}{0!1!} A + \dfrac{1}{0!2!} A^2 + \dfrac{1}{0!3!} A^3 + \cdots + \dfrac{1}{0!(k-1)!} A^{k-1} \\
& \hspace{4em} + \dfrac{-1}{1!0!} A + \dfrac{-1}{1!1!} A^2 + \dfrac{-1}{1!2!} A^3 + \cdots + \dfrac{-1}{1!(k-2)!} A^{k-1} \\
& \hspace{8em} + \dfrac{1}{2!0!} A^2 + \dfrac{1}{2!1!} A^3 + \cdots + \dfrac{1}{2!(k-3)!} A^{k-1} \\
& \hspace{22em} \vdots \\
& \hspace{19em} + \dfrac{(-1)^{k-1}}{(k-1)!0!} A^{k-1}
\end{align*}
可以看到,当$1\leqslant n \leqslant k-1$时,上式右边$A^n$的系数等于
$$\sum\limits_{i=0}^n \dfrac{(-1)^i}{i!(n-i)!} = \dfrac{1}{n!} \sum\limits_{i=0}^n C_n^i 1^{n-i} (-1)^i = \dfrac{1}{n!} (1-1)^n = 0.$$
于是有$\sum\limits_{i=0}^{k-1} \dfrac{1}{i!} A^i \sum\limits_{j=0}^{k-1} \dfrac{1}{j!} (-A)^j = \dfrac{1}{0!0!} I = I.$ 所以$\sum\limits_{i=0}^{k-1} \dfrac{1}{i!} A^i$是可逆的,它的逆就是 $\sum\limits_{j=0}^{k-1} \dfrac{1}{j!} (-A)^j.$

事实上,我们有所谓的方阵函数$f: M_n(\mathbb{F}) \to M_n(\mathbb{F})$, 可以从多项式扩展定义到一般的(复)函数(由所谓的代表多项式定义),例如$\sin, \cos, \exp, \log$等。这些方阵函数保留了很多原来函数的性质。这部分内容在学习完方阵的Jordan标准形之后比较好理解,所以这里暂时不展开讲了。这题的关键就在于利用实值函数的展开式,猜测一个逆,然后用矩阵的乘法去验证。

\fi  % fi of \ifIncludeAnswer

\newpageorvspace

{\bf 第二题} (习题4.7第5题) 设$A \in \mathbb{F}^{m\times n}, \operatorname{rank} A = r.$
\begin{itemize}
\item[(1)] 从$A$中任意取出$s$行组成$s\times n$矩阵$B,$ 证明:$\operatorname{rank} B \geqslant r + s - m;$
\item[(2)] 从$A$中任意指定$s$个行和$t$个列,这些行和列的交叉位置的元组成的$s\times t$矩阵记为$D,$ 求证:$\operatorname{rank} D \geqslant r + s + t - m - n.$
\end{itemize}

\ifIncludeAnswer

\newpageorvspace

\textbf{证明}: (1) 我们记这$s$行的下标为$i_1, \ldots, i_s$, 余下的行的下标为$i_{s+1}, \ldots, i_m.$

方法一:考虑$m-s+1$个行向量组$\{v_{i_1}, \ldots, v_{i_s}\}, \{v_{i_1}, \ldots, v_{i_s}, v_{i_{s+1}}\}, \ldots, \{v_{i_1}, \ldots, v_{i_s}, \ldots, v_{i_m}\},$ 其中$v$为$A$的行向量。记这些向量组的秩为$r_0, \ldots, r_{m-s},$ 那么
$$r_0 = \operatorname{rank} B, r_{m-s} = \operatorname{rank} A = r, \text{ 并且有 } r_{k+1} - r_k \leqslant 1, ~ k = 0, \ldots, m-s-1,$$
其中等号成立当且仅当$v_{s+k+1}$落在前一个向量组张成的空间中。所以有
$$r - \operatorname{rank} B = r_{m-s} - r_0 = \sum\limits_{k=0}^{m-s-1} (r_{k+1} - r_k) \leqslant \sum\limits_{k=0}^{m-s-1} 1 = m-s.$$

方法二:令$P$为$m$阶单位阵删去第$i_{s+1}, \ldots, i_m$行组成的$s\times m$矩阵,那么有$PA = B,$ 且容易看出$\operatorname{rank} P = s.$ 根据之前证明的矩阵乘积的秩关系
$$\operatorname{rank} PA \geqslant \operatorname{rank} P + \operatorname{rank} A - m$$
即有
$$\operatorname{rank} B \geqslant s + r - m.$$

(2) 令$B$为从$A$中取出这$s$行构成的$s\times n$矩阵,那么根据(1)中结论,有
$$\operatorname{rank} B \geqslant r+s-m.$$
从矩阵$B$中取对应题设的$t$列,可以得到矩阵$D.$ 那么将(1)中结论应用到$B^T$与$D^T,$ 会有
$$\operatorname{rank} D = \operatorname{rank} D^T \geqslant \operatorname{rank} B^T + t - n = \operatorname{rank} B + t - n \geqslant s + r - m + t - n.$$


\fi  % fi of \ifIncludeAnswer

\newpageorvspace

{\bf 第三题} (习题4.7第9题) 设$A \in \mathbb{F}^{m\times n},$ 求证:$\operatorname{rank}(I_m - AA^T) - \operatorname{rank}(I_n - A^TA) = m - n.$

\ifIncludeAnswer

\newpageorvspace

\textbf{证明}:容易想到$I_m - AA^T$与$I_n - A^TA$都是某一个(同一个)分块矩阵进行了不同的``初等行列变换''得到的某些位置上的矩阵。考虑分块矩阵$M = \begin{pmatrix} A & I_m \\ I_n & A^T \end{pmatrix}.$ 一方面我们以$I_n$为``中心''保持不变做``初等行列变换''有
$$\begin{pmatrix} I_m & -A \\ 0 & I_n \end{pmatrix} \begin{pmatrix} A & I_m \\ I_n & A^T \end{pmatrix} \begin{pmatrix} I_n & -A^T \\ 0 & I_m \end{pmatrix} = \begin{pmatrix} I_m & -A \\ 0 & I_n \end{pmatrix} \begin{pmatrix} A & I_m - AA^T \\ I_n & 0 \end{pmatrix} = \begin{pmatrix} 0 & I_m - AA^T \\ I_n & 0 \end{pmatrix},$$
从而有$\operatorname{rank} M = \operatorname{rank} \begin{pmatrix} 0 & I_m - AA^T \\ I_n & 0 \end{pmatrix} = n + \operatorname{rank} (I_m - AA^T).$

另一方面以$I_m$为``中心''保持不变做``初等行列变换''有
$$\begin{pmatrix} I_m & 0 \\ -A^T & I_n \end{pmatrix} \begin{pmatrix} A & I_m \\ I_n & A^T \end{pmatrix} \begin{pmatrix} I_n & 0 \\ -A & I_m \end{pmatrix} = \begin{pmatrix} I_m & 0 \\ -A^T & I_n \end{pmatrix} \begin{pmatrix} 0 & I_m \\ I_n - A^TA & A^T \end{pmatrix} = \begin{pmatrix} 0 & I_m \\ I_n - A^TA & 0 \end{pmatrix},$$
从而有$\operatorname{rank} M = \operatorname{rank} \begin{pmatrix} 0 & I_m \\ I_n - A^TA & 0 \end{pmatrix} = m + \operatorname{rank} (I_n - A^TA),$ 进而有
$$\operatorname{rank}M = m + \operatorname{rank} (I_n - A^TA) = n + \operatorname{rank} (I_m - AA^T),$$
即有
$$\operatorname{rank}(I_m - AA^T) - \operatorname{rank}(I_n - A^TA) = m - n.$$
类似的从一个``中间分块矩阵''通过不同的``初等行列变换''组合得到我们想要的不同形式的矩阵的方法,在之前我们证明Sherman--Morrison formula,以及Woodbury matrix identity的时候已经介绍过了,这里又应用了一次。

\fi  % fi of \ifIncludeAnswer

\newpageorvspace

{\bf 第四题} (习题4.7第10题) 矩阵的广义逆。
\begin{itemize}
\item[(1)] 对任意矩阵$A \in \mathbb{F}^{m\times n},$ 存在矩阵$A^- \in \mathbb{F}^{n\times m}$满足条件$AA^-A = A.$ 什么条件下$A^-$由$A$唯一决定?($A^-$称为$A$的广义逆(generalized inverse matrix)。)
\item[(2)] 设$A \in \mathbb{F}^{m\times n}, \beta \in \mathbb{F}^{m\times 1}, A^- \in \mathbb{F}^{n\times m}$满足条件$AA^-A = A.$ 求证:\\
线性方程组$AX = \beta$有解的充分必要条件是$AA^-\beta = \beta;$\\
方程组有解时的通解为$X = A^-\beta + (I_n - A^-A)Y, \forall Y \in \mathbb{F}^{n\times 1}.$
\end{itemize}

\ifIncludeAnswer

\newpageorvspace

\textbf{证明}:(1) 我们先证明广义逆$A^-$的存在性。考虑$A$的相抵标准形,即设$P,Q$分别为$m$阶与$n$阶可逆阵,使得$A = P \begin{pmatrix} I_r & 0 \\ 0 & 0 \end{pmatrix}_{m\times n}\!\!\!\!\!\!\!\!\!\!\! Q$, 那么
$$A^- = Q^{-1} \begin{pmatrix} I_r & * \\ *' & *'' \end{pmatrix}_{n\times m}\!\!\!\!\!\!\!\!\!\!\! P^{-1}$$
其中$r = \operatorname{rank} A$, $*,*',*''$分别是大小为$r \times (m-r)$, $(n-r) \times r$, 以及$(n-r) \times (m-r)$的块,块中元素可以任取。那么很容易验证满足$AA^-A = A.$ 从以上广义逆$A^-$的形式可以看出,一个矩阵$A$的广义逆并不唯一。要使得$A^-$由$A$唯一决定,只有让$*,*',*''$都不存在,即$m - r = n - r = 0,$ 即$A$为可逆方阵。

\vspace{0.5em}

广义逆这个名字指的是``像是逆''的一个矩阵:
$$(AA^-)A = A = A(A^-A)$$
$AA^-$与$A^-A$都``像是''单位阵(左、右)作用于$A$上。他们分别在$A$的列空间与行空间上是恒等变换。这个虽然弱于在全空间上是恒等变换,但已经是比较强的条件了(比不变子空间强很多)。

\vspace{0.5em}

(2) $\Rightarrow:$ 若线性方程组$AX = \beta$有解,任取一个解,记为$X_0,$ 那么有
$$AA^-\beta = AA^-(AX_0) = (AA^-A)X_0 = AX_0 = \beta.$$

$\Leftarrow:$ 若$AA^-\beta = \beta,$ 那么令$X_0 = A^-\beta,$ 我们会有
$$AX_0 = A(A^-\beta) = AA^-\beta = \beta,$$
即知$X_0 = A^-\beta$为线性方程组$AX = \beta$的一个解。

这一问比较好理解,就是我们知道$AX = \beta$有解,当且仅当$\beta$属于$A$的列空间$\operatorname{col}(A),$ 而上面已经提到了,$AA^-$在$A$的列空间上是恒等变换,自然会把$\beta$映成$\beta.$

要证明线性方程组$AX = \beta$有解时的通解为$X = A^-\beta + (I - A^-A)Y, \forall Y \in \mathbb{F}^{m\times 1},$ 只要证明对应的齐次线性方程组$AX = 0$的解的全体为$X = (I_n - A^-A)Y, \forall Y \in \mathbb{F}^{n\times 1}.$

由于广义逆$A^-$满足$AA^-A = A,$ 即$A(I_n - A^-A) = 0,$ 所以
$$\operatorname{col}(I_n - A^-A) \subseteq \operatorname{Null} (A),$$
其中$\operatorname{col}(I_n - A^-A)$为矩阵$I_n - A^-A$的列空间,即由$I_n - A^-A$的列张成的线性空间,$\operatorname{Null} (A)$为矩阵$A$的零化空间,即$AX = 0$的解空间。要证明以上包含关系实际上是相等的关系,我们需要考察他们的维数。

沿用第(1)问的记号,我们有
$$A^-A = Q^{-1} \begin{pmatrix} I_r & * \\ *' & *'' \end{pmatrix}_{n\times m}\!\!\!\!\!\!\!\!\!\!\! P^{-1} P \begin{pmatrix} I_r & 0 \\ 0 & 0 \end{pmatrix}_{m\times n}\!\!\!\!\!\!\!\!\!\!\! Q = Q^{-1} \begin{pmatrix} I_r & 0 \\ *' & 0 \end{pmatrix}_{n\times n}\!\!\!\!\!\!\!\!\!\! Q.$$
所以
$$\dim \left( \operatorname{col}(I_n - A^-A) \right) = \operatorname{rank} (I_n - A^-A) = \operatorname{rank} \left( Q^{-1} \begin{pmatrix} 0 & 0 \\ -*' & I_{n-r} \end{pmatrix}_{n\times n}\!\!\!\!\!\!\!\!\!\! Q \hspace{1em} \right) = n - r = \dim \left( \operatorname{Null} (A) \right).$$
也就是说,我们实质上有$$\{ (I_n - A^-A)Y ~|~ Y \in \mathbb{F}^{n\times 1} \} = \operatorname{col}(I_n - A^-A) = \operatorname{Null} (A).$$
所以齐次线性方程组$AX = 0$的解的全体为$X = (I_n - A^-A)Y, \forall Y \in \mathbb{F}^{n\times 1}.$

\vspace{0.5em}

以上是当我们取定了某一个广义逆$A^-$时,对线性方程组$AX = \beta$解集的刻画。实际上,当$\beta \neq 0$时,我们还有另一种刻画,即非齐次线性方程组$AX = \beta$解集等于
$$\{ A^-\beta ~|~ A^- \text{ 是$A$的广义逆 } \}.$$
我们已经知道了形如$A^-\beta$的向量是$AX = \beta$的解,这里的$A^-$是矩阵$A$的任何一个广义逆。我们只要证明对$AX = \beta$的任何一个解$X_0$都存在某个广义逆$A^-,$ 使得$X_0 = A^-\beta.$

我们利用$A$的相抵标准形以及对应的$A^-$的一般形式。设$E_1, E_2, E_3$分别为元素未定的大小为$r \times (m-r)$, $(n-r) \times r$, 以及$(n-r) \times (m-r)$的矩阵。我们想要求解下列的方程
$$Q^{-1} \begin{pmatrix} I_r & E_1 \\ E_2 & E_3 \end{pmatrix} P^{-1} \beta = X_0.$$
我们已知的是$AX_0 = \beta,$ 即$P \begin{pmatrix} I_r & 0 \\ 0 & 0 \end{pmatrix} Q X_0 = \beta.$ 记$X_0' = QX_0, \beta' = P^{-1}\beta,$ 那么我们有
$$\begin{pmatrix} I_r & 0 \\ 0 & 0 \end{pmatrix} X_0' = \beta', \quad \text{要求解矩阵方程} \begin{pmatrix} I_r & E_1 \\ E_2 & E_3 \end{pmatrix} \beta' = X_0'.$$
即求解
$$X_0' = \begin{pmatrix} I_r & E_1 \\ E_2 & E_3 \end{pmatrix} \begin{pmatrix} I_r & 0 \\ 0 & 0 \end{pmatrix} X_0' = \begin{pmatrix} I_r & 0 \\ E_2 & 0 \end{pmatrix} X_0'.$$
令$X_0' = \begin{pmatrix} v_1 \\ v_2 \end{pmatrix},$ 其中$v_1 \in \mathbb{F}^{r \times 1}, v_2 \in \mathbb{F}^{(n-s)\times 1},$ 有
$$\beta' = \begin{pmatrix} I_r & 0 \\ 0 & 0 \end{pmatrix} \begin{pmatrix} v_1 \\ v_2 \end{pmatrix} = \begin{pmatrix} v_1 \\ 0 \end{pmatrix}, \quad \text{相应矩阵方程化为} \begin{pmatrix} v_1 \\ v_2 \end{pmatrix} = \begin{pmatrix} I_r & 0 \\ E_2 & 0 \end{pmatrix} \begin{pmatrix} v_1 \\ v_2 \end{pmatrix} = \begin{pmatrix} v_1 \\ E_2v_1 \end{pmatrix}.$$
由于我们假设了$\beta \neq 0,$ 即有$\beta' \neq 0,$ 那么$v_1 \neq 0$. 设$v_1$的第$i$位元素等于$a \in \mathbb{F}, a \neq 0,$ 那么取
$$E_2 = (0, \ldots, 0, \underbrace{\dfrac{1}{a}v_2}_{\text{第$i$列}}, 0, \ldots, 0),$$
即可以满足$E_2v_1 = v_2.$ 其余的$E_1, E_3$任取,即可满足
$$\begin{pmatrix} I_r & E_1 \\ E_2 & E_3 \end{pmatrix} \beta' = \begin{pmatrix} I_r & E_1 \\ E_2 & E_3 \end{pmatrix} \begin{pmatrix} v_1 \\ 0 \end{pmatrix} = \begin{pmatrix} v_1 \\ E_2v_1 \end{pmatrix} = \begin{pmatrix} v_1 \\ v_2 \end{pmatrix} = X_0'.$$
也就是说,对于非齐次线性方程组$AX = \beta$的任取的一个解$X_0,$ 我们都找到了相应的$A$的广义逆
$$A^- = Q^{-1} \begin{pmatrix} I_r & E_1 \\ E_2 & E_3 \end{pmatrix} P^{-1}$$
使得$X_0 = A^-\beta.$

\vspace{1em}

由于广义逆并不唯一,我们可以在$AA^-A = A$以外多加一些条件($A^-AA^- = A^-,$ $AA^-$与$A^-A$都对称),得到满足唯一性的Moore–Penrose广义逆(或伪逆,pseudoinverse)$A^+.$ $A^+$的构造可以由$A$的奇异值分解引出。$A^+$可以用于求解最小二乘问题。

\fi  % fi of \ifIncludeAnswer

\newpageorvspace

{\bf 第五题} (习题5.1第4题) (综合除法) 设$f(x) = a_nx^n + a_{n-1}x^{n-1} + \cdots + a_1x + a_0$是数域$\mathbb{F}$上的多项式,$c\in\mathbb{F}$. 求证:$x-c$除$f(x)$的商$q(x) = b_{n-1}x^{n-1} + \cdots + b_1x + b_0$和余式$r$可以用如下的算法得出
$$
\begin{array}{c|cccccccc}
c & & a_n & a_{n-1} & \cdots & a_i & \cdots & a_1 & a_0 \\
& +) & & cb_{n-1} & \cdots & cb_i & \cdots & cb_1 & cb_0 \\ \hline
& & b_{n-1} & b_{n-1} & \cdots & b_{i-1} & \cdots & b_0 & r
\end{array}
$$
其中$b_{n-1} = a_n$, $b_{i-1} = a_i + cb_i \ (\forall 1 \leqslant i \leqslant n)$, $r = a_0 + cb_0$.

\ifIncludeAnswer

\newpageorvspace

\textbf{证明}:

考虑$f(x) = (x-c)g(x) + r$, 即
\begin{align*}
a_nx^n + a_{n-1}x^{n-1} + \cdots + a_1x + a_0 = & \ (x-c) (b_{n-1}x^{n-1} + \cdots + b_1x + b_0) + r \\
= & \ b_{n-1}x^{n} + \cdots + b_1x^2 + b_0x + r \\
& \ - cb_{n-1}x^{n-1} - \cdots - cb_1x - cb_0
\end{align*}
移项之后即有
$$a_nx^n + (a_{n-1}+cb_{n-1})x^{n-1} + \cdots + (a_1+cb_1)x + (a_0 + cb_0) = b_{n-1}x^{n} + \cdots + b_1x^2 + b_0x + r,$$
对应次项的系数相等,从而有$b_{n-1} = a_n, b_{n-2} = a_{n-1}+cb_{n-1}, \ldots, b_0 = a_1+cb_1, r = a_0 + cb_0$.

\fi  % fi of \ifIncludeAnswer


\end{document}
