
% \renewcommand{\newpageorvspace}{\newpage}
\renewcommand{\newpageorvspace}{\vspace{2em}}

\date{2021-12-24  第七次习题课}

\begin{document}

\maketitle

{\bf 习题5.4 第10题}. 求证方程$x^3 - x^2 - \frac{1}{2}x - \frac{1}{6} = 0$的根不可能全为实数。

{\bf 证明}:对于一个一般的$n$次多项式$f(x) = a_{n}x^{n}+a_{n-1}x^{n-1}+\cdots +a_{1}x+a_{0}$, 其判别式(discriminant)被定义为
$$\Delta = a_n^{2n-2} \prod_{i< j} (r_i - r_j)^2$$
其中$r_1,\ldots, r_n$为$f(x)$在其分裂域(或者某个代数闭域)上的$n$个根。对于3次实多项式$f(x) = ax^3+bx^2+cx_1+d$来说,如果它有3个实根,那么显然它的判别式是一个非负实数;如果他有复根$\beta$, 那么$\beta$的复共轭$\bar{\beta}$也是它的复根。设它的另一个实根为$\alpha$,那么它的判别式为
$$\Delta = a^4 (\alpha-\beta)^2(\alpha-\bar{\beta})^2(\beta-\bar{\beta})^2 = a^4 (\alpha^2-\alpha(\beta+\bar{\beta})+\beta\bar{\beta})^2 (\beta-\bar{\beta})^2$$
因为$\alpha^2-\alpha(\beta+\bar{\beta})+\beta\bar{\beta}$是实数,$\beta-\bar{\beta}$是纯虚数,所以判别式在这种情况下必然是一个负实数。

关于多项式判别式的计算,有如下结果
$$\Delta = \frac{(-1)^{n(n-1)/2}}{a_n} R(f,f')$$
其中$R(f,f')$为$f$与其导函数$f'$的结式。对于一般的三次方程,其判别式为
$$\Delta = b^{2}c^{2}-4ac^{3}-4b^{3}d-27a^{2}d^{2}+18abcd$$
对于本题的$x^3 - x^2 - \frac{1}{2}x - \frac{1}{6}$, 其判别式为
$$(-1)^2(-\frac{1}{2})^2 - 4(-\frac{1}{2})^3 - 4(-1)^3(-\frac{1}{6}) - 27(-\frac{1}{6})^2 + 18(-1)(-\frac{1}{2})(-\frac{1}{6}) = -\frac{13}{6} < 0$$
或者做变量替换$x\to x+\frac{1}{3}$得到$x^{3} - \frac{5 x}{6} - \frac{11}{27}$,这个多项式的判别式的计算更简单
$$\Delta = -4(-\frac{5}{6})^3 - 27 (\frac{11}{27})^2 = -\frac{13}{6} < 0.$$
所以它有一个实根,两个共轭的复根。

\newpageorvspace


{\bf 习题5.4 第11题}. 分别在复数域、实数域上将下列多项式分解为不可约多项式的乘积:(4) $x^{2n} + x^n + 1$.

{\bf 解}. 多项式方程$y^2 + y + 1 = 0$的复根为$y = e^{2\pi i/3}, e^{4\pi i/3}$, 所以在复数域上有
$$x^{2n} + x^n + 1 = \prod_{k=0}^{n-1} \left(x - e^{\frac{\theta_1 i}{n} + \frac{2k\pi i}{n}}\right) \left(x - e^{\frac{\theta_2 i}{n} + \frac{2k\pi i}{n}}\right), \quad \theta_1 = \frac{2\pi}{3}, \theta_2 = \frac{4\pi}{3}.$$
要求该多项式在实数域上的分解,只要将其复数域上的分解式中共轭的一次式拿到一起,得到实系数的二次式即可。对于任意$0\leqslant k \leqslant n-1$, 取$k' = n-k-1$, 即有$0\leqslant k' \leqslant n-1$, 且$(\frac{\theta_1 i}{n} + \frac{2k\pi i}{n}) + (\frac{\theta_2 i}{n} + \frac{2k'\pi i}{n}) = 2\pi i,$ 即$e^{\frac{\theta_1 i}{n} + \frac{2k\pi i}{n}}$与$e^{\frac{\theta_2 i}{n} + \frac{2k'\pi i}{n}}$互为复共轭,即有
$$
\begin{cases}
e^{\frac{\theta_1 i}{n} + \frac{2k\pi i}{n}} + e^{\frac{\theta_2 i}{n} + \frac{2k'\pi i}{n}} = \cos(\frac{\theta_1 i}{n} + \frac{2k\pi i}{n}) \\
e^{\frac{\theta_1 i}{n} + \frac{2k\pi i}{n}} \cdot e^{\frac{\theta_2 i}{n} + \frac{2k'\pi i}{n}} = 1
\end{cases}
$$
所以$x^{2n} + x^n + 1$在实数域上的分解为
$$x^{2n} + x^n + 1 = \prod_{k=0}^{n-1} \left( x^2 -2 \cos \left( \frac{\theta_1 i}{n} + \frac{2k\pi i}{n} \right) + 1 \right)$$

\newpageorvspace


{\bf 第三题}. 设正整数$m,n$满足$(m,n)=1$, 用$\Omega_m, \Omega_n$分别表示$x^m-1$与$x^n-1$在$\mathbb{C}$内的全部根的集合。证明$\Omega_m \cap \Omega_n = \{1\}$.

{\bf 证明}:正整数$m,n$满足$(m,n)=1$, 则存在整数$a,b\in\mathbb{Z}$, 使得$am+bn=1$. 任取$\alpha\in \Omega_m \cap \Omega_n$, 有
$$\alpha^m = \alpha^n = 1,$$
那么$\alpha^{am} = \alpha^{bn} = 1,$ 进而有
$$\alpha = \alpha^{am+bn} = \alpha^{am} \cdot \alpha^{bn} = 1 \cdot 1 = 1$$

另一种证明方法:$\Omega_m = \{ e^{\frac{2\pi k}{m}} \ |\ 0\leqslant k \leqslant m-1 \}$, $\Omega_n = \{ e^{\frac{2\pi k}{n}} \ |\ 0\leqslant k \leqslant n-1 \}$, 那么只要证明任取$0 < k < m$, $0 < k' < n$, 有$e^{\frac{2\pi k}{m}} \neq e^{\frac{2\pi k'}{n}}$即可。在这种情况下,只要证明$\frac{k}{m} \neq \frac{k'}{n}$即可。用反证法,假设$kn=k'm$, 由于$(m,n)=1$, 所以必须有
$$m|k, n|k'$$
同时成立,但这是不可能的。


\newpageorvspace


{\bf 习题5.5 第2题}. 整系数多项式$f(x)$能否同时满足$f(10)=10, f(20)=20, f(30)=40$?

{\bf 解.} 假设$f(x) = a_nx^n + \cdots + a_1x + a_0$满足题设条件。任取$\alpha,\beta\in\mathbb{Z}$, 有
\begin{align*}
f(\alpha) - f(\beta) & = a_n(\alpha^n-\beta^n) + \cdots + a_1(\alpha-\beta) \\
& = a_n(\alpha-\beta)(\alpha^{n-1}+\alpha^{n-2}\beta+\cdots+\alpha\beta^{n-2}+\beta^{n-1}) + \cdots + a_1(\alpha-\beta),
\end{align*}
所以$(\alpha-\beta) \ |\ (f(\alpha) - f(\beta))$, 但题目中有
$$(30-10) \nmid (f(30)-f(10)).$$

如果是有理系数的话,则肯定有这样的多项式:Lagrange插值多项式
\begin{align*}
f(x) & = 10\frac{(x-20)(x-30)}{(10-20)(10-30)} + 20\frac{(x-10)(x-30)}{(20-10)(20-30)} + 40\frac{(x-10)(x-20)}{(30-10)(30-20)} \\
& = \frac{x^2}{20} - \frac{x}{2} + 10
\end{align*}
即是满足$f(10)=10, f(20)=20, f(30)=40$的有理系数多项式。


\newpageorvspace


{\bf 习题5.5 第8题}. 设$a,b,n$都是非零整数,$n\geqslant 2$, 且$ab\ |\ (a+b)^n$. 求证$b|a^{n-1}$.

{\bf 证明.} 由于
$$ab\ |\ a^n + C_n^1a^{n-1}b + C_n^2a^{n-2}b^2 + \cdots + C_n^{n-1}ab^{n-1} + b^n$$
且右边前$n$项都被$a$整除,所以第$n+1$项$b^n$也被$a$整除,即存在整数$h$使得$b^n = ah$. 故有
$$b\ |\ a^{n-1} + C_n^1a^{n-2}b + \cdots + C_n^{n-1}b^{n-1} + h,$$
从此式能推出$b\ |\ a^{n-1} + h$. 也就是说,我们有
$$
\begin{cases}
b\ |\ a^{n-1} + h \\
b^n = ah
\end{cases}
$$
设$b$的素因子分解为有限乘积$b = \pm \prod p_i^{k_i}$, $p_i$为素数,$k_i>0$. 那么上面第一式告诉我们,要么$p_i^{k_i}$同时整除$a^{n-1}$与$h$,要么都不整除$a^{n-1}$与$h$. 上面第一式告诉我们,都不整除$a^{n-1}$与$h$的$p_i^{k_i}$是不存在的。于是$b$同时整除除$a^{n-1}$与$h$.

以上,我们排除的是$a^{n-1}$与$h$各含有$b$的部分素因子的情况。

另一种证法:考虑与$\frac{a^{n-1}}{b}$对称的$\frac{b^{n-1}}{a}$, 并构造有理系数多项式
$$f(x) = \left(x-\frac{a^{n-1}}{b}\right) \left(x-\frac{b^{n-1}}{a}\right) = x^2 + \frac{a^n+b^n}{ab} x + a^{n-2}b^{n-2}$$
由于
$$ab\ |\ a^n + C_n^1a^{n-1}b + C_n^2a^{n-2}b^2 + \cdots + C_n^{n-1}ab^{n-1} + b^n,$$
故$ab\ |\ a^n + b^n,$ 所以$f(x)$是整系数多项式。由有理根定理知道$f(x)$的有理根的既约分数形式的分母整除$f(x)$的最高次项系数1,故$f(x)$的有理根实际上都是整根,即$\frac{a^{n-1}}{b}$与$\frac{b^{n-1}}{a}$都是整数。


\end{document}
