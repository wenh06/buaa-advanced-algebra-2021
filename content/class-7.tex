
% \renewcommand{\newpageorvspace}{\newpage}
\renewcommand{\newpageorvspace}{\vspace{2em}}

\date{2021-12-24  第七次习题课}

\begin{document}

\maketitle

{\bf 习题5.4 第10题}. 


\newpageorvspace


{\bf 习题5.4 第11题}. (4)


\newpageorvspace



{\bf 第三题}. 设正整数$m,n$满足$(m,n)=1$, 用$\Omega_m, \Omega_n$分别表示$x^m-1$与$x^n-1$在$\mathbb{C}$内的全部根的集合。证明$\Omega_m \cap \Omega_n = \{1\}$.

{\bf 证明}:正整数$m,n$满足$(m,n)=1$, 则存在整数$a,b\in\mathbb{Z}$, 使得$am+bn=1$. 任取$\alpha\in \Omega_m \cap \Omega_n$, 有
$$\alpha^m = \alpha^n = 1,$$
那么$\alpha^{am} = \alpha^{bn} = 1,$ 进而有
$$\alpha = \alpha^{am+bn} = \alpha^{am} \cdot \alpha^{bn} = 1 \cdot 1 = 1$$

另一种证明方法:$\Omega_m = \{ e^{\frac{2\pi k}{m}} \ |\ 0\leqslant k \leqslant m-1 \}$, $\Omega_n = \{ e^{\frac{2\pi k}{n}} \ |\ 0\leqslant k \leqslant n-1 \}$, 那么只要证明任取$0 < k < m$, $0 < k' < n$, 有$e^{\frac{2\pi k}{m}} \neq e^{\frac{2\pi k'}{n}}$即可。


\newpageorvspace


{\bf 习题5.5 第2题}. 


\newpageorvspace


{\bf 习题5.5 第8题}. 


\newpageorvspace


\end{document}
