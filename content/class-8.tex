
\date{2022-12-16 第八次习题课}
\author{}

\begin{document}

\maketitle

\ifLargeLayout
\larger[2]
\fi

{\bf 第一题} (习题5.6第1题) 证明:二元多项式$x^2 + y^2 - 1$在任意数域上都不可约。

\ifIncludeAnswer

\newpageorvspace

{\bf 证明}: 我们用反证法,假设二元多项式$f(x, y) = x^2 + y^2 - 1$可约,那么由于$\deg f = 2,$ 它能分解成两个一次多项式的乘积,设为
\ifLargeLayout
\begin{align*}
f(x, y) & = (a_1x + b_1y + c_1)(a_2x + b_2y + c_2) \\
& = a_1a_2 x^2 + b_1b_2 y^2 + c_1c_2 + (a_1b_2 + a_2b_1)xy + (a_1c_2 + a_2c_1)x \\
& \phantom{=} + (b_1c_2 + b_2c_1)y
\end{align*}
\else  % else of \ifLargeLayout
\begin{align*}
f(x, y) & = (a_1x + b_1y + c_1)(a_2x + b_2y + c_2) \\
& = a_1a_2 x^2 + b_1b_2 y^2 + c_1c_2 + (a_1b_2 + a_2b_1)xy + (a_1c_2 + a_2c_1)x + (b_1c_2 + b_2c_1)y
\end{align*}
\fi  % fi of \ifLargeLayout
比较系数有
$$
\begin{cases}
1 = a_1a_2 \\
1 = b_1b_2 \\
-1 = c_1c_2 \\
0 = a_1b_2 + a_2b_1 \\
0 = a_1c_2 + a_2c_1 \\
0 = b_1c_2 + b_2c_1
\end{cases}
$$
后三个等式可变形为
$$
\begin{cases}
a_1b_2 = - a_2b_1 \\
a_2c_1 = - a_1c_2 \\
b_1c_2 = - b_2c_1
\end{cases}
$$
将以上三个等式乘在一起有$a_1a_2b_1b_2c_1c_2 = (-1)^3 a_1a_2b_1b_2c_1c_2,$ 进而有$-1 = 1,$ 不可能成立。

\fi  % fi of \ifIncludeAnswer

\newpageorvspace

{\bf 第二题} (习题5.7第1题) 设$a, b, c$都是实数。求证$a, b, c$都是正数的充分必要条件是:$a + b + c > 0,$ $ab + ac + bc > 0,$ $abc > 0.$

\vspace{0.5em}

Hint: 这题可以直接去做,也可以利用Descartes' Rule of Signs:一般地,设$f(x) = a_n x^n + a_{n-1} x^{n-1} + \cdots + a_1 x + a_0$为$n$次实系数多项式,$a_n \neq 0.$ (一般还假设$a_0 \neq 0,$ 否则可以化为更低次的情况) 记$f(x)$所有非零系数按相应的幂次降幂排列的序列记为
$$b_m, b_{m-1}, \ldots, b_0,$$
相应的幂次记为
$$d_m > d_{m-1} > \cdots > d_0.$$
定义
\begin{align*}
s(f) & := \sum\limits_{i=1}^{m} \dfrac{1}{2} \left\lvert \dfrac{|b_{i}|}{b_{i}} - \dfrac{|b_{i-1}|}{b_{i-1}} \right\rvert \\
& = \# \{ i ~|~ 1 \leqslant i \leqslant m, ~ b_{i}b_{i-1} < 0 \} \\
p(f) & := \text{$f(x)$的正实根数量}
\end{align*}
(重根计算重数)那么
$$2 | (s(f) - p(f)) \geqslant 0.$$
这里的$s(f)$实际上就是$f(x)$的非零系数序列$b_m, b_{m-1}, \ldots, b_0$相邻两项符号的变化次数。

将此结论应用到多项式$f(-x)$上,可得出$f(x)$负实根数目的结论:
$$2 | (s'(f) - p'(f)) \geqslant 0,$$
\begin{align*}
s'(f) & := \sum\limits_{i=1}^{m} \dfrac{1}{2} \left\lvert (-1)^{d_{i}}\dfrac{|b_{i}|}{b_{i}} - (-1)^{d_{i-1}}\dfrac{|b_{i-1}|}{b_{i-1}} \right\rvert \\
& = \# \{ i ~|~ 1 \leqslant i \leqslant m, ~ (-1)^{d_i+d_{i-1}}b_{i}b_{i-1} < 0 \} \\
p'(f) & := \text{$f(x)$的负实根数量}
\end{align*}

\ifIncludeAnswer

\newpageorvspace

{\bf 证明}: 先假定Descartes' Rule of Signs成立,并将其应用到此题。假设$a, b, c$都是实数,但不知道正负。现在有的条件是$a + b + c > 0,$ $ab + ac + bc > 0,$ $abc > 0.$ 考察多项式
$$f(x) = (x-a)(x-b)(x-c) = x^3 - (a+b+c)x^2 + (ab+ac+bc)x -abc.$$
其非零系数按降幂排列为
$$1, - (a+b+c), ab+ac+bc, -abc,$$
这个序列的符号为
$$+, ~ -, ~ +, ~-,$$
变号次数为3,从而知其正实根数目为3或1. 而考察序列
$$1\cdot(-1)^3, - (a+b+c)\cdot(-1)^2, (ab+ac+bc)\cdot(-1)^1, -abc\cdot(-1)^0$$
即序列
$$-1, - (a+b+c), -(ab+ac+bc), -abc$$
相邻两项的符号变化,知$f(x)$的负实根个数为0. 所以在题设情况下,$f(x)$的根$a, b, c$都是正的。于是充分性得证。必要性是显然的。

\vspace{0.5em}

下面我们对$f$的次数用归纳法证明Descartes' Rule of Signs.

$n = 1$时,$f(x) = a_1 x + a_0.$ 我们不考虑$a_0 = 0$的平凡情况。当$s(f) = 1$时,即$a_1, a_0$异号时,$f(x) = 0$的根$-a_0 / a_1$是正实数,有$p(f) = 1.$ $s(f) = 0$时,$a_1, a_0$同号,$f(x) = 0$的根$-a_0 / a_1$是负实数,有$p(f) = 0.$ 于是,$n = \deg(f) = 1$时,总有$s(f) - p(f) = 0.$

假设对所有次数$\leqslant n$的实系数多项式$f,$ 都有$2 | s(f) - p(f) \geqslant 0$成立。现任取一个$n+1$次实系数多项式
$$f(x) = a_{n+1}x^{n+1} + a_nx^n + \cdots + a_1x + a_0, ~~ a_{n+1}, a_0 \neq 0.$$
那么$g(x) = a_nx^n + \cdots + a_1x + a_0$就是一个次数$\leqslant n$的多项式,根据归纳假设,首先有
$$2 | s(g) - p(g).$$
我们不妨设$a_n \neq 0.$ 由于$s(g) - p(g)$是一个偶数,那么$s(g), p(g)$必须同时是偶数或者同时是奇数。我们分情况讨论。

情况1. $s(g), p(g)$同时为偶数。那么由于$s(g)$为偶数,即序列$a_n, \ldots, a_1, a_0$变号次数为偶数,因此$a_n$必须与$a_0$同号。

情况1.1 $a_{n+1}, a_n, a_0$都是正实数。由于$a_{n+1}, a_n$同号,所以$s(f) = s(g).$ 由于最高次项系数$a_{n+1}$与常数项$a_0$同号,所以$f(x)$的正实根数目$p(f)$为偶数(随后证明)。于是$2 | s(f) - p(f).$

情况1.2 $a_{n+1}$是负实数,$a_n, a_0$都是正实数。此时$a_{n+1}, a_n$异号,$s(f) = s(g) + 1$为奇数。由于最高次项系数$a_{n+1}$与常数项$a_0$异号,$f(x)$的正实根数目$p(f)$为奇数。于是依然有$2 | s(f) - p(f).$

情况1.3 $a_{n+1}$是正实数,$a_n, a_0$都是负实数;情况1.4 $a_{n+1}$是负实数,$a_n, a_0$都是负实数;以及情况2. $s(g), p(g)$同时为奇数的相应的4种情况都能类似证明有$2 | s(f) - p(f).$

现在我们来证明,若$n+1$次多项式$f(x)$的最高次项系数$a_{n+1}$与常数项$a_0$同号,则$f(x)$的正实根数目为偶数。将$f(x)$的互不相等的正实根按从小到大排列为
$$0 < \lambda_1 < \lambda_2 < \cdots < \lambda_m,$$
对应的重数分别为$n_1, n_2, \ldots, n_m.$ 容易看出在$x \in [0, +\infty)$内,$f(x)$的值只有可能在$\lambda_1, \ldots, \lambda_m$附近变号。对$1 \leqslant i \leqslant m$有
$$f(x) = (x - \lambda_i)^{n_i} g_i(x),$$
$g_i(x)$在$\lambda_i$的某个小邻域内非零(从而是恒正或者恒负的)。若$n_i$为偶数,那么在这个小领域内,$f(x)$的值恒非负或者恒非正,不变号。类似可知,若$n_i$为奇数,则多项式$f(x)$在$\lambda_i$的某个小邻域内变号一次。$f(x)$的最高次项系数$a_{n+1}$与常数项$a_0$同号,所以要么$f(0) < 0, \lim\limits_{x\to+\infty} f(x) = -\infty$ (对应$a_{n+1}, a_0$同负) 或者$f(0) > 0, \lim\limits_{x\to+\infty} f(x) = +\infty$ (对应$a_{n+1}, a_0$同正)。那么$f(x)$的值在$x \in [0, +\infty)$内需要变号偶数次。也就是说$n_1, n_2, \ldots, n_m$中的奇数的数目为一个偶数。所以
$$p(f) = n_1 + n_2 + \cdots + n_m$$
是一个偶数。同理可证,若$n+1$次多项式$f(x)$的最高次项系数$a_{n+1}$与常数项$a_0$异号,则$f(x)$的正实根数目为奇数。

最后,我们需要证明$s(f) - p(f) \geqslant 0.$ 考虑$f(x)$的微商$f'(x).$ 它是一个$n$次实系数多项式,而且系数符号都不变(除了$a_0$变为0)。于是$s(f) = s(f') + 1,$ 或者$s(f) = s(f').$ 另一方面,由于$f(x)$有$p(f)$个正实根。当其中有$m > 1$个互异实根,它们构成了$m - 1$个闭区间。由微分中值定理,这些区间上都至少有一个点满足$f'(x) = 0.$ 而对于其中的每个$k$重根($k > 1$)都是$f'(x)$的至少$k-1$重根。于是有$p(f') \geqslant p(f) - 1.$ 当$p(f) \leqslant 1$时,$p(f') \geqslant p(f) - 1$平凡成立。对$f'(x)$用归纳假设,有
$$s(f) \geqslant s(f') \geqslant p(f') \geqslant p(f) - 1,$$
即$s(f) - p(f) \geqslant -1.$ 由于$s(f) - p(f)$为偶数,所以上述不等式实际上为$s(f) - p(f) \geqslant 0.$

\vspace{1em}

另一种做法:必要性显然。我们只要证明充分性。首先,因为有$abc > 0,$ 于是$a, b, c$中的负实数数目为偶数。若此数目为零,则证明完毕。下面我们只要证明$a, b, c$中有两个负实数的情况不可能成立即可。

我们用反证法,不妨设$a, b < 0, c > 0.$ 由于$a + b + c > 0,$ 因此有
$$c > -(a + b) > 0,$$
那么
$$(a + b)c < -(a + b)^2 < 0,$$
进而有
$$ab + (a + b)c < ab - (a + b)^2 = -\dfrac{(a + b)^2}{2} - \dfrac{3}{4}(a^2 + b^2) < 0.$$
这与条件$ab + ac + bc > 0$矛盾。因此$a, b, c$中不可能有两个负实数。因此充分性得证。

\fi  % fi of \ifIncludeAnswer

\newpageorvspace

{\bf 第三题} (习题5.7第2题) 设$1, \omega_1, \ldots, \omega_{n-1}$是$x^n - 1,$ $n \geqslant 2,$ 的全部不同的复数根。求证:
$$(1 - \omega_1)(1 - \omega_2) \cdots (1 - \omega_{n-1}) = n$$

\ifIncludeAnswer

\newpageorvspace

{\bf 证明}: 令$f(x) = x^n - 1 = (x - \omega_0)(x - \omega_1) \cdots (x - \omega_{n-1}),$ 其中约定$\omega_0 = 1.$ 那么$f(x)$的微商为
$$f'(x) = nx^{n-1} = \sum\limits_{i=0}^{n-1} \prod\limits_{\substack{0 \leqslant j < n \\ j \neq i}} (x - \omega_j).$$
在上式中取$x = 1,$ 由于上式右边的和式中,$i \neq 0$的乘积中都有$(x - 1)$这一项,所以
$$n = \prod\limits_{\substack{0 \leqslant j < n \\ j \neq i}} (1 - \omega_j) = \prod\limits_{0 < j < n } (x - \omega_j) = (1 - \omega_1)(1 - \omega_2) \cdots (1 - \omega_{n-1}).$$

\fi  % fi of \ifIncludeAnswer

\newpageorvspace

{\bf 第四题} 设$f(x) \in \mathbb{C}, f(x) | f(x^n), n \in \mathbb{Z}^+.$ 证明$f(x)$的根是零或者单位根。

\ifIncludeAnswer

\newpageorvspace

{\bf 证明}: 不妨设$f(x)$的最高次项次数为$m,$ 系数为$1.$ 那么$f(x)$可以在$\mathbb{C}[x]$内分解为1次式乘积
$$f(x) = \prod\limits_{i=1}^m (x - \lambda_i),$$
其中$\lambda_1, \ldots, \lambda_m \in \mathbb{C}$为$f(x)$在$\mathbb{C}$内的$m$个根。那么对于$f(x^n),$ 我们有
$$f(x^n) = \prod\limits_{i=1}^m (x^n - \lambda_i) = \prod\limits_{i=1}^m \prod\limits_{j=1}^n (x - \mu_{ij}),$$
其中$\mu_{ij}$是$z^n = \lambda_i$在$\mathbb{C}$内的$m$个根。

如果有$f(x) | f(x^n),$ 即$\left.\prod\limits_{i=1}^m (x - \lambda_i) \middle| \prod\limits_{i=1}^m \prod\limits_{j=1}^n (x - \mu_{ij}) \right.,$ 那么$\forall i = 1, \ldots, m,$ 都存在下标$1 \leqslant k(i) \leqslant m, 1 \leqslant j(i) \leqslant n,$ 使得$\lambda_i = \mu_{k(i)j(i)}.$ (并且$(k(1), j(1)), \ldots, (k(m), j(m))$是$m$个互异的二元组)

取定了这$m$个二元组之后,$\forall i = 1, \ldots, m,$ 如果$k_i = i,$ 那么我们有$\lambda_i = \mu_{ij(i)}.$ 两边同时$n$次方之后即有
$\lambda_i^n = \mu_{ij(i)}^n = \lambda_i,$ 此时必有$\lambda_i$为零或为$n - 1$次单位根。

若$k_i \neq i,$ 我们可以考虑序列
$$k^{(0)}(i) := i, ~~ k^{(1)}(i) := k(i), ~~ k^{(2)}(i) := k(k(i)), ~~ \ldots$$
此序列前$m+1$个元素内必然出现重复的元素:
\begin{figure}[H]
\centering
\begin{tikzpicture}
\node[circle, fill, inner sep=2] at (0, -1) (p0) {};
\node[above = 0.1 of p0.north] {$k^{(0)}(i)$};
\node[circle, fill, inner sep=2] at (3, 0) (p1) {};
\node[above=0.1 of p1.north] {$k^{(1)}(i)$};
\node[circle, fill, inner sep=2] at (6, 0) (p2) {};
\node[right = 0.1 of p2.east] {$k^{(a)}(i) = k^{(b)}(i)$};
\node[circle, fill, inner sep=2] at (7.5, -2) (p3) {};
\node[right = 0.1 of p3.east] {$k^{(a+1)}(i)$};
\node[circle, fill, inner sep=2] at (6.5, -3.5) (p4) {};
\node[right = 0.1 of p4.east] {$k^{(a+2)}(i)$};
\node[circle, fill, inner sep=2] at (5, -3.5) (p5) {};
\node[below = 0.1 of p5.south] {$k^{(a+3)}(i)$};
\node[circle, fill, inner sep=2] at (3.5, -1.5) (p6) {};
\node[left = 0.1 of p6.west] {$k^{(b-1)}(i)$};
\draw (p0) -- (p1);
\draw[dotted, very thick] (p1) -- (p2);
\draw (p2) -- (p3) -- (p4) -- (p5);
\draw[dotted, very thick] (p5) -- (p6);
\draw (p6) -- (p2);
\end{tikzpicture}
\end{figure}
不妨设有$k^{(a)}(i) = k^{(b)}(i),$ $0 \leqslant a < b \leqslant m+1,$ 那么会有
$$\mu_{k^{(b)}(i)j^{(b)}(i)} = \lambda_{k^{(b-1)}(i)} = \mu_{k^{(b-1)}(i)j^{(b-1)}(i)}^n = \cdots = \lambda_{k^{(a)}(i)}^{n(b-a-1)}.$$
($j^{(*)}(i)$的定义类似$k^{(*)}(i)$) 两边同时$n$次方之后有
$$\lambda_{k^{(a)}(i)}^{n(b-a)} = \mu_{k^{(b)}(i)j^{(b)}(i)}^n = \lambda_{k^{(b)}(i)} = \lambda_{k^{(a)}(i)},$$
也就是说$\lambda_{k^{(a)}(i)}$是零或者$n(b-a) - 1$次单位根。但是我们又有
$$\lambda_{k^{(a)}(i)} = \mu_{k^{(a)}(i)j^{(a)}(i)}^n = \lambda_{k^{(a-1)}(i)}^n = \mu_{k^{(a-1)}(i)j^{(a-1)}(i)}^{2n} = \cdots = \lambda_{i}^{an},$$
即$\lambda_{i}$的$an$次幂是零或者某个单位根,那么它本身就是零或者(另一个)单位根。

\fi  % fi of \ifIncludeAnswer

\newpageorvspace

{\bf 第五题} (习题5.4 第10题) 求证方程$x^3 - x^2 - \frac{1}{2}x - \frac{1}{6} = 0$的根不可能全为实数。

\vspace{0.5em}

Hint: 我们再来回顾一下上次课讲的系数在某个域$\mathbb{F}$内的$n$次多项式
$$
f(x) = a_{n}x^{n} + a_{n-1}x^{n-1} + \cdots + a_{0}, ~ a_n \neq 0,
$$
判别式的定义
$$\operatorname{Disc}_x(f) = a_n^{2n-2} \prod_{1 \leqslant i < j \leqslant n} (\lambda_i - \lambda_j)^2 \in \mathbb{F}.$$
这里的$\lambda_1, \ldots, \lambda_n$为$f(x)$在代数闭域$\bar{\mathbb{F}}$内的根。它可以通过$f$与其微商$f'$的结式$R(f,f')$来计算:
$$\operatorname{Disc}_x(f) = \frac{(-1)^{n(n-1)/2}}{a_n} R(f,f')$$

对于一个一般形式的三次多项式$ax^3 + bx^2 + cx + d,$ 它的判别式等于
$$b^{2}c^{2}-4ac^{3}-4b^{3}d-27a^{2}d^{2}+18abcd.$$
通过变量替换,三次多项式一般可以化为$x^3 + px + q$的形式,而它的判别式等于
$$-4p^3 - 27q^2.$$

\ifIncludeAnswer

\newpageorvspace

{\bf 证明}:对于一个一般的$n$次多项式$f(x) = a_{n}x^{n}+a_{n-1}x^{n-1}+\cdots +a_{1}x+a_{0}$, 其判别式为
$$\operatorname{Disc}_x(f) = a_n^{2n-2} \prod_{1 \leqslant i < j \leqslant n} (\lambda_i - \lambda_j)^2 \in \mathbb{F}.$$
这里的$\lambda_1, \ldots, \lambda_n$为$f(x)$在代数闭域$\bar{\mathbb{F}}$内的根。对于3次实多项式$f(x) = ax^3+bx^2+cx_1+d$来说,如果它有3个实根,那么显然它的判别式是一个非负实数;如果他有复根$\beta$, 那么$\beta$的复共轭$\bar{\beta}$也是它的复根。设它的另一个实根为$\alpha$,那么它的判别式为
$$\Delta = a^4 (\alpha-\beta)^2(\alpha-\bar{\beta})^2(\beta-\bar{\beta})^2 = a^4 (\alpha^2-\alpha(\beta+\bar{\beta})+\beta\bar{\beta})^2 (\beta-\bar{\beta})^2$$
因为$\alpha^2-\alpha(\beta+\bar{\beta})+\beta\bar{\beta}$是实数,$\beta-\bar{\beta}$是纯虚数,所以判别式在这种情况下必然是一个负实数。

对于一般的三次方程,其判别式为
$$\Delta = b^{2}c^{2}-4ac^{3}-4b^{3}d-27a^{2}d^{2}+18abcd$$
对于本题的$x^3 - x^2 - \frac{1}{2}x - \frac{1}{6}$, 其判别式为
\ifLargeLayout
\begin{multline*}
(-1)^2(-\frac{1}{2})^2 - 4(-\frac{1}{2})^3 - 4(-1)^3(-\frac{1}{6}) - 27(-\frac{1}{6})^2 + 18(-1)(-\frac{1}{2})(-\frac{1}{6}) \\
= -\frac{13}{6} < 0
\end{multline*}
\else
$$(-1)^2(-\frac{1}{2})^2 - 4(-\frac{1}{2})^3 - 4(-1)^3(-\frac{1}{6}) - 27(-\frac{1}{6})^2 + 18(-1)(-\frac{1}{2})(-\frac{1}{6}) = -\frac{13}{6} < 0$$
\fi  % fi of LargeLayout
或者做变量替换$x\to x+\frac{1}{3}$得到$x^{3} - \frac{5}{6}x - \frac{11}{27}$,这个多项式的判别式的计算更简单
$$\Delta = -4(-\frac{5}{6})^3 - 27 (\frac{11}{27})^2 = -\frac{13}{6} < 0.$$
所以它有一个实根,两个共轭的复根。

\fi  % fi of \ifIncludeAnswer

% \newpageorvspace

% {\bf 第六题} 

% \ifIncludeAnswer

% \newpageorvspace

% {\bf 解}: 

% \fi  % fi of \ifIncludeAnswer

\end{document}
