
%%%%%%%%%%%%%%%%%%%%%%%%%%%%%%%%%%%%%%%%
%           main body
%%%%%%%%%%%%%%%%%%%%%%%%%%%%%%%%%%%%%%%%


%-----------------------------------
%	TITLE PAGE
%-----------------------------------

\title[Pseudoinverse and Least Squares]{\LARGE \bfseries 伪逆与最小二乘法} % The short title appears at the bottom of every slide, the full title is only on the title page
\author[] % Your institution as it will appear on the bottom of every slide, may be shorthand to save space
{\Large \bfseries Pseudoinverse and Least Squares}
% \author{Singular Value Decomposition} % Your name
% \date{\today} % Date, can be changed to a custom date
\date{}

%------------------------------------------------

\begin{document}


%------------------------------------------------

\begin{frame}

\titlepage % Print the title page as the first slide

\end{frame}

%------------------------------------------------

\begin{frame}

考虑解线性方程组$A\mathbf{x} = \mathbf{b}$的问题。我们知道当$A$是可逆方阵的时候,该线性方程组有唯一解$\mathbf{x} = A^{-1}\mathbf{b}$.

\vspace{1em}

当$A$没有逆的时候,有什么补救办法呢?

\vspace{1em}

希望:定义一个伪逆(Pseudoinverse),记作$A^+$, 使得当$A$可逆的时候,$A$的伪逆就是它的逆,即
$$A^+ = A^{-1}$$
如果可能的话,还要求当$A\mathbf{x} = \mathbf{b}$无解时,$\mathbf{x} = A^+\mathbf{b}$是``近似程度''最好的``解''。

\end{frame}

%------------------------------------------------

\begin{frame}
\frametitle{内容提要} % Table of contents slide, comment this block out to remove it
\tableofcontents % Throughout your presentation, if you choose to use \section{} and \subsection{} commands, these will automatically be printed on this slide as an overview of your presentation
\end{frame}

%-----------------------------------
%	PRESENTATION SLIDES
%-----------------------------------

\section{广义逆}

%------------------------------------------------

\subsection{广义逆的定义与性质}

%------------------------------------------------

\begin{frame}

\begin{block}{矩阵的广义逆,Generalized Inverse}
$m\times n$矩阵$A$的广义逆指的是一个$n\times m$的矩阵$X$, 满足
$$AXA = A$$
通常把$A$的广义逆记作$A^-$.
\end{block}

\pause

\begin{block}{广义逆的普遍存在性}
任意矩阵$A$的广义逆$A^-$总存在。设$P,Q$分别为$m$阶与$n$阶可逆阵,使得$A = P \begin{pmatrix} I_r & 0 \\ 0 & 0 \end{pmatrix} Q$, 那么
$$A^- = Q^{-1} \begin{pmatrix} I_r & * \\ *' & *'' \end{pmatrix} P^{-1}$$
其中$r = r(A)$, $*,*',*''$分别是大小为$r \times (n-r)$, $(m-r) \times r$, 以及$(m-r) \times (n-r)$的块,块中元素可以任取。
\end{block}

\end{frame}

%------------------------------------------------

\subsection{广义逆与线性方程组的解}

%------------------------------------------------

\begin{frame}

\begin{block}{广义逆与齐次线性方程组}
齐次线性方程组$A\mathbf{x} = \mathbf{0}$的解的全体为
$$\mathbf{x} = (I_n - A^-A)\mathbf{z},$$
其中$A^-$取遍$A$的所有广义逆,$\mathbf{z}$取遍所有的$n$维列向量。
\end{block}

\pause

\begin{block}{广义逆与非齐次线性方程组}
非齐次线性方程组$A\mathbf{x} = \mathbf{b}$有解,当且仅当存在$A$的广义逆$A^-$使得
$$\mathbf{b} = A A^- \mathbf{b}.$$
有解时,它的解的全体为$A^-\mathbf{b}$, $A^-$取遍$A$的所有广义逆。
\end{block}

\end{frame}

%------------------------------------------------

\begin{frame}

{
\large
广义逆只解决了我们的第一部分要求,即它``像是''逆,而且当矩阵的确可逆的时候它就是矩阵通常意义下的逆。但是它并不唯一,而且仍然没有解决求``最佳近似解''的需求。
}

\end{frame}

%------------------------------------------------

\section{Moore-Penrose伪逆}

%------------------------------------------------

\subsection{伪逆的定义与性质}

%------------------------------------------------

\begin{frame}

\begin{block}{Moore-Penrose伪逆,Pseudoinverse}
$m\times n$矩阵$A$的Moore-Penrose伪逆,或简称伪逆,指的是同时满足如下四个条件的$n\times m$矩阵$X$
\[
\begin{cases}
AXA = A \\
XAX = X \\
(AX)^T = AX \\
(XA)^T = XA
\end{cases}
\]
通常,$A$的伪逆被记作$A^+$.
\end{block}

\end{frame}

%------------------------------------------------

\begin{frame}

\begin{block}{伪逆的存在性与唯一性}
任意$m\times n$矩阵$A$的伪逆$A^+$都存在,而且
$$A^+ = V\Sigma^+U^T,$$
其中$A = U\Sigma V^T$为矩阵$A$的奇异值分解,奇异值为$\sigma_1, \ldots, \sigma_r$,
$$\Sigma^+ = \begin{pmatrix}
\frac{1}{\sigma_1} & & & \\ & \ddots & & \\ & & \frac{1}{\sigma_r} & \\ & & & \bigzero \end{pmatrix}_{n\times m}$$
\end{block}

\end{frame}

%------------------------------------------------

\begin{frame}

\begin{block}{伪逆的另一种表达}
$A$的奇异值分解又可以写作\(\displaystyle A = U\Sigma V^T = \sum\limits_{i=1}^r \sigma_i \mathbf{u}_i \mathbf{v}_i^T \). 于是我们又有$A^+$的另一种表达
$$A^+ = \sum\limits_{i=1}^r \sigma_i^{-1} \mathbf{v}_i \mathbf{u}_i^T$$
其中这些$\mathbf{v}_i, \mathbf{u}_i$分别是正交阵$V$与$U$的列。
\end{block}

\end{frame}

%------------------------------------------------

\subsection{伪逆与线性方程组}

%------------------------------------------------

\begin{frame}

\begin{block}{伪逆与线性方程组}
对于线性方程组$A\mathbf{x} = \mathbf{b}$($\mathbf{b}$可以是零向量也可以不是),它若有解,则解可以由系数矩阵$A$的Moore-Penrose伪逆统一给出:
$$A^+\mathbf{b} + (I - A^+A)\mathbf{w}$$
其中$\mathbf{w}$是任意一个$n$维列向量。当然,$A\mathbf{x} = \mathbf{b}$有解的充要条件为
$$AA^+\mathbf{b} = \mathbf{b}$$
\end{block}

\end{frame}

%------------------------------------------------

\section{最小二乘法}

%------------------------------------------------

\begin{frame}

\begin{block}{最小二乘法:问题分析}
我们来考虑非齐次线性方程组$A\mathbf{x} = \mathbf{b}$无解的情况,$A$是一个$m\times n$矩阵。此时,我们有
$$A\mathbf{x} = \mathbf{b}\text{无解} \Longleftrightarrow \mathbf{b}\not\in C(A)$$

转而:在$C(A)$中寻找$A\widehat{\mathbf{x}}$, 使得它与$\mathbf{b}$最接近,即求
$$\argmin_{\widehat{\mathbf{x}}}\Vert A\widehat{\mathbf{x}} - \mathbf{b} \Vert,$$
并称其为一个最小二乘解。

\vspace{2em}
可以将以上结果与之前的广义逆相关的结果对比来看。
\end{block}

\end{frame}

%------------------------------------------------

\begin{frame}

\begin{block}{最小二乘法:法方程组}
{\bfseries 事实}:$\mathbb{R}^m$中的一个点,与一个线性子空间中距离最短的点是这个点在这个子空间上的投影。于是
\begin{align*}
\Vert A\widehat{\mathbf{x}} - \mathbf{b} \Vert \text{取得最小值} & \Longleftrightarrow A\widehat{\mathbf{x}} - \mathbf{b} \perp C(A) \\
& \Longleftrightarrow A^T (A\widehat{\mathbf{x}} - \mathbf{b}) = \mathbf{0} \\
& \Longleftrightarrow A^TA\widehat{\mathbf{x}} = A^T\mathbf{b}
\end{align*}

于是问题转化为了求解非齐次线性方程组
$$A^TA\widehat{\mathbf{x}} = A^T\mathbf{b}$$

\vspace{1em}
\pause

以上方程组(被称作法方程组,Normal Equations)的特点:
\begin{itemize}
\item 总有解;
\item 即使解$\widehat{\mathbf{x}}$不唯一,投影$A\widehat{\mathbf{x}}$是唯一的。
\end{itemize}
\end{block}

\end{frame}

%------------------------------------------------

\begin{frame}

\begin{block}{最小二乘解}
\begin{itemize}
\item<1-> 若$r(A) = n$, 那么$r(A^T) = r(A) = n$. 此时$A^TA$可逆,$A^TA\widehat{\mathbf{x}} = A^T\mathbf{b}$有唯一解。于是我们得到唯一的最小二乘解
$$\widehat{\mathbf{x}} = (A^TA)^{-1}A^T\mathbf{b}.$$
\item<2-> 若$r(A) < n$, 那么$r(A^T) = r(A) < n$. 此时$A^TA$不可逆,$A^TA\widehat{\mathbf{x}} = A^T\mathbf{b}$有无穷多组解,亦即最小二乘解有无穷多组。
\end{itemize}
\end{block}

\end{frame}

%------------------------------------------------

\begin{frame}

\begin{block}{最小二乘解}
{\bfseries 重要}:$\mathbf{x}^+ := A^+\mathbf{b}$是长度最小的最小二乘解。其中$A^+$为矩阵$A$的Moore-Penrose伪逆。

\vspace{1em}
\pause

证明:以下设$A$的奇异值分解为$A = U\Sigma V^T$.

(1)$\mathbf{x}^+$是最小二乘解:将其代入法方程组$A^TA\widehat{\mathbf{x}} = A^T\mathbf{b}$左边有
\begin{align*}
A^TA\cdot\mathbf{x}^+ & = A^TAA^+ \cdot \mathbf{b} \\
& = (V\Sigma^TU^T)(U\Sigma V^T)(V\Sigma^+ U^T) \mathbf{b} = V\Sigma^T\Sigma\Sigma^+ U^T \cdot \mathbf{b} \\
& = V \begin{pmatrix} \Sigma_0^T & 0 \\ 0 & 0 \end{pmatrix} \begin{pmatrix} \Sigma_0 & 0 \\ 0 & 0 \end{pmatrix} \begin{pmatrix} \Sigma_0^{-1} & 0 \\ 0 & 0 \end{pmatrix} U^T \cdot \mathbf{b} \\
& = V \Sigma^T U^T \cdot \mathbf{b} = A^T \cdot \mathbf{b} = \text{法方程组右边}
\end{align*}
其中$\Sigma_0 = diag(\sigma_1, \ldots, \sigma_r)$. 由上式即知,$\mathbf{x}^+$是最小二乘解。
\end{block}

\end{frame}

%------------------------------------------------

\begin{frame}

\begin{block}{最小二乘解}
(2)$\mathbf{x}^+$是最小二乘解中长度最小者:设$\widehat{\mathbf{x}}_0$也是一个最小二乘解,即满足$A^TA\widehat{\mathbf{x}}_0 = A^T\mathbf{b}$. 于是
\begin{align*}
& A^TA(\widehat{\mathbf{x}}_0-\mathbf{x}^+) = A^T\mathbf{b} - A^T\mathbf{b} = \mathbf{0} \\
\Longrightarrow & \Vert A (\widehat{\mathbf{x}}_0-\mathbf{x}^+) \Vert = (\widehat{\mathbf{x}}_0-\mathbf{x}^+)^T A^TA (\widehat{\mathbf{x}}_0-\mathbf{x}^+) = 0 \\
\Longrightarrow & A (\widehat{\mathbf{x}}_0-\mathbf{x}^+) = \mathbf{0}
\end{align*}

\pause

$A$用其奇异值分解又可写作$A = \sum\limits_{i=1}^r \sigma_i \mathbf{u}_i \mathbf{v}_i^T$. 于是从上式我们有
\begin{align*}
A (\widehat{\mathbf{x}}_0-\mathbf{x}^+) = \mathbf{0} & \Longrightarrow \sum\limits_{i=1}^r \sigma_i \mathbf{u}_i \mathbf{v}_i^T (\widehat{\mathbf{x}}_0-\mathbf{x}^+) = \mathbf{0} \\
& \Longrightarrow \sum\limits_{i=1}^r \left( \sigma_i \cdot \left( \mathbf{v}_i^T (\widehat{\mathbf{x}}_0-\mathbf{x}^+) \right) \right) \mathbf{u}_i = \mathbf{0}
\end{align*}
\end{block}

\end{frame}

%------------------------------------------------

\begin{frame}

\begin{block}{最小二乘解}
但是,我们知道$\mathbf{u}_1,\ldots,\mathbf{u}_r$是线性无关的(甚至是相互正交的),而且$\sigma_1,\ldots,\sigma_r$都是大于0的实数。于是
$$\mathbf{v}_1^T (\widehat{\mathbf{x}}_0-\mathbf{x}^+) = \cdots = \mathbf{v}_r^T (\widehat{\mathbf{x}}_0-\mathbf{x}^+) = 0.$$
那么
\begin{align*}
\langle \mathbf{x}^+, \widehat{\mathbf{x}}_0 - \mathbf{x}^+ \rangle & = (\mathbf{x}^+)^T (\widehat{\mathbf{x}}_0 - \mathbf{x}^+) = (A^+\mathbf{b})^T (\widehat{\mathbf{x}}_0 - \mathbf{x}^+) \\
& = \mathbf{b}^T (A^+)^T (\widehat{\mathbf{x}}_0 - \mathbf{x}^+) \\
& = \mathbf{b}^T \left( \sum\limits_{i=1}^r \sigma_i^{-1} \mathbf{v}_i \mathbf{u}_i^T \right)^T (\widehat{\mathbf{x}}_0 - \mathbf{x}^+) \\
& = \mathbf{b}^T \left( \sum\limits_{i=1}^r \sigma_i^{-1} \mathbf{u}_i \underbrace{\mathbf{v}_i^T (\widehat{\mathbf{x}}_0 - \mathbf{x}^+)}_{=\mathbf{0}} \right) = 0
\end{align*}
\end{block}

\end{frame}

%------------------------------------------------

\begin{frame}

\begin{block}{最小二乘解}
因此,我们有
$$\Vert \widehat{\mathbf{x}}_0 \Vert^2 = \Vert \mathbf{x}^+ + (\widehat{\mathbf{x}}_0 - \mathbf{x}^+) \Vert^2 = \Vert \mathbf{x}^+ \Vert^2 + \Vert (\widehat{\mathbf{x}}_0 - \mathbf{x}^+) \Vert^2 \geqslant \Vert \mathbf{x}^+ \Vert^2$$
也就是说$\mathbf{x}^+ = A^+\mathbf{b}$的确是长度最小的最小二乘解。
\end{block}

\end{frame}

%------------------------------------------------

\begin{frame}

\begin{block}{最小二乘解}
{\color{red}注意}!最小二乘解$\mathbf{x}^+ = A^+\mathbf{b}$包含了$A^TA$可逆时的唯一最小二乘解$(A^TA)^{-1}A^T\mathbf{b}$.

\vspace{1em}
\pause

根据矩阵乘积的秩关系
$$A^TA\text{可逆} \Longrightarrow n = r(A^TA) = r(A) = r(\Sigma),$$
于是,$\Sigma$就必须形如$\Sigma = \begin{pmatrix} \Sigma_0 \\ 0 \end{pmatrix}$, 其中$\Sigma_0 = diag(\sigma_1, \ldots, \sigma_n)$. 于是
$$A^+ = V\begin{pmatrix} \Sigma_0^{-1} & 0 \end{pmatrix} U^T$$
另一方面,
\begin{align*}
(A^TA)^{-1}A^T & = (V\Sigma^TU^TU\Sigma V^T)^{-1} V\Sigma^TU^T \\
& = V \Sigma_0^{-2} V^T V\Sigma^TU^T = V\begin{pmatrix} \Sigma_0^{-1} & 0 \end{pmatrix} U^T
\end{align*}
\end{block}

\end{frame}

%------------------------------------------------

\end{document}
